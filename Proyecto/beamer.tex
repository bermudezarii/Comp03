\documentclass{beamer}
\usepackage{pgfplots}
\usepackage{xcolor}
\usepackage{listings}
\definecolor{mGreen}{rgb}{0,0.6,0}
\definecolor{mGray}{rgb}{0.5,0.5,0.5}
\definecolor{mPurple}{rgb}{0.58,0,0.82}
\definecolor{backgroundColour}{rgb}{0.95,0.95,0.92}
\lstdefinestyle{CStyle}{
  backgroundcolor=\color{backgroundColour},
  commentstyle=\color{mGreen},
  keywordstyle=\color{magenta},
  numberstyle=\tiny\color{mGray},
  stringstyle=\color{mPurple},
  basicstyle=\footnotesize,
  breakatwhitespace=false,
  breaklines=true,
  captionpos=b,
  keepspaces=true,
  numbersep=5pt,
  showspaces=false,
  showstringspaces=false,
  showtabs=false,
  tabsize=2,
  language=C 
}
\usetheme{progressbar}
 
 
\usecolortheme{crane}
 
 
\setbeamercolor{frametitle}{fg=brown}
 
 
\title{Analizador Sint\'actico}
\subtitle{Proyecto 1}
\author{Ariana Berm\'udez,Ximena Bola\~nos, Dylan Rodr\'iguez}
\institute{Instituto Tecnol\'ogico de Costa Rica}
\date{\today}
\begin{document}
\begin{frame}
 \titlepage 
 \end{frame}\begin{frame}
 \frametitle{An\'alisis Sint\'actico}
  Se hizo un analizador sint\'actico con la ayuda de la herramienta de Bison, para el lenguaje C y que corre en C, este analizador trabaja en conjunto con Flex, para tomar los tokens que este le otorga y revisar con las gram\'aticas que les sean ingresadas. \end{frame}\begin{frame}
 \frametitle{Bison}
 Bison convierte de una gram\'atica libre de contexto a un analizador sint\'actico que emplea las tablas de Parsing LALR(1), siendo: \begin{itemize} \item L: Left algo \item A: ... \item L: ... \item R: rightmost \item (1): donde este uno significa que tiene como lookahead solo un s\'imbolo. \end{itemize} Cabe destacar que Bison es compatible con Yacc. Sirve con C, C++ y Java. \end{frame}\begin{frame}[fragile]
\frametitle{C\'odigo}
\begin{lstlisting}[style=CStyle]


 

 
char e [ ] = "**3<HRZcir+3@OXakt;=GOXds*\?HRZcir*7HNZ`i19JS\\p*H[m1:CJSz*>H[`mr25\
\?Hx,P,B2Gs-KTfzRdv1SeyCR-ISeu.<Ev+9+P,z,4PfzIdvO2*HRZcir6GPis=MU*3HRZcir*HZi\
1JS\\epy*>H[m1JSey*DH[m*3<HZiu-@P*3HZi<N]q1JS\\epy:[m1CJSeny06I[m*4\?HRZcir,\
\?*6HRZcir1J]q2K*H[m2K*H[m2@K]qtO@M2DK]q,]q1JS\\epy[m1:JSey+[m*3<HRZcir13Gt,\
=GVs*3<HRZcir1J]qz*HF*AH2;DK]qua0=G2:K]q]q1CJS\\pDVu1:JS*D!+3:BIOSY`egilqtxz\
\177.0249<==>EJMUY`ejov#$59@CJOXYZbfhlnrxy&+.57=@IMR[``bcfmnq!#),@" , * f ; 
int main ( int j , char * k [ ] ) { 
int a , b , c , d , g , h , i = 19 ; 
printf ( "       " ) ; 
for ( g = 0 [ f = ( char * ) calloc ( 80 + ( h = atoi ( 1 [ k ] ) ) , 1 ) ] = 1 ; g <= h ; g ++ ) { 
if ( ( g > 30 ) && ( f [ i - 2 ] + f [ i - 1 ] != 0 ) ) i ++ ; 
for ( d = c = 0 ; d < i ; d ++ ) { 
 printf ( "%c" , c [ " 01./:;|\\" ] ) ;  f [ d ] = ( e [ b = c * 9 , b += ( c = d [ f ] ) , ( ( ( a = e [ 345 + b ] + b / 19 * 85 - 33 - b / 40 * 12 - b / 80 * 4 ) [ e ] 
- 42 ) / 9 - f [ d + 1 ] ) ?  ( ( e [ ++ a ] - 42) / 9 -f [ d + 1] ) ?   ( ( e [ ++ a ] - 42) / 9 -f [ d + 1] ) ?   ( ( e [ ++ a ] - 42) / 9 -f [ d + 1] ) ?   ( ( e [ ++ a ] - 42) / 9 -f [ d + 1] ) ?   ( ( e [ ++ a ] - 42) / 9 -f [ d + 1] ) ?   ( ( e [ ++ a ] - 42) / 9 -f [ d + 1] ) ?   ( ( e [ ++ a ] - 42) / 9 -f [ d + 1] ) ?   ( ( e [ ++ a ] - 42) / 9 -f [ d + 1] ) ?  0 : a : a : a : a : a : a : a : a : a ] - 42 ) % 9 ; } 
 printf ( "%c" , c [ " 01./:;|\\" ] ) ;  if ( 0 [ f ] - 1 ) printf ( "\n%6i " , g ) ; else printf ( "\n       " ) ; } 
} \end{lstlisting}
\end{frame}
\end{document}
