\documentclass{beamer}
\usepackage{pgfplots}
\usepackage{xcolor}
\usepackage{minted}
\definecolor{mGreen}{rgb}{0,0.6,0}
\definecolor{mGray}{rgb}{0.5,0.5,0.5}
\definecolor{mPurple}{rgb}{0.58,0,0.82}
\definecolor{backgroundColour}{rgb}{0.95,0.95,0.92}
\usetheme{progressbar}
 
 
\usecolortheme{crane}
 
 
\setbeamercolor{frametitle}{fg=brown}
 
 
\title{Analizador Sint\'actico}
\subtitle{Proyecto 1}
\author{Ariana Berm\'udez,Ximena Bola\~nos, Dylan Rodr\'iguez}
\institute{Instituto Tecnol\'ogico de Costa Rica}
\date{\today}
\begin{document}
\begin{frame}
 \titlepage 
 \end{frame}\begin{frame}
 \frametitle{An\'alisis Sint\'actico}
  Se hizo un analizador sint\'actico con la ayuda de la herramienta de Bison, para el lenguaje C y que corre en C, este analizador trabaja en conjunto con Flex, para tomar los tokens que este le otorga y revisar con las gram\'aticas que les sean ingresadas. \end{frame}\begin{frame}
 \frametitle{Bison}
 Bison convierte de una gram\'atica libre de contexto a un analizador sint\'actico que emplea las tablas de Parsing LALR(1), siendo: \begin{itemize} \item L: Left algo \item A: ... \item L: ... \item R: rightmost \item (1): donde este uno significa que tiene como lookahead solo un s\'imbolo. \end{itemize} Cabe destacar que Bison es compatible con Yacc. Sirve con C, C++ y Java. \end{frame}\begin{frame}[fragile]
\frametitle{C\'odigo}
\begin{minted}[fontsize=\tiny]{c}

for_each_string_list_item ( refname , refnames ) { 
if ( get_packed_ref ( refs , refname -> string ) ) { 
needs_repacking = 1 ; 
break ; 
} 
} \end{minted}
\end{frame}
\end{document}
