\documentclass{beamer}
\usepackage{pgfplots}
\usepackage{xcolor}
\usepackage{listings}
\definecolor{mGreen}{rgb}{0,0.6,0}
\definecolor{mGray}{rgb}{0.5,0.5,0.5}
\definecolor{mPurple}{rgb}{0.58,0,0.82}
\definecolor{backgroundColour}{rgb}{0.95,0.95,0.92}
\lstdefinestyle{CStyle}{
  backgroundcolor=\color{backgroundColour},
  commentstyle=\color{mGreen},
  keywordstyle=\color{magenta},
  numberstyle=\tiny\color{mGray},
  stringstyle=\color{mPurple},
  basicstyle=\footnotesize,
  breakatwhitespace=false,
  breaklines=true,
  captionpos=b,
  keepspaces=true,
  numbers=left,
  numbersep=5pt,
  showspaces=false,
  showstringspaces=false,
  showtabs=false,
  tabsize=2,
  language=C 
}
\usetheme{progressbar}
 
 
\usecolortheme{crane}
 
 
\setbeamercolor{frametitle}{fg=brown}
 
 
\title{Analizador L\'exico}
\subtitle{Proyecto 1}
\author{Ariana Berm\'udez,Ximena Bola\~nos, Dylan Rodr\'iguez}
\institute{Instituto Tecnol\'ogico de Costa Rica}
\date{\today}
\begin{document}
\begin{frame}
 \titlepage 
 \end{frame}\begin{frame}
 \frametitle{An\'alisis Sint\'actico}
  Se hizo un analizador sint\'actico con la ayuda de la herramienta de Bison, para el lenguaje C y que corre en C, este analizador trabaja en conjunto con Flex, para tomar los tokens que este le otorga y revisar con las gram\'aticas que les sean ingresadas.\end{frame}\begin{frame}
 \frametitle{Bison}
 jaajaj\end{frame}\begin{frame}[fragile]
\frametitle{C\'odigo Preprocesado (Sin Pretty Print)}
\begin{lstlisting}[style=CStyle]

gtk / gtk . h > 
stdlib . h > 
string . h > 
math . h > 
time . h > 
unistd . h > 
dirent . h > 
sys / types . h > 
sys / stat . h > 
libpq - fe . h > 
locale . h > 

struct \end{lstlisting}
\end{frame}
\begin{frame}[fragile]
\frametitle{C\'odigo Preprocesado (Sin Pretty Print)}
\begin{lstlisting}[style=CStyle]
PARAMETROS_EXAMINER parametros ; 


 


 

 

 

 


 

 

 
int N_preguntas = 0 ; 
int \end{lstlisting}
\end{frame}
\begin{frame}[fragile]
\frametitle{C\'odigo Preprocesado (Sin Pretty Print)}
\begin{lstlisting}[style=CStyle]
N_versiones = 0 ; 
int N_estudiantes = 0 ; 
int N_temas = 0 ; 
int N_subtemas = 0 ; 
int N_ajustes = 0 ; 
long double Nota_minima , Nota_maxima ; 
long double Nota_minima_ajustada , Nota_maxima_ajustada ; 


 int Frecuencias [  10 ] ; 


 


 

 \end{lstlisting}
\end{frame}
\begin{frame}[fragile]
\frametitle{C\'odigo Preprocesado (Sin Pretty Print)}
\begin{lstlisting}[style=CStyle]


 

 

 

 

 

 

 

 

 

 

 

 

 \end{lstlisting}
\end{frame}
\begin{frame}[fragile]
\frametitle{C\'odigo Preprocesado (Sin Pretty Print)}
\begin{lstlisting}[style=CStyle]


 
struct VERSION 
{ 
char codigo [  5 ] ; 
struct PREGUNTA_VERSION 
{ 
char codigo [  7 ] ; 
char respuesta ; 
int orden_version [ 5 ] ; 
} * preguntas ; 
} * versiones = NULL ; 
struct PREGUNTA 
{ \end{lstlisting}
\end{frame}
\begin{frame}[fragile]
\frametitle{C\'odigo Preprocesado (Sin Pretty Print)}
\begin{lstlisting}[style=CStyle]

char tema [ CODIGO_TEMA_SIZE + 1 ] ; 
char subtema [ CODIGO_SUBTEMA_SIZE + 1 ] ; 
char autor [  101 ] ; 
int orden_tema ; 
int orden_subtema ; 
char ejercicio [  7 ] ; 
int secuencia ; 
char pregunta [  7 ] ; 
int buenos ; 
int malos ; 
int acumulado_opciones [ 5 ] ; 
long double suma_seleccion [ 5 ] ; 
long \end{lstlisting}
\end{frame}
\begin{frame}[fragile]
\frametitle{C\'odigo Preprocesado (Sin Pretty Print)}
\begin{lstlisting}[style=CStyle]
double media_ex_1 [ 5 ] ; 
long double media_ex_0 [ 5 ] ; 
long double Rpb_opcion [ 5 ] ; 
char correcta ; 
long double previo ; 
long double desviacion ; 
long double porcentaje ; 
long double suma_buenos ; 
long double suma_malos ; 
long double Rpb ; 
long double alfa_sin ; 
int flags [  16 ] ; 
char texto_pregunta [  501 + 1 ] ; 
int \end{lstlisting}
\end{frame}
\begin{frame}[fragile]
\frametitle{C\'odigo Preprocesado (Sin Pretty Print)}
\begin{lstlisting}[style=CStyle]
ajuste ; 
int revision_especial ; 
int correctas_nuevas [ 5 ] ; 
int actualizar ; 
int excluir ; 
int encoger ; 
int verbatim ; 
int header_encoger ; 
int header_verbatim ; 
int slide [ 5 ] ; 
int encoger_opcion [ 5 ] ; 
int verbatim_opcion [ 5 ] ; 
int grupo_inicio ; 
int \end{lstlisting}
\end{frame}
\begin{frame}[fragile]
\frametitle{C\'odigo Preprocesado (Sin Pretty Print)}
\begin{lstlisting}[style=CStyle]
grupo_final ; 
} ; 
struct PREGUNTA * preguntas = NULL ; 
struct PREGUNTA * resumen_tema_subtema = NULL ; 
struct PREGUNTA * resumen_tema = NULL ; 


 

 

 

 

 

 int Banderas [ 6 ] ; 
char * colores [ ] = { "green" , "blue" , "cyan" , "yellow" , "red" , "violet" } ; 
char \end{lstlisting}
\end{frame}
\begin{frame}[fragile]
\frametitle{C\'odigo Preprocesado (Sin Pretty Print)}
\begin{lstlisting}[style=CStyle]
* banderas [ ] = { "FL-verde.jpg" , "FL-azul.jpg" , "FL-cyan.jpg" , "FL-amarilla.jpg" , "FL-roja.jpg" , "fix.png" } ; 


 

 


 

 

 

 

 

 

 

 
int \end{lstlisting}
\end{frame}
\begin{frame}[fragile]
\frametitle{C\'odigo Preprocesado (Sin Pretty Print)}
\begin{lstlisting}[style=CStyle]
Niveles_Discriminacion ; 
int Niveles_Dificultad ; 
int Frecuencia_total [ 11 ] [ 21 ] ; 
char * Beamer_aspectratio [ ] = { "43" , "169" , "1610" , "149" , "141" , "54" , "32" } ; 
PGconn * DATABASE ; 
GtkBuilder * builder ; 
GError * error ; 
GtkWidget * window1 = NULL ; 
GtkWidget * window2 = NULL ; 
GtkWidget * window3 = NULL ; 
GtkWidget * window4 = NULL ; 
GtkWidget * window5 = NULL ; 
GtkWidget * window6 = NULL ; 
GtkWidget \end{lstlisting}
\end{frame}
\begin{frame}[fragile]
\frametitle{C\'odigo Preprocesado (Sin Pretty Print)}
\begin{lstlisting}[style=CStyle]
* window7 = NULL ; 
GtkWidget * EB_analisis = NULL ; 
GtkWidget * FR_analisis = NULL ; 
GtkSpinButton * SP_examen = NULL ; 
GtkWidget * EN_descripcion = NULL ; 
GtkWidget * EN_pre_examen = NULL ; 
GtkWidget * EN_pre_examen_descripcion = NULL ; 
GtkWidget * EN_esquema = NULL ; 
GtkWidget * EN_esquema_descripcion = NULL ; 
GtkWidget * EN_materia = NULL ; 
GtkWidget * EN_materia_descripcion = NULL ; 
GtkWidget * EN_institucion = NULL ; 
GtkWidget * EN_escuela = NULL ; 
GtkWidget \end{lstlisting}
\end{frame}
\begin{frame}[fragile]
\frametitle{C\'odigo Preprocesado (Sin Pretty Print)}
\begin{lstlisting}[style=CStyle]
* EN_programa = NULL ; 
GtkWidget * EN_profesor = NULL ; 
GtkWidget * EN_fecha = NULL ; 
GtkWidget * EN_N_preguntas = NULL ; 
GtkWidget * EN_N_versiones = NULL ; 
GtkWidget * EN_N_estudiantes = NULL ; 
GtkWidget * EB_prediccion = NULL ; 
GtkWidget * FR_prediccion = NULL ; 
GtkWidget * EB_real = NULL ; 
GtkWidget * FR_real = NULL ; 
GtkWidget * EB_grafico = NULL ; 
GtkWidget * FR_grafico = NULL ; 
GtkToggleButton * CK_smooth = NULL ; 
GtkWidget \end{lstlisting}
\end{frame}
\begin{frame}[fragile]
\frametitle{C\'odigo Preprocesado (Sin Pretty Print)}
\begin{lstlisting}[style=CStyle]
* EB_preguntas = NULL ; 
GtkWidget * FR_preguntas = NULL ; 
GtkWidget * EB_ajustes = NULL ; 
GtkWidget * FR_ajustes = NULL ; 
GtkWidget * EB_formato = NULL ; 
GtkWidget * FR_formato = NULL ; 
GtkWidget * FR_procesado = NULL ; 
GtkWidget * FR_botones = NULL ; 
GtkWidget * FR_pregunta_actual = NULL ; 
GtkWidget * SC_preguntas = NULL ; 
GtkTextView * TV_pregunta = NULL ; 
GtkTextBuffer * buffer_TV_pregunta ; 
GtkComboBox * CB_ajuste = NULL ; 
GtkToggleButton \end{lstlisting}
\end{frame}
\begin{frame}[fragile]
\frametitle{C\'odigo Preprocesado (Sin Pretty Print)}
\begin{lstlisting}[style=CStyle]
* TG_A = NULL ; 
GtkToggleButton * TG_B = NULL ; 
GtkToggleButton * TG_C = NULL ; 
GtkToggleButton * TG_D = NULL ; 
GtkToggleButton * TG_E = NULL ; 
GtkToggleButton * CK_no_actualiza = NULL ; 
GtkToggleButton * CK_excluir = NULL ; 
GtkToggleButton * CK_encoger = NULL ; 
GtkToggleButton * CK_verbatim = NULL ; 
GtkToggleButton * CK_header_encoger = NULL ; 
GtkToggleButton * CK_header_verbatim = NULL ; 
GtkToggleButton * CK_slide [ 5 ] ; 
GtkToggleButton * CK_encoger_opcion [ 5 ] ; 
GtkToggleButton \end{lstlisting}
\end{frame}
\begin{frame}[fragile]
\frametitle{C\'odigo Preprocesado (Sin Pretty Print)}
\begin{lstlisting}[style=CStyle]
* CK_verbatim_opcion [ 5 ] ; 
GtkWidget * EB_beamer = NULL ; 
GtkWidget * FR_beamer = NULL ; 
GtkComboBox * CB_estilo = NULL ; 
GtkComboBox * CB_color = NULL ; 
GtkComboBox * CB_font = NULL ; 
GtkComboBox * CB_size = NULL ; 
GtkComboBox * CB_aspecto = NULL ; 
GtkToggleButton * CK_general = NULL ; 
GtkToggleButton * CK_sin_banderas = NULL ; 
GtkWidget * EN_media_prediccion = NULL ; 
GtkWidget * EN_desviacion_prediccion = NULL ; 
GtkWidget * EN_alfa_prediccion = NULL ; 
GtkWidget \end{lstlisting}
\end{frame}
\begin{frame}[fragile]
\frametitle{C\'odigo Preprocesado (Sin Pretty Print)}
\begin{lstlisting}[style=CStyle]
* EN_Rpb_prediccion = NULL ; 
GtkWidget * EN_media_real = NULL ; 
GtkWidget * EN_desviacion_real = NULL ; 
GtkWidget * EN_alfa_real = NULL ; 
GtkWidget * EN_Rpb_real = NULL ; 
GtkWidget * BN_undo = NULL ; 
GtkWidget * BN_print = NULL ; 
GtkWidget * BN_slides = NULL ; 
GtkWidget * BN_save = NULL ; 
GtkWidget * BN_terminar = NULL ; 
GtkWidget * BN_ok = NULL ; 
GtkWidget * BN_error_encontrado_Beamer = NULL ; 
GtkWidget * EB_generando_beamer = NULL ; 
GtkWidget \end{lstlisting}
\end{frame}
\begin{frame}[fragile]
\frametitle{C\'odigo Preprocesado (Sin Pretty Print)}
\begin{lstlisting}[style=CStyle]
* FR_generando_beamer = NULL ; 
GtkWidget * EB_generando_analisis = NULL ; 
GtkWidget * FR_generando_analisis = NULL ; 
GtkWidget * FR_error_Beamer = NULL ; 
GtkWidget * BN_confirma_revision = NULL ; 
GtkWidget * BN_cancela_revision = NULL ; 
GtkWidget * EB_revisando_beamer = NULL ; 
GtkWidget * FR_revisando_beamer = NULL ; 
GtkWidget * FR_error_encontrado_Beamer = NULL ; 
GtkLabel * LB_error_encontrado_Beamer = NULL ; 
GtkWidget * PB_analisis = NULL ; 
GtkWidget * PB_beamer = NULL ; 
GtkWidget * PB_revisando_beamer = NULL ; 
GtkSpinButton \end{lstlisting}
\end{frame}
\begin{frame}[fragile]
\frametitle{C\'odigo Preprocesado (Sin Pretty Print)}
\begin{lstlisting}[style=CStyle]
* SP_resolucion = NULL ; 
GtkSpinButton * SP_color = NULL ; 
GtkSpinButton * SP_rotacion = NULL ; 
void Actualice_Estadisticas ( int buena , char * ejercicio , int secuencia , long double Nota , char respuesta , int opcion_original [ 5 ] ) ; 
void Actualiza_Porcentajes ( ) ; 
void Analisis_General ( FILE * Archivo_Latex , long double alfa , long double Rpb , int Beamer_o_reporte ) ; 
void Analiza_Ajuste ( FILE * Archivo_Latex , struct PREGUNTA item , int modo ) ; 
void Analiza_Banderas ( FILE * Archivo_Latex , struct PREGUNTA item , int beamer , int N_flags , int i , char * Descripcion ) ; 
void Asigna_Banderas ( ) ; 
void Beamer_Cierre ( FILE * Archivo_Latex ) ; 
void Beamer_Cover ( FILE * Archivo_Latex ) ; 
void Beamer_Dificultad_vs_Discriminacion ( FILE * Archivo_Latex ) ; 
void Beamer_Failure ( ) ; 
void \end{lstlisting}
\end{frame}
\begin{frame}[fragile]
\frametitle{C\'odigo Preprocesado (Sin Pretty Print)}
\begin{lstlisting}[style=CStyle]
Beamer_Gracias ( FILE * Archivo_Latex ) ; 
void Beamer_Grafico_Pastel ( FILE * Archivo_Latex ) ; 
void Beamer_Datos_Generales ( FILE * Archivo_Latex , gchar * institucion , gchar * escuela , gchar * programa , gchar * materia_descripcion , gchar * profesor , gchar * descripcion , gchar * fecha , char * codigo , long double media_real , long double desviacion_real , long double alfa , long double Rpb ) ; 
void Beamer_Histograma_Notas ( FILE * Archivo_Latex , long double media_real , long double desviacion_real , long double media_prediccion , long double desviacion_prediccion ) ; 
void Beamer_Histograma_Temas ( FILE * Archivo_Latex ) ; 
void Beamer_Preamble ( FILE * Archivo_Latex , int aspecto , gchar * size , gchar * estilo , gchar * color , gchar * font , gchar * materia_descripcion , gchar * descripcion , gchar * profesor , gchar * programa , gchar * escuela , gchar * institucion , gchar * fecha ) ; 
void Beamer_Preamble_reducido ( FILE * Archivo_Latex , int aspecto , gchar * size , gchar * estilo , gchar * color , gchar * font ) ; 
void Beamer_Preguntas ( FILE * Archivo_Latex , GtkWidget * PB , long double base , long double limite ) ; 
void Beamer_TOC ( FILE * Archivo_Latex ) ; 
void Cajita_con_bandera ( FILE * Archivo_Latex , char * mensaje , int color , int modo ) ; 
void Calcula_ajuste ( int i ) ; 
void Calcula_estadisticas_examen ( ) ; 
void Calcula_nota_ajustada ( int k_version , char * respuestas , int * n , int * m ) ; 
void \end{lstlisting}
\end{frame}
\begin{frame}[fragile]
\frametitle{C\'odigo Preprocesado (Sin Pretty Print)}
\begin{lstlisting}[style=CStyle]
Calcula_Notas ( char * version , char * respuestas , int * n_buenas , int * n_ajustado , int * m_ajustado ) ; 
void Calcular_Tabla ( ) ; 
void Cambia_Pregunta ( ) ; 
void Cambio_A ( GtkWidget * widget , gpointer user_data ) ; 
void Cambio_Ajuste ( GtkWidget * widget , gpointer user_data ) ; 
void Cambio_B ( GtkWidget * widget , gpointer user_data ) ; 
void Cambio_C ( GtkWidget * widget , gpointer user_data ) ; 
void Cambio_D ( GtkWidget * widget , gpointer user_data ) ; 
void Cambio_E ( GtkWidget * widget , gpointer user_data ) ; 
void Cambio_encoger ( GtkWidget * widget , gpointer user_data ) ; 
void Cambio_encoger_opcion ( int i ) ; 
void Cambio_Examen ( ) ; 
void Cambio_excluir ( GtkWidget * widget , gpointer user_data ) ; 
void \end{lstlisting}
\end{frame}
\begin{frame}[fragile]
\frametitle{C\'odigo Preprocesado (Sin Pretty Print)}
\begin{lstlisting}[style=CStyle]
Cambio_header_encoger ( GtkWidget * widget , gpointer user_data ) ; 
void Cambio_header_verbatim ( GtkWidget * widget , gpointer user_data ) ; 
void Cambio_no_actualizar ( GtkWidget * widget , gpointer user_data ) ; 
void Cambio_slide ( int i ) ; 
void Cambio_verbatim ( GtkWidget * widget , gpointer user_data ) ; 
void Cambio_verbatim_opcion ( int i ) ; 
void Carga_preguntas_examen ( ) ; 
long double CDF ( long double X , long double Media , long double Desv ) ; 
void Color_ajustes ( int k ) ; 
void Color_Fila ( FILE * Archivo_Latex , int flags [  16 ] ) ; 
void colores_pastel ( FILE * Archivo_Latex ) ; 
void Connect_Widgets ( ) ; 
void Construye_versiones ( char * examen ) ; 
void \end{lstlisting}
\end{frame}
\begin{frame}[fragile]
\frametitle{C\'odigo Preprocesado (Sin Pretty Print)}
\begin{lstlisting}[style=CStyle]
Continuar_banderas ( FILE * Archivo_Latex , int i , char * Descripcion ) ; 
void Crea_archivo_datos_pastel ( int * Empates ) ; 
void Dificultad_vs_Discriminacion ( ) ; 
void Establece_Directorio ( char * Directorio , gchar * materia , char * year , char * month , char * day ) ; 
void Fin_de_Programa ( GtkWidget * widget , gpointer user_data ) ; 
void Fin_Ventana ( GtkWidget * widget , gpointer user_data ) ; 
void Genera_Beamer ( ) ; 
int Genera_Beamer_reducido ( ) ; 
void Graba_Ajustes ( ) ; 
void Imprime_Opcion ( FILE * Archivo_Latex , PGresult * res , long double Porcentaje , int pregunta , int opcion ) ; 
void Imprime_Opcion_Beamer ( FILE * Archivo_Latex , PGresult * res , long double Porcentaje , int pregunta , int opcion ) ; 
void Imprime_pregunta ( int i , FILE * Archivo_Latex , char * prefijo ) ; 
void Imprime_pregunta_Beamer ( int i , FILE * Archivo_Latex , char * prefijo , char * tema_descripcion ) ; 
void \end{lstlisting}
\end{frame}
\begin{frame}[fragile]
\frametitle{C\'odigo Preprocesado (Sin Pretty Print)}
\begin{lstlisting}[style=CStyle]
Imprime_Reporte ( ) ; 
void Interface_Coloring ( ) ; 
void Inicializa_Tabla_estadisticas ( ) ; 
void Init_Fields ( ) ; 
void Lista_de_Notas ( FILE * Archivo_Latex ) ; 
void Lista_de_Preguntas ( FILE * Archivo_Latex , GtkWidget * PB , long double base , long double limite ) ; 
void Lista_de_Preguntas_Beamer ( FILE * Archivo_Latex , GtkWidget * PB , long double base , long double limite ) ; 
int main ( int argc , char * argv [ ] ) ; 
void Marca_agua_ajuste ( FILE * Archivo_Latex , int i ) ; 
void on_BN_ok_clicked ( GtkWidget * widget , gpointer user_data ) ; 
void on_BN_print_clicked ( GtkWidget * widget , gpointer user_data ) ; 
void on_BN_save_clicked ( GtkWidget * widget , gpointer user_data ) ; 
void on_BN_slides_clicked ( GtkWidget * widget , gpointer user_data ) ; 
void \end{lstlisting}
\end{frame}
\begin{frame}[fragile]
\frametitle{C\'odigo Preprocesado (Sin Pretty Print)}
\begin{lstlisting}[style=CStyle]
on_BN_terminar_clicked ( GtkWidget * widget , gpointer user_data ) ; 
void on_BN_undo_clicked ( GtkWidget * widget , gpointer user_data ) ; 
void on_CB_ajuste_changed ( GtkWidget * widget , gpointer user_data ) ; 
void on_CK_encoger_A_toggled ( GtkWidget * widget , gpointer user_data ) ; 
void on_CK_encoger_B_toggled ( GtkWidget * widget , gpointer user_data ) ; 
void on_CK_encoger_C_toggled ( GtkWidget * widget , gpointer user_data ) ; 
void on_CK_encoger_D_toggled ( GtkWidget * widget , gpointer user_data ) ; 
void on_CK_encoger_E_toggled ( GtkWidget * widget , gpointer user_data ) ; 
void on_CK_encoger_toggled ( GtkWidget * widget , gpointer user_data ) ; 
void on_CK_excluir_toggled ( GtkWidget * widget , gpointer user_data ) ; 
void on_CK_header_encoger_toggled ( GtkWidget * widget , gpointer user_data ) ; 
void on_CK_header_verbatim_toggled ( GtkWidget * widget , gpointer user_data ) ; 
void on_CK_no_actualiza_toggled ( GtkWidget * widget , gpointer user_data ) ; 
void \end{lstlisting}
\end{frame}
\begin{frame}[fragile]
\frametitle{C\'odigo Preprocesado (Sin Pretty Print)}
\begin{lstlisting}[style=CStyle]
on_CK_slide_A_toggled ( GtkWidget * widget , gpointer user_data ) ; 
void on_CK_slide_B_toggled ( GtkWidget * widget , gpointer user_data ) ; 
void on_CK_slide_C_toggled ( GtkWidget * widget , gpointer user_data ) ; 
void on_CK_slide_D_toggled ( GtkWidget * widget , gpointer user_data ) ; 
void on_CK_slide_E_toggled ( GtkWidget * widget , gpointer user_data ) ; 
void on_CK_verbatim_A_toggled ( GtkWidget * widget , gpointer user_data ) ; 
void on_CK_verbatim_B_toggled ( GtkWidget * widget , gpointer user_data ) ; 
void on_CK_verbatim_C_toggled ( GtkWidget * widget , gpointer user_data ) ; 
void on_CK_verbatim_D_toggled ( GtkWidget * widget , gpointer user_data ) ; 
void on_CK_verbatim_E_toggled ( GtkWidget * widget , gpointer user_data ) ; 
void on_CK_verbatim_toggled ( GtkWidget * widget , gpointer user_data ) ; 
void on_EN_examen_activate ( GtkWidget * widget , gpointer user_data ) ; 
void on_SC_preguntas_value_changed ( GtkWidget * widget , gpointer user_data ) ; 
void \end{lstlisting}
\end{frame}
\begin{frame}[fragile]
\frametitle{C\'odigo Preprocesado (Sin Pretty Print)}
\begin{lstlisting}[style=CStyle]
on_SP_examen_activate ( GtkWidget * widget , gpointer user_data ) ; 
void on_TG_A_toggled ( GtkWidget * widget , gpointer user_data ) ; 
void on_TG_B_toggled ( GtkWidget * widget , gpointer user_data ) ; 
void on_TG_C_toggled ( GtkWidget * widget , gpointer user_data ) ; 
void on_TG_D_toggled ( GtkWidget * widget , gpointer user_data ) ; 
void on_TG_E_toggled ( GtkWidget * widget , gpointer user_data ) ; 
void on_WN_ex4010_destroy ( GtkWidget * widget , gpointer user_data ) ; 
void Prepara_Grafico_Normal ( long double media , long double desviacion , long double media_pred , long double desviacion_pred , long double width ) ; 
void Prepara_Grafico_Pastel ( FILE * Archivo_Latex ) ; 
void Prepara_Histograma_Notas ( ) ; 
void Prepara_Histograma_Subtemas ( ) ; 
void Prepara_Histograma_Temas ( ) ; 
void prepara_opciones ( char * opciones_frame , int i , int k ) ; 
void \end{lstlisting}
\end{frame}
\begin{frame}[fragile]
\frametitle{C\'odigo Preprocesado (Sin Pretty Print)}
\begin{lstlisting}[style=CStyle]
Read_Only_Fields ( ) ; 
void Quita_espacios ( char * hilera ) ; 
void Resumen_de_Banderas ( FILE * Archivo_Latex , int modo ) ; 
void Update_PB ( GtkWidget * PB , long double R ) ; 
void Tabla_Datos_Generales ( FILE * Archivo_Latex , char * institucion , char * escuela , char * programa , char * materia_descripcion , char * profesor , char * descripcion , char * fecha , char * codigo , long double media_real , long double desviacion_real , long double alfa , long double Rpb , int Beamer_o_reporte ) ; 
int main ( int argc , char * argv [ ] ) 
{ 
DATABASE = EX_connect_data_base ( ) ; 
if ( PQstatus ( DATABASE ) == CONNECTION_BAD ) 
{ 
fprintf ( stderr , "Connection to database failed.\n" ) ; 
fprintf ( stderr , "%s" , PQerrorMessage ( DATABASE ) ) ; 
PQfinish ( DATABASE ) ; 
} \end{lstlisting}
\end{frame}
\begin{frame}[fragile]
\frametitle{C\'odigo Preprocesado (Sin Pretty Print)}
\begin{lstlisting}[style=CStyle]

else 
{ 
gtk_init ( & argc , & argv ) ; 
setlocale ( LC_NUMERIC , "en_US.UTF-8" ) ; 
carga_parametros_EXAMINER ( & parametros , DATABASE ) ; 
builder = gtk_builder_new ( ) ; 
if ( ! gtk_builder_add_from_file ( builder , ".interfaces/EX4010.glade" , & error ) ) 
{ 
g_warning ( "%s\n" , error -> message ) ; 
g_error_free ( error ) ; 
} 
else 
{ \end{lstlisting}
\end{frame}
\begin{frame}[fragile]
\frametitle{C\'odigo Preprocesado (Sin Pretty Print)}
\begin{lstlisting}[style=CStyle]

gtk_builder_connect_signals ( builder , NULL ) ; 
Connect_Widgets ( ) ; 
Read_Only_Fields ( ) ; 
Interface_Coloring ( ) ; 
gtk_widget_show ( window1 ) ; 
Init_Fields ( ) ; 
gtk_main ( ) ; 
} 
} 
return 0 ; 
} 
void Connect_Widgets ( ) 
{ \end{lstlisting}
\end{frame}
\begin{frame}[fragile]
\frametitle{C\'odigo Preprocesado (Sin Pretty Print)}
\begin{lstlisting}[style=CStyle]

window1 = GTK_WIDGET ( gtk_builder_get_object ( builder , "WN_ex4010" ) ) ; 
window2 = GTK_WIDGET ( gtk_builder_get_object ( builder , "WN_procesado" ) ) ; 
window3 = GTK_WIDGET ( gtk_builder_get_object ( builder , "WN_generando_beamer" ) ) ; 
window4 = GTK_WIDGET ( gtk_builder_get_object ( builder , "WN_generando_analisis" ) ) ; 
window5 = GTK_WIDGET ( gtk_builder_get_object ( builder , "WN_error_Beamer" ) ) ; 
window6 = GTK_WIDGET ( gtk_builder_get_object ( builder , "WN_revisando_beamer" ) ) ; 
window7 = GTK_WIDGET ( gtk_builder_get_object ( builder , "WN_error_encontrado_Beamer" ) ) ; 
EB_analisis = GTK_WIDGET ( gtk_builder_get_object ( builder , "EB_analisis" ) ) ; 
FR_analisis = GTK_WIDGET ( gtk_builder_get_object ( builder , "FR_analisis" ) ) ; 
SP_examen = ( GtkSpinButton * ) GTK_WIDGET ( gtk_builder_get_object ( builder , "SP_examen" ) ) ; 
EN_descripcion = GTK_WIDGET ( gtk_builder_get_object ( builder , "EN_descripcion" ) ) ; 
EN_pre_examen = GTK_WIDGET ( gtk_builder_get_object ( builder , "EN_pre_examen" ) ) ; 
EN_pre_examen_descripcion \end{lstlisting}
\end{frame}
\begin{frame}[fragile]
\frametitle{C\'odigo Preprocesado (Sin Pretty Print)}
\begin{lstlisting}[style=CStyle]
= GTK_WIDGET ( gtk_builder_get_object ( builder , "EN_pre_examen_descripcion" ) ) ; 
EN_esquema = GTK_WIDGET ( gtk_builder_get_object ( builder , "EN_esquema" ) ) ; 
EN_esquema_descripcion = GTK_WIDGET ( gtk_builder_get_object ( builder , "EN_esquema_descripcion" ) ) ; 
EN_materia = GTK_WIDGET ( gtk_builder_get_object ( builder , "EN_materia" ) ) ; 
EN_materia_descripcion = GTK_WIDGET ( gtk_builder_get_object ( builder , "EN_materia_descripcion" ) ) ; 
EN_institucion = GTK_WIDGET ( gtk_builder_get_object ( builder , "EN_institucion" ) ) ; 
EN_escuela = GTK_WIDGET ( gtk_builder_get_object ( builder , "EN_escuela" ) ) ; 
EN_programa = GTK_WIDGET ( gtk_builder_get_object ( builder , "EN_programa" ) ) ; 
EN_profesor = GTK_WIDGET ( gtk_builder_get_object ( builder , "EN_profesor" ) ) ; 
EN_fecha = GTK_WIDGET ( gtk_builder_get_object ( builder , "EN_fecha" ) ) ; 
EN_N_preguntas = GTK_WIDGET ( gtk_builder_get_object ( builder , "EN_N_preguntas" ) ) ; 
EN_N_versiones = GTK_WIDGET ( gtk_builder_get_object ( builder , "EN_N_versiones" ) ) ; 
EN_N_estudiantes = GTK_WIDGET ( gtk_builder_get_object ( builder , "EN_N_estudiantes" ) ) ; 
EB_prediccion \end{lstlisting}
\end{frame}
\begin{frame}[fragile]
\frametitle{C\'odigo Preprocesado (Sin Pretty Print)}
\begin{lstlisting}[style=CStyle]
= GTK_WIDGET ( gtk_builder_get_object ( builder , "EB_prediccion" ) ) ; 
FR_prediccion = GTK_WIDGET ( gtk_builder_get_object ( builder , "FR_prediccion" ) ) ; 
EN_media_prediccion = GTK_WIDGET ( gtk_builder_get_object ( builder , "EN_media_prediccion" ) ) ; 
EN_desviacion_prediccion = GTK_WIDGET ( gtk_builder_get_object ( builder , "EN_desviacion_prediccion" ) ) ; 
EN_alfa_prediccion = GTK_WIDGET ( gtk_builder_get_object ( builder , "EN_alfa_prediccion" ) ) ; 
EN_Rpb_prediccion = GTK_WIDGET ( gtk_builder_get_object ( builder , "EN_Rpb_prediccion" ) ) ; 
EB_real = GTK_WIDGET ( gtk_builder_get_object ( builder , "EB_real" ) ) ; 
FR_real = GTK_WIDGET ( gtk_builder_get_object ( builder , "FR_real" ) ) ; 
EN_media_real = GTK_WIDGET ( gtk_builder_get_object ( builder , "EN_media_real" ) ) ; 
EN_desviacion_real = GTK_WIDGET ( gtk_builder_get_object ( builder , "EN_desviacion_real" ) ) ; 
EN_alfa_real = GTK_WIDGET ( gtk_builder_get_object ( builder , "EN_alfa_real" ) ) ; 
EN_Rpb_real = GTK_WIDGET ( gtk_builder_get_object ( builder , "EN_Rpb_real" ) ) ; 
FR_procesado = GTK_WIDGET ( gtk_builder_get_object ( builder , "FR_procesado" ) ) ; 
EB_grafico \end{lstlisting}
\end{frame}
\begin{frame}[fragile]
\frametitle{C\'odigo Preprocesado (Sin Pretty Print)}
\begin{lstlisting}[style=CStyle]
= GTK_WIDGET ( gtk_builder_get_object ( builder , "EB_grafico" ) ) ; 
FR_grafico = GTK_WIDGET ( gtk_builder_get_object ( builder , "FR_grafico" ) ) ; 
SP_resolucion = ( GtkSpinButton * ) GTK_WIDGET ( gtk_builder_get_object ( builder , "SP_resolucion" ) ) ; 
SP_color = ( GtkSpinButton * ) GTK_WIDGET ( gtk_builder_get_object ( builder , "SP_color" ) ) ; 
SP_rotacion = ( GtkSpinButton * ) GTK_WIDGET ( gtk_builder_get_object ( builder , "SP_rotacion" ) ) ; 
CK_smooth = ( GtkToggleButton * ) GTK_WIDGET ( gtk_builder_get_object ( builder , "CK_smooth" ) ) ; 
FR_botones = GTK_WIDGET ( gtk_builder_get_object ( builder , "FR_botones" ) ) ; 
EB_preguntas = GTK_WIDGET ( gtk_builder_get_object ( builder , "EB_preguntas" ) ) ; 
FR_preguntas = GTK_WIDGET ( gtk_builder_get_object ( builder , "FR_preguntas" ) ) ; 
EB_ajustes = GTK_WIDGET ( gtk_builder_get_object ( builder , "EB_ajustes" ) ) ; 
FR_ajustes = GTK_WIDGET ( gtk_builder_get_object ( builder , "FR_ajustes" ) ) ; 
EB_formato = GTK_WIDGET ( gtk_builder_get_object ( builder , "EB_formato" ) ) ; 
FR_formato = GTK_WIDGET ( gtk_builder_get_object ( builder , "FR_formato" ) ) ; 
FR_pregunta_actual \end{lstlisting}
\end{frame}
\begin{frame}[fragile]
\frametitle{C\'odigo Preprocesado (Sin Pretty Print)}
\begin{lstlisting}[style=CStyle]
= GTK_WIDGET ( gtk_builder_get_object ( builder , "FR_pregunta_actual" ) ) ; 
SC_preguntas = GTK_WIDGET ( gtk_builder_get_object ( builder , "SC_preguntas" ) ) ; 
TV_pregunta = ( GtkTextView * ) GTK_WIDGET ( gtk_builder_get_object ( builder , "TV_pregunta" ) ) ; 
buffer_TV_pregunta = gtk_text_view_get_buffer ( TV_pregunta ) ; 
EB_beamer = GTK_WIDGET ( gtk_builder_get_object ( builder , "EB_beamer" ) ) ; 
FR_beamer = GTK_WIDGET ( gtk_builder_get_object ( builder , "FR_beamer" ) ) ; 
CB_estilo = ( GtkComboBox * ) GTK_WIDGET ( gtk_builder_get_object ( builder , "CB_estilo" ) ) ; 
CB_color = ( GtkComboBox * ) GTK_WIDGET ( gtk_builder_get_object ( builder , "CB_color" ) ) ; 
CB_font = ( GtkComboBox * ) GTK_WIDGET ( gtk_builder_get_object ( builder , "CB_font" ) ) ; 
CB_size = ( GtkComboBox * ) GTK_WIDGET ( gtk_builder_get_object ( builder , "CB_size" ) ) ; 
CB_aspecto = ( GtkComboBox * ) GTK_WIDGET ( gtk_builder_get_object ( builder , "CB_aspecto" ) ) ; 
CK_general = ( GtkToggleButton * ) GTK_WIDGET ( gtk_builder_get_object ( builder , "CK_general" ) ) ; 
CK_sin_banderas = ( GtkToggleButton * ) GTK_WIDGET ( gtk_builder_get_object ( builder , "CK_sin_banderas" ) ) ; 
TG_A \end{lstlisting}
\end{frame}
\begin{frame}[fragile]
\frametitle{C\'odigo Preprocesado (Sin Pretty Print)}
\begin{lstlisting}[style=CStyle]
= ( GtkToggleButton * ) GTK_WIDGET ( gtk_builder_get_object ( builder , "TG_A" ) ) ; 
TG_B = ( GtkToggleButton * ) GTK_WIDGET ( gtk_builder_get_object ( builder , "TG_B" ) ) ; 
TG_C = ( GtkToggleButton * ) GTK_WIDGET ( gtk_builder_get_object ( builder , "TG_C" ) ) ; 
TG_D = ( GtkToggleButton * ) GTK_WIDGET ( gtk_builder_get_object ( builder , "TG_D" ) ) ; 
TG_E = ( GtkToggleButton * ) GTK_WIDGET ( gtk_builder_get_object ( builder , "TG_E" ) ) ; 
CB_ajuste = ( GtkComboBox * ) GTK_WIDGET ( gtk_builder_get_object ( builder , "CB_ajuste" ) ) ; 
CK_no_actualiza = ( GtkToggleButton * ) GTK_WIDGET ( gtk_builder_get_object ( builder , "CK_no_actualiza" ) ) ; 
CK_excluir = ( GtkToggleButton * ) GTK_WIDGET ( gtk_builder_get_object ( builder , "CK_excluir" ) ) ; 
CK_encoger = ( GtkToggleButton * ) GTK_WIDGET ( gtk_builder_get_object ( builder , "CK_encoger" ) ) ; 
CK_verbatim = ( GtkToggleButton * ) GTK_WIDGET ( gtk_builder_get_object ( builder , "CK_verbatim" ) ) ; 
CK_header_encoger = ( GtkToggleButton * ) GTK_WIDGET ( gtk_builder_get_object ( builder , "CK_header_encoger" ) ) ; 
CK_header_verbatim = ( GtkToggleButton * ) GTK_WIDGET ( gtk_builder_get_object ( builder , "CK_header_verbatim" ) ) ; 
CK_slide [ 0 ] = ( GtkToggleButton * ) GTK_WIDGET ( gtk_builder_get_object ( builder , "CK_slide_A" ) ) ; 
CK_slide \end{lstlisting}
\end{frame}
\begin{frame}[fragile]
\frametitle{C\'odigo Preprocesado (Sin Pretty Print)}
\begin{lstlisting}[style=CStyle]
[ 1 ] = ( GtkToggleButton * ) GTK_WIDGET ( gtk_builder_get_object ( builder , "CK_slide_B" ) ) ; 
CK_slide [ 2 ] = ( GtkToggleButton * ) GTK_WIDGET ( gtk_builder_get_object ( builder , "CK_slide_C" ) ) ; 
CK_slide [ 3 ] = ( GtkToggleButton * ) GTK_WIDGET ( gtk_builder_get_object ( builder , "CK_slide_D" ) ) ; 
CK_slide [ 4 ] = ( GtkToggleButton * ) GTK_WIDGET ( gtk_builder_get_object ( builder , "CK_slide_E" ) ) ; 
CK_encoger_opcion [ 0 ] = ( GtkToggleButton * ) GTK_WIDGET ( gtk_builder_get_object ( builder , "CK_encoger_A" ) ) ; 
CK_encoger_opcion [ 1 ] = ( GtkToggleButton * ) GTK_WIDGET ( gtk_builder_get_object ( builder , "CK_encoger_B" ) ) ; 
CK_encoger_opcion [ 2 ] = ( GtkToggleButton * ) GTK_WIDGET ( gtk_builder_get_object ( builder , "CK_encoger_C" ) ) ; 
CK_encoger_opcion [ 3 ] = ( GtkToggleButton * ) GTK_WIDGET ( gtk_builder_get_object ( builder , "CK_encoger_D" ) ) ; 
CK_encoger_opcion [ 4 ] = ( GtkToggleButton * ) GTK_WIDGET ( gtk_builder_get_object ( builder , "CK_encoger_E" ) ) ; 
CK_verbatim_opcion [ 0 ] = ( GtkToggleButton * ) GTK_WIDGET ( gtk_builder_get_object ( builder , "CK_verbatim_A" ) ) ; 
CK_verbatim_opcion [ 1 ] = ( GtkToggleButton * ) GTK_WIDGET ( gtk_builder_get_object ( builder , "CK_verbatim_B" ) ) ; 
CK_verbatim_opcion [ 2 ] = ( GtkToggleButton * ) GTK_WIDGET ( gtk_builder_get_object ( builder , "CK_verbatim_C" ) ) ; 
CK_verbatim_opcion [ 3 ] = ( GtkToggleButton * ) GTK_WIDGET ( gtk_builder_get_object ( builder , "CK_verbatim_D" ) ) ; 
CK_verbatim_opcion \end{lstlisting}
\end{frame}
\begin{frame}[fragile]
\frametitle{C\'odigo Preprocesado (Sin Pretty Print)}
\begin{lstlisting}[style=CStyle]
[ 4 ] = ( GtkToggleButton * ) GTK_WIDGET ( gtk_builder_get_object ( builder , "CK_verbatim_E" ) ) ; 
EB_generando_beamer = GTK_WIDGET ( gtk_builder_get_object ( builder , "EB_generando_beamer" ) ) ; 
FR_generando_beamer = GTK_WIDGET ( gtk_builder_get_object ( builder , "FR_generando_beamer" ) ) ; 
EB_generando_analisis = GTK_WIDGET ( gtk_builder_get_object ( builder , "EB_generando_analisis" ) ) ; 
FR_generando_analisis = GTK_WIDGET ( gtk_builder_get_object ( builder , "FR_generando_analisis" ) ) ; 
FR_error_Beamer = GTK_WIDGET ( gtk_builder_get_object ( builder , "FR_error_Beamer" ) ) ; 
EB_revisando_beamer = GTK_WIDGET ( gtk_builder_get_object ( builder , "EB_revisando_beamer" ) ) ; 
FR_revisando_beamer = GTK_WIDGET ( gtk_builder_get_object ( builder , "FR_revisando_beamer" ) ) ; 
FR_error_encontrado_Beamer = GTK_WIDGET ( gtk_builder_get_object ( builder , "FR_error_encontrado_Beamer" ) ) ; 
LB_error_encontrado_Beamer = ( GtkLabel * ) GTK_WIDGET ( gtk_builder_get_object ( builder , "LB_error_encontrado_Beamer" ) ) ; 
BN_save = GTK_WIDGET ( gtk_builder_get_object ( builder , "BN_save" ) ) ; 
BN_slides = GTK_WIDGET ( gtk_builder_get_object ( builder , "BN_slides" ) ) ; 
BN_print = GTK_WIDGET ( gtk_builder_get_object ( builder , "BN_print" ) ) ; 
BN_undo \end{lstlisting}
\end{frame}
\begin{frame}[fragile]
\frametitle{C\'odigo Preprocesado (Sin Pretty Print)}
\begin{lstlisting}[style=CStyle]
= GTK_WIDGET ( gtk_builder_get_object ( builder , "BN_undo" ) ) ; 
BN_terminar = GTK_WIDGET ( gtk_builder_get_object ( builder , "BN_terminar" ) ) ; 
BN_ok = GTK_WIDGET ( gtk_builder_get_object ( builder , "BN_ok" ) ) ; 
PB_analisis = GTK_WIDGET ( gtk_builder_get_object ( builder , "PB_analisis" ) ) ; 
PB_beamer = GTK_WIDGET ( gtk_builder_get_object ( builder , "PB_beamer" ) ) ; 
PB_revisando_beamer = GTK_WIDGET ( gtk_builder_get_object ( builder , "PB_revisando_beamer" ) ) ; 
BN_confirma_revision = GTK_WIDGET ( gtk_builder_get_object ( builder , "BN_confirma_revision" ) ) ; 
BN_cancela_revision = GTK_WIDGET ( gtk_builder_get_object ( builder , "BN_cancela_revision" ) ) ; 
BN_error_encontrado_Beamer = GTK_WIDGET ( gtk_builder_get_object ( builder , "BN_error_encontrado_Beamer" ) ) ; 
} 
void Read_Only_Fields ( ) 
{ 
gtk_widget_set_can_focus ( EN_descripcion , FALSE ) ; 
gtk_widget_set_can_focus \end{lstlisting}
\end{frame}
\begin{frame}[fragile]
\frametitle{C\'odigo Preprocesado (Sin Pretty Print)}
\begin{lstlisting}[style=CStyle]
( EN_pre_examen , FALSE ) ; 
gtk_widget_set_can_focus ( EN_pre_examen_descripcion , FALSE ) ; 
gtk_widget_set_can_focus ( EN_esquema , FALSE ) ; 
gtk_widget_set_can_focus ( EN_esquema_descripcion , FALSE ) ; 
gtk_widget_set_can_focus ( EN_materia , FALSE ) ; 
gtk_widget_set_can_focus ( EN_materia_descripcion , FALSE ) ; 
gtk_widget_set_can_focus ( EN_fecha , FALSE ) ; 
gtk_widget_set_can_focus ( EN_N_preguntas , FALSE ) ; 
gtk_widget_set_can_focus ( EN_N_versiones , FALSE ) ; 
gtk_widget_set_can_focus ( EN_N_estudiantes , FALSE ) ; 
gtk_widget_set_can_focus ( EN_profesor , FALSE ) ; 
gtk_widget_set_can_focus ( EN_institucion , FALSE ) ; 
gtk_widget_set_can_focus ( EN_escuela , FALSE ) ; 
gtk_widget_set_can_focus \end{lstlisting}
\end{frame}
\begin{frame}[fragile]
\frametitle{C\'odigo Preprocesado (Sin Pretty Print)}
\begin{lstlisting}[style=CStyle]
( EN_programa , FALSE ) ; 
gtk_widget_set_can_focus ( EN_media_prediccion , FALSE ) ; 
gtk_widget_set_can_focus ( EN_desviacion_prediccion , FALSE ) ; 
gtk_widget_set_can_focus ( EN_alfa_prediccion , FALSE ) ; 
gtk_widget_set_can_focus ( EN_Rpb_prediccion , FALSE ) ; 
gtk_widget_set_can_focus ( EN_media_real , FALSE ) ; 
gtk_widget_set_can_focus ( EN_desviacion_real , FALSE ) ; 
gtk_widget_set_can_focus ( EN_alfa_real , FALSE ) ; 
gtk_widget_set_can_focus ( EN_Rpb_real , FALSE ) ; 
gtk_widget_set_can_focus ( GTK_WIDGET ( TV_pregunta ) , FALSE ) ; 
} 
void Interface_Coloring ( ) 
{ 
GdkColor \end{lstlisting}
\end{frame}
\begin{frame}[fragile]
\frametitle{C\'odigo Preprocesado (Sin Pretty Print)}
\begin{lstlisting}[style=CStyle]
color ; 
gdk_color_parse ( MAIN_WINDOW , & color ) ; 
gtk_widget_modify_bg ( window1 , GTK_STATE_NORMAL , & color ) ; 
gdk_color_parse ( MAIN_AREA , & color ) ; 
gtk_widget_modify_bg ( EB_analisis , GTK_STATE_NORMAL , & color ) ; 
gdk_color_parse ( SECONDARY_AREA , & color ) ; 
gtk_widget_modify_bg ( EB_ajustes , GTK_STATE_NORMAL , & color ) ; 
gtk_widget_modify_bg ( EB_formato , GTK_STATE_NORMAL , & color ) ; 
gdk_color_parse ( THIRD_AREA , & color ) ; 
gtk_widget_modify_bg ( EB_preguntas , GTK_STATE_NORMAL , & color ) ; 
gtk_widget_modify_bg ( EB_beamer , GTK_STATE_NORMAL , & color ) ; 
gtk_widget_modify_bg ( EB_grafico , GTK_STATE_NORMAL , & color ) ; 
gdk_color_parse ( SPECIAL_AREA_1 , & color ) ; 
gtk_widget_modify_bg \end{lstlisting}
\end{frame}
\begin{frame}[fragile]
\frametitle{C\'odigo Preprocesado (Sin Pretty Print)}
\begin{lstlisting}[style=CStyle]
( EB_prediccion , GTK_STATE_NORMAL , & color ) ; 
gdk_color_parse ( SPECIAL_AREA_2 , & color ) ; 
gtk_widget_modify_bg ( EB_real , GTK_STATE_NORMAL , & color ) ; 
gdk_color_parse ( IMPORTANT_WINDOW , & color ) ; 
gtk_widget_modify_bg ( window2 , GTK_STATE_NORMAL , & color ) ; 
gtk_widget_modify_bg ( window5 , GTK_STATE_NORMAL , & color ) ; 
gtk_widget_modify_bg ( window7 , GTK_STATE_NORMAL , & color ) ; 
gdk_color_parse ( IMPORTANT_FR , & color ) ; 
gtk_widget_modify_bg ( FR_procesado , GTK_STATE_NORMAL , & color ) ; 
gtk_widget_modify_bg ( FR_error_Beamer , GTK_STATE_NORMAL , & color ) ; 
gtk_widget_modify_bg ( FR_error_encontrado_Beamer , GTK_STATE_NORMAL , & color ) ; 
gdk_color_parse ( STANDARD_FRAME , & color ) ; 
gtk_widget_modify_bg ( FR_analisis , GTK_STATE_NORMAL , & color ) ; 
gtk_widget_modify_bg \end{lstlisting}
\end{frame}
\begin{frame}[fragile]
\frametitle{C\'odigo Preprocesado (Sin Pretty Print)}
\begin{lstlisting}[style=CStyle]
( FR_preguntas , GTK_STATE_NORMAL , & color ) ; 
gtk_widget_modify_bg ( FR_beamer , GTK_STATE_NORMAL , & color ) ; 
gtk_widget_modify_bg ( FR_botones , GTK_STATE_NORMAL , & color ) ; 
gtk_widget_modify_bg ( FR_grafico , GTK_STATE_NORMAL , & color ) ; 
gtk_widget_modify_bg ( FR_generando_analisis , GTK_STATE_NORMAL , & color ) ; 
gtk_widget_modify_bg ( FR_generando_beamer , GTK_STATE_NORMAL , & color ) ; 
gtk_widget_modify_bg ( FR_revisando_beamer , GTK_STATE_NORMAL , & color ) ; 
gtk_widget_modify_bg ( FR_ajustes , GTK_STATE_NORMAL , & color ) ; 
gtk_widget_modify_bg ( FR_formato , GTK_STATE_NORMAL , & color ) ; 
gtk_widget_modify_bg ( FR_prediccion , GTK_STATE_NORMAL , & color ) ; 
gtk_widget_modify_bg ( FR_real , GTK_STATE_NORMAL , & color ) ; 
gtk_widget_modify_bg ( FR_pregunta_actual , GTK_STATE_NORMAL , & color ) ; 
gdk_color_parse ( STANDARD_ENTRY , & color ) ; 
gtk_widget_modify_bg \end{lstlisting}
\end{frame}
\begin{frame}[fragile]
\frametitle{C\'odigo Preprocesado (Sin Pretty Print)}
\begin{lstlisting}[style=CStyle]
( GTK_WIDGET ( SP_examen ) , GTK_STATE_NORMAL , & color ) ; 
gdk_color_parse ( BUTTON_PRELIGHT , & color ) ; 
gtk_widget_modify_bg ( BN_save , GTK_STATE_PRELIGHT , & color ) ; 
gtk_widget_modify_bg ( BN_slides , GTK_STATE_PRELIGHT , & color ) ; 
gtk_widget_modify_bg ( BN_undo , GTK_STATE_PRELIGHT , & color ) ; 
gtk_widget_modify_bg ( BN_terminar , GTK_STATE_PRELIGHT , & color ) ; 
gtk_widget_modify_bg ( BN_print , GTK_STATE_PRELIGHT , & color ) ; 
gtk_widget_modify_bg ( BN_cancela_revision , GTK_STATE_PRELIGHT , & color ) ; 
gtk_widget_modify_bg ( BN_error_encontrado_Beamer , GTK_STATE_PRELIGHT , & color ) ; 
gtk_widget_modify_bg ( BN_confirma_revision , GTK_STATE_PRELIGHT , & color ) ; 
gtk_widget_modify_bg ( BN_cancela_revision , GTK_STATE_PRELIGHT , & color ) ; 
gtk_widget_modify_bg ( BN_ok , GTK_STATE_PRELIGHT , & color ) ; 
gdk_color_parse ( BUTTON_ACTIVE , & color ) ; 
gtk_widget_modify_bg \end{lstlisting}
\end{frame}
\begin{frame}[fragile]
\frametitle{C\'odigo Preprocesado (Sin Pretty Print)}
\begin{lstlisting}[style=CStyle]
( BN_save , GTK_STATE_ACTIVE , & color ) ; 
gtk_widget_modify_bg ( BN_slides , GTK_STATE_ACTIVE , & color ) ; 
gtk_widget_modify_bg ( BN_undo , GTK_STATE_ACTIVE , & color ) ; 
gtk_widget_modify_bg ( BN_terminar , GTK_STATE_ACTIVE , & color ) ; 
gtk_widget_modify_bg ( BN_print , GTK_STATE_ACTIVE , & color ) ; 
gtk_widget_modify_bg ( BN_cancela_revision , GTK_STATE_ACTIVE , & color ) ; 
gtk_widget_modify_bg ( BN_error_encontrado_Beamer , GTK_STATE_ACTIVE , & color ) ; 
gtk_widget_modify_bg ( BN_confirma_revision , GTK_STATE_ACTIVE , & color ) ; 
gtk_widget_modify_bg ( BN_cancela_revision , GTK_STATE_ACTIVE , & color ) ; 
gtk_widget_modify_bg ( BN_ok , GTK_STATE_ACTIVE , & color ) ; 
gdk_color_parse ( PROCESSING_PB , & color ) ; 
gtk_widget_modify_bg ( GTK_WIDGET ( PB_analisis ) , GTK_STATE_PRELIGHT , & color ) ; 
gtk_widget_modify_bg ( GTK_WIDGET ( PB_beamer ) , GTK_STATE_PRELIGHT , & color ) ; 
gtk_widget_modify_bg \end{lstlisting}
\end{frame}
\begin{frame}[fragile]
\frametitle{C\'odigo Preprocesado (Sin Pretty Print)}
\begin{lstlisting}[style=CStyle]
( GTK_WIDGET ( PB_revisando_beamer ) , GTK_STATE_PRELIGHT , & color ) ; 
gdk_color_parse ( PROCESSING_WINDOW , & color ) ; 
gtk_widget_modify_bg ( EB_generando_analisis , GTK_STATE_NORMAL , & color ) ; 
gtk_widget_modify_bg ( EB_generando_beamer , GTK_STATE_NORMAL , & color ) ; 
gtk_widget_modify_bg ( EB_revisando_beamer , GTK_STATE_NORMAL , & color ) ; 
} 
void Init_Fields ( ) 
{ 
char Examen [  6 ] ; 
char hilera [ 100 ] ; 
int month , year , day ; 
PGresult * res ; 
int i , Last ; 
GdkColor \end{lstlisting}
\end{frame}
\begin{frame}[fragile]
\frametitle{C\'odigo Preprocesado (Sin Pretty Print)}
\begin{lstlisting}[style=CStyle]
color ; 
res = PQEXEC ( DATABASE , "SELECT codigo_examen from EX_examenes order by codigo_examen DESC limit 1" ) ; 
Last = 0 ; 
if ( PQntuples ( res ) ) 
{ 
Last = atoi ( PQgetvalue ( res , 0 , 0 ) ) ; 
} 
PQclear ( res ) ; 
if ( Last > 1 ) 
{ 
gtk_widget_set_sensitive ( GTK_WIDGET ( SP_examen ) , 1 ) ; 
gtk_spin_button_set_range ( SP_examen , 1.0 , ( long double ) Last ) ; 
} 
else \end{lstlisting}
\end{frame}
\begin{frame}[fragile]
\frametitle{C\'odigo Preprocesado (Sin Pretty Print)}
\begin{lstlisting}[style=CStyle]

{ 
gtk_widget_set_sensitive ( GTK_WIDGET ( SP_examen ) , 0 ) ; 
gtk_spin_button_set_range ( SP_examen , 0.0 , ( long double ) Last ) ; 
} 
gtk_spin_button_set_value ( SP_examen , Last ) ; 
if ( preguntas ) free ( preguntas ) ; 
preguntas = NULL ; 
if ( resumen_tema ) free ( resumen_tema ) ; 
resumen_tema = NULL ; 
if ( resumen_tema_subtema ) free ( resumen_tema_subtema ) ; 
resumen_tema_subtema = NULL ; 
if ( versiones ) 
{ \end{lstlisting}
\end{frame}
\begin{frame}[fragile]
\frametitle{C\'odigo Preprocesado (Sin Pretty Print)}
\begin{lstlisting}[style=CStyle]

for ( i = 0 ; i < N_versiones ; i ++ ) free ( versiones [ i ] . preguntas ) ; 
free ( versiones ) ; 
versiones = NULL ; 
} 
gtk_entry_set_text ( GTK_ENTRY ( EN_fecha ) , "\0" ) ; 
N_preguntas = N_versiones = N_estudiantes = 0 ; 
sprintf ( hilera , "%7d" , N_preguntas ) ; 
gtk_entry_set_text ( GTK_ENTRY ( EN_N_preguntas ) , hilera ) ; 
gtk_entry_set_text ( GTK_ENTRY ( EN_N_versiones ) , hilera ) ; 
gtk_entry_set_text ( GTK_ENTRY ( EN_N_estudiantes ) , hilera ) ; 
gtk_entry_set_text ( GTK_ENTRY ( EN_descripcion ) , "\0" ) ; 
gtk_entry_set_text ( GTK_ENTRY ( EN_pre_examen ) , "\0" ) ; 
gtk_entry_set_text \end{lstlisting}
\end{frame}
\begin{frame}[fragile]
\frametitle{C\'odigo Preprocesado (Sin Pretty Print)}
\begin{lstlisting}[style=CStyle]
( GTK_ENTRY ( EN_pre_examen_descripcion ) , "\0" ) ; 
gtk_entry_set_text ( GTK_ENTRY ( EN_esquema ) , "\0" ) ; 
gtk_entry_set_text ( GTK_ENTRY ( EN_esquema_descripcion ) , "\0" ) ; 
gtk_entry_set_text ( GTK_ENTRY ( EN_materia ) , "\0" ) ; 
gtk_entry_set_text ( GTK_ENTRY ( EN_materia_descripcion ) , "\0" ) ; 
gtk_entry_set_text ( GTK_ENTRY ( EN_institucion ) , "\0" ) ; 
gtk_entry_set_text ( GTK_ENTRY ( EN_escuela ) , "\0" ) ; 
gtk_entry_set_text ( GTK_ENTRY ( EN_programa ) , "\0" ) ; 
gtk_entry_set_text ( GTK_ENTRY ( EN_profesor ) , "\0" ) ; 
gtk_widget_set_sensitive ( GTK_WIDGET ( SP_examen ) , 1 ) ; 
gtk_widget_set_sensitive ( EN_pre_examen , 0 ) ; 
gtk_widget_set_sensitive ( EN_pre_examen_descripcion , 0 ) ; 
gtk_widget_set_sensitive ( EN_esquema , 0 ) ; 
gtk_widget_set_sensitive \end{lstlisting}
\end{frame}
\begin{frame}[fragile]
\frametitle{C\'odigo Preprocesado (Sin Pretty Print)}
\begin{lstlisting}[style=CStyle]
( EN_esquema_descripcion , 0 ) ; 
gtk_widget_set_sensitive ( EN_materia , 0 ) ; 
gtk_widget_set_sensitive ( EN_materia_descripcion , 0 ) ; 
gtk_widget_set_sensitive ( EN_fecha , 0 ) ; 
gtk_widget_set_sensitive ( EN_N_preguntas , 0 ) ; 
gtk_widget_set_sensitive ( EN_N_versiones , 0 ) ; 
gtk_widget_set_sensitive ( EN_N_estudiantes , 0 ) ; 
gtk_widget_set_sensitive ( EN_institucion , 0 ) ; 
gtk_widget_set_sensitive ( EN_escuela , 0 ) ; 
gtk_widget_set_sensitive ( EN_programa , 0 ) ; 
gtk_widget_set_sensitive ( EN_profesor , 0 ) ; 
gtk_widget_set_sensitive ( EN_descripcion , 0 ) ; 
gtk_widget_set_sensitive ( FR_prediccion , 0 ) ; 
gtk_widget_set_sensitive \end{lstlisting}
\end{frame}
\begin{frame}[fragile]
\frametitle{C\'odigo Preprocesado (Sin Pretty Print)}
\begin{lstlisting}[style=CStyle]
( FR_real , 0 ) ; 
gtk_widget_set_sensitive ( FR_grafico , 0 ) ; 
gtk_widget_set_sensitive ( FR_preguntas , 0 ) ; 
gtk_widget_set_sensitive ( FR_beamer , 0 ) ; 
gtk_spin_button_set_value ( SP_resolucion , 10 ) ; 
gtk_spin_button_set_value ( SP_color , 7 ) ; 
gtk_spin_button_set_value ( SP_rotacion , 30 ) ; 
gtk_toggle_button_set_active ( CK_smooth , TRUE ) ; 
gtk_range_set_range ( GTK_RANGE ( SC_preguntas ) , ( gdouble ) 0.0 , ( gdouble ) 1.0 ) ; 
gtk_range_set_value ( GTK_RANGE ( SC_preguntas ) , ( gdouble ) 0.0 ) ; 
gtk_text_buffer_set_text ( buffer_TV_pregunta , "\0" , - 1 ) ; 
gtk_toggle_button_set_active ( TG_A , FALSE ) ; 
gtk_toggle_button_set_active ( TG_B , FALSE ) ; 
gtk_toggle_button_set_active \end{lstlisting}
\end{frame}
\begin{frame}[fragile]
\frametitle{C\'odigo Preprocesado (Sin Pretty Print)}
\begin{lstlisting}[style=CStyle]
( TG_C , FALSE ) ; 
gtk_toggle_button_set_active ( TG_D , FALSE ) ; 
gtk_toggle_button_set_active ( TG_E , FALSE ) ; 
gtk_combo_box_set_active ( CB_ajuste , - 1 ) ; 
gtk_toggle_button_set_active ( CK_no_actualiza , FALSE ) ; 
gtk_toggle_button_set_active ( CK_excluir , FALSE ) ; 
gtk_toggle_button_set_active ( CK_encoger , FALSE ) ; 
gtk_toggle_button_set_active ( CK_verbatim , FALSE ) ; 
gtk_toggle_button_set_active ( CK_header_encoger , FALSE ) ; 
gtk_toggle_button_set_active ( CK_header_verbatim , FALSE ) ; 
for ( i = 0 ; i < 5 ; i ++ ) 
{ 
gtk_toggle_button_set_active ( CK_slide [ i ] , FALSE ) ; 
gtk_toggle_button_set_active \end{lstlisting}
\end{frame}
\begin{frame}[fragile]
\frametitle{C\'odigo Preprocesado (Sin Pretty Print)}
\begin{lstlisting}[style=CStyle]
( CK_encoger_opcion [ i ] , FALSE ) ; 
gtk_toggle_button_set_active ( CK_verbatim_opcion [ i ] , FALSE ) ; 
gtk_widget_set_sensitive ( GTK_WIDGET ( CK_encoger_opcion [ i ] ) , 0 ) ; 
gtk_widget_set_sensitive ( GTK_WIDGET ( CK_verbatim_opcion [ i ] ) , 0 ) ; 
} 
gtk_widget_set_sensitive ( BN_save , 0 ) ; 
gtk_widget_set_sensitive ( BN_slides , 0 ) ; 
gtk_widget_set_sensitive ( BN_print , 0 ) ; 
gtk_widget_set_sensitive ( BN_undo , 1 ) ; 
gdk_color_parse ( SECONDARY_AREA , & color ) ; 
gtk_widget_modify_bg ( EB_ajustes , GTK_STATE_NORMAL , & color ) ; 
gtk_combo_box_set_active ( CB_estilo , parametros . Beamer_Estilo ) ; 
gtk_combo_box_set_active ( CB_color , parametros . Beamer_Color ) ; 
gtk_combo_box_set_active \end{lstlisting}
\end{frame}
\begin{frame}[fragile]
\frametitle{C\'odigo Preprocesado (Sin Pretty Print)}
\begin{lstlisting}[style=CStyle]
( CB_font , parametros . Beamer_Font ) ; 
gtk_combo_box_set_active ( CB_size , parametros . Beamer_Size ) ; 
gtk_combo_box_set_active ( CB_aspecto , parametros . Beamer_Aspecto ) ; 
gtk_toggle_button_set_active ( CK_general , FALSE ) ; 
gtk_toggle_button_set_active ( CK_sin_banderas , FALSE ) ; 
if ( Last > 1 ) 
gtk_widget_grab_focus ( GTK_WIDGET ( SP_examen ) ) ; 
else 
Cambio_Examen ( ) ; 
} 
void Cambio_Examen ( ) 
{ 
char examen [ 10 ] ; 
char \end{lstlisting}
\end{frame}
\begin{frame}[fragile]
\frametitle{C\'odigo Preprocesado (Sin Pretty Print)}
\begin{lstlisting}[style=CStyle]
hilera [ 200 ] ; 
char Descripcion [ 2 *  201 ] = "*** Examen no está registrado ***" ; 
char PG_command [ 4000 ] ; 
long double media_prediccion , desviacion_prediccion , alfa_prediccion , Rpb_prediccion ; 
PGresult * res , * res_aux , * res_aux_2 ; 
int i , j , k ; 
k = ( int ) gtk_spin_button_get_value_as_int ( SP_examen ) ; 
sprintf ( examen , "%05d" , k ) ; 
sprintf ( PG_command , "SELECT codigo_examen, descripcion, pre_examen, profesor, EX_examenes.materia, institucion, escuela, programa, esquema, descripcion_pre_examen, descripcion_esquema, descripcion_materia, nombre, prediccion_media, prediccion_desviacion, prediccion_alfa, prediccion_Rpb, Year, month, day, ejecutado from EX_examenes, EX_pre_examenes, EX_esquemas, BD_materias, BD_personas where codigo_examen = '%s' and pre_examen = codigo_pre_examen and esquema = codigo_esquema and EX_examenes.materia = codigo_materia and codigo_tema = '          ' and codigo_subtema = '          ' and profesor = codigo_persona" , examen ) ; 
res = PQEXEC ( DATABASE , PG_command ) ; 
if ( PQntuples ( res ) ) 
{ 
gtk_widget_set_sensitive ( GTK_WIDGET ( SP_examen ) , 0 ) ; 
gtk_widget_set_sensitive \end{lstlisting}
\end{frame}
\begin{frame}[fragile]
\frametitle{C\'odigo Preprocesado (Sin Pretty Print)}
\begin{lstlisting}[style=CStyle]
( EN_pre_examen , 1 ) ; 
gtk_widget_set_sensitive ( EN_pre_examen_descripcion , 1 ) ; 
gtk_widget_set_sensitive ( EN_esquema , 1 ) ; 
gtk_widget_set_sensitive ( EN_esquema_descripcion , 1 ) ; 
gtk_widget_set_sensitive ( EN_materia , 1 ) ; 
gtk_widget_set_sensitive ( EN_materia_descripcion , 1 ) ; 
gtk_widget_set_sensitive ( EN_fecha , 1 ) ; 
gtk_widget_set_sensitive ( EN_N_preguntas , 1 ) ; 
gtk_widget_set_sensitive ( EN_N_versiones , 1 ) ; 
gtk_widget_set_sensitive ( EN_N_estudiantes , 1 ) ; 
gtk_widget_set_sensitive ( EN_institucion , 1 ) ; 
gtk_widget_set_sensitive ( EN_escuela , 1 ) ; 
gtk_widget_set_sensitive ( EN_programa , 1 ) ; 
gtk_widget_set_sensitive \end{lstlisting}
\end{frame}
\begin{frame}[fragile]
\frametitle{C\'odigo Preprocesado (Sin Pretty Print)}
\begin{lstlisting}[style=CStyle]
( EN_profesor , 1 ) ; 
gtk_widget_set_sensitive ( EN_descripcion , 1 ) ; 
gtk_widget_set_sensitive ( FR_prediccion , 1 ) ; 
gtk_widget_set_sensitive ( FR_real , 1 ) ; 
gtk_widget_set_sensitive ( FR_ajustes , 1 ) ; 
gtk_widget_set_sensitive ( FR_grafico , 1 ) ; 
gtk_widget_set_sensitive ( FR_preguntas , 1 ) ; 
gtk_widget_set_sensitive ( FR_beamer , 1 ) ; 
gtk_widget_set_sensitive ( BN_save , 1 ) ; 
gtk_widget_set_sensitive ( BN_slides , 1 ) ; 
gtk_widget_set_sensitive ( BN_print , 1 ) ; 
sprintf ( PG_command , "SELECT DISTINCT version from EX_examenes_preguntas where examen = '%s' order by version" , examen ) ; 
res_aux = PQEXEC ( DATABASE , PG_command ) ; 
N_versiones \end{lstlisting}
\end{frame}
\begin{frame}[fragile]
\frametitle{C\'odigo Preprocesado (Sin Pretty Print)}
\begin{lstlisting}[style=CStyle]
= PQntuples ( res_aux ) ; 
PQclear ( res_aux ) ; 
sprintf ( PG_command , "SELECT * from EX_examenes_preguntas where examen = '%s'" , examen ) ; 
res_aux = PQEXEC ( DATABASE , PG_command ) ; 
N_preguntas = PQntuples ( res_aux ) ; 
N_preguntas = N_preguntas / N_versiones ; 
PQclear ( res_aux ) ; 
strcpy ( Descripcion , PQgetvalue ( res , 0 , 1 ) ) ; 
gtk_entry_set_text ( GTK_ENTRY ( EN_pre_examen ) , PQgetvalue ( res , 0 , 2 ) ) ; 
gtk_entry_set_text ( GTK_ENTRY ( EN_pre_examen_descripcion ) , PQgetvalue ( res , 0 , 9 ) ) ; 
gtk_entry_set_text ( GTK_ENTRY ( EN_esquema ) , PQgetvalue ( res , 0 , 8 ) ) ; 
gtk_entry_set_text ( GTK_ENTRY ( EN_esquema_descripcion ) , PQgetvalue ( res , 0 , 10 ) ) ; 
gtk_entry_set_text ( GTK_ENTRY ( EN_profesor ) , PQgetvalue ( res , 0 , 12 ) ) ; 
gtk_entry_set_text \end{lstlisting}
\end{frame}
\begin{frame}[fragile]
\frametitle{C\'odigo Preprocesado (Sin Pretty Print)}
\begin{lstlisting}[style=CStyle]
( GTK_ENTRY ( EN_materia ) , PQgetvalue ( res , 0 , 4 ) ) ; 
gtk_entry_set_text ( GTK_ENTRY ( EN_materia_descripcion ) , PQgetvalue ( res , 0 , 11 ) ) ; 
gtk_entry_set_text ( GTK_ENTRY ( EN_institucion ) , PQgetvalue ( res , 0 , 5 ) ) ; 
gtk_entry_set_text ( GTK_ENTRY ( EN_escuela ) , PQgetvalue ( res , 0 , 6 ) ) ; 
gtk_entry_set_text ( GTK_ENTRY ( EN_programa ) , PQgetvalue ( res , 0 , 7 ) ) ; 
sprintf ( hilera , "%7d" , N_preguntas ) ; 
gtk_entry_set_text ( GTK_ENTRY ( EN_N_preguntas ) , hilera ) ; 
sprintf ( hilera , "%7d" , N_versiones ) ; 
gtk_entry_set_text ( GTK_ENTRY ( EN_N_versiones ) , hilera ) ; 
media_prediccion = atof ( PQgetvalue ( res , 0 , 13 ) ) ; 
sprintf ( hilera , "%8.3Lf" , media_prediccion ) ; 
gtk_entry_set_text ( GTK_ENTRY ( EN_media_prediccion ) , hilera ) ; 
desviacion_prediccion = atof ( PQgetvalue ( res , 0 , 14 ) ) ; 
sprintf \end{lstlisting}
\end{frame}
\begin{frame}[fragile]
\frametitle{C\'odigo Preprocesado (Sin Pretty Print)}
\begin{lstlisting}[style=CStyle]
( hilera , "%8.3Lf" , desviacion_prediccion ) ; 
gtk_entry_set_text ( GTK_ENTRY ( EN_desviacion_prediccion ) , hilera ) ; 
alfa_prediccion = atof ( PQgetvalue ( res , 0 , 15 ) ) ; 
sprintf ( hilera , "%7.4Lf" , alfa_prediccion ) ; 
gtk_entry_set_text ( GTK_ENTRY ( EN_alfa_prediccion ) , hilera ) ; 
Rpb_prediccion = atof ( PQgetvalue ( res , 0 , 16 ) ) ; 
sprintf ( hilera , "%7.4Lf" , Rpb_prediccion ) ; 
gtk_entry_set_text ( GTK_ENTRY ( EN_Rpb_prediccion ) , hilera ) ; 
sprintf ( hilera , "%02d/%02d/%02d" , atoi ( PQgetvalue ( res , 0 , 19 ) ) , atoi ( PQgetvalue ( res , 0 , 18 ) ) , atoi ( PQgetvalue ( res , 0 , 17 ) ) ) ; 
gtk_entry_set_text ( GTK_ENTRY ( EN_fecha ) , hilera ) ; 
preguntas = ( struct PREGUNTA * ) malloc ( ( sizeof ( struct PREGUNTA ) * N_preguntas ) ) ; 
resumen_tema = ( struct PREGUNTA * ) malloc ( ( sizeof ( struct PREGUNTA ) * N_preguntas ) ) ; 
resumen_tema_subtema = ( struct PREGUNTA * ) malloc ( ( sizeof ( struct PREGUNTA ) * N_preguntas ) ) ; 
Construye_versiones \end{lstlisting}
\end{frame}
\begin{frame}[fragile]
\frametitle{C\'odigo Preprocesado (Sin Pretty Print)}
\begin{lstlisting}[style=CStyle]
( examen ) ; 
Carga_preguntas_examen ( ) ; 
Actualiza_Porcentajes ( ) ; 
Inicializa_Tabla_estadisticas ( ) ; 
Calcula_estadisticas_examen ( ) ; 
if ( * PQgetvalue ( res , 0 , 20 ) == 't' ) 
{ 
gtk_widget_set_sensitive ( FR_ajustes , 0 ) ; 
gtk_widget_set_sensitive ( window1 , 0 ) ; 
gtk_widget_show ( window2 ) ; 
} 
else 
{ 
gtk_widget_set_sensitive \end{lstlisting}
\end{frame}
\begin{frame}[fragile]
\frametitle{C\'odigo Preprocesado (Sin Pretty Print)}
\begin{lstlisting}[style=CStyle]
( FR_ajustes , 1 ) ; 
} 
if ( N_estudiantes ) 
gtk_widget_grab_focus ( BN_print ) ; 
else 
gtk_widget_grab_focus ( BN_undo ) ; 
} 
else 
gtk_widget_grab_focus ( GTK_WIDGET ( SP_examen ) ) ; 
PQclear ( res ) ; 
gtk_entry_set_text ( GTK_ENTRY ( EN_descripcion ) , Descripcion ) ; 
} 
void Construye_versiones ( char * examen ) 
{ \end{lstlisting}
\end{frame}
\begin{frame}[fragile]
\frametitle{C\'odigo Preprocesado (Sin Pretty Print)}
\begin{lstlisting}[style=CStyle]

int i , j , k ; 
char PG_command [ 3000 ] ; 
PGresult * res , * res_aux ; 
int respuesta_original ; 
int opcion_1 , opcion_2 , opcion_3 , opcion_4 , opcion_5 ; 
sprintf ( PG_command , "SELECT DISTINCT version from EX_examenes_preguntas where examen = '%s' order by version" , examen ) ; 
res = PQEXEC ( DATABASE , PG_command ) ; 
N_versiones = PQntuples ( res ) ; 
versiones = ( struct VERSION * ) malloc ( sizeof ( struct VERSION ) * N_versiones ) ; 
for ( i = 0 ; i < N_versiones ; i ++ ) 
{ 
strcpy ( versiones [ i ] . codigo , PQgetvalue ( res , i , 0 ) ) ; 
sprintf \end{lstlisting}
\end{frame}
\begin{frame}[fragile]
\frametitle{C\'odigo Preprocesado (Sin Pretty Print)}
\begin{lstlisting}[style=CStyle]
( PG_command , "SELECT respuesta, opcion_1, opcion_2, opcion_3, opcion_4, opcion_5, codigo_pregunta from ex_examenes_preguntas, bd_texto_preguntas where examen = '%.5s' and version = '%.4s' and codigo_pregunta = codigo_unico_pregunta order by posicion" , examen , PQgetvalue ( res , i , 0 ) ) ; 
res_aux = PQEXEC ( DATABASE , PG_command ) ; 
N_preguntas = PQntuples ( res_aux ) ; 
versiones [ i ] . preguntas = ( struct PREGUNTA_VERSION * ) malloc ( sizeof ( struct PREGUNTA_VERSION ) * ( N_preguntas ) ) ; 
for ( j = 0 ; j < N_preguntas ; j ++ ) 
{ 
respuesta_original = * PQgetvalue ( res_aux , j , 0 ) - 'A' ; 
opcion_1 = atoi ( PQgetvalue ( res_aux , j , 1 ) ) ; 
opcion_2 = atoi ( PQgetvalue ( res_aux , j , 2 ) ) ; 
opcion_3 = atoi ( PQgetvalue ( res_aux , j , 3 ) ) ; 
opcion_4 = atoi ( PQgetvalue ( res_aux , j , 4 ) ) ; 
opcion_5 = atoi ( PQgetvalue ( res_aux , j , 5 ) ) ; 
if ( respuesta_original == opcion_1 ) versiones [ i ] . preguntas [ j ] . respuesta = 'A' ; 
if \end{lstlisting}
\end{frame}
\begin{frame}[fragile]
\frametitle{C\'odigo Preprocesado (Sin Pretty Print)}
\begin{lstlisting}[style=CStyle]
( respuesta_original == opcion_2 ) versiones [ i ] . preguntas [ j ] . respuesta = 'B' ; 
if ( respuesta_original == opcion_3 ) versiones [ i ] . preguntas [ j ] . respuesta = 'C' ; 
if ( respuesta_original == opcion_4 ) versiones [ i ] . preguntas [ j ] . respuesta = 'D' ; 
if ( respuesta_original == opcion_5 ) versiones [ i ] . preguntas [ j ] . respuesta = 'E' ; 
versiones [ i ] . preguntas [ j ] . orden_version [ 0 ] = opcion_1 ; 
versiones [ i ] . preguntas [ j ] . orden_version [ 1 ] = opcion_2 ; 
versiones [ i ] . preguntas [ j ] . orden_version [ 2 ] = opcion_3 ; 
versiones [ i ] . preguntas [ j ] . orden_version [ 3 ] = opcion_4 ; 
versiones [ i ] . preguntas [ j ] . orden_version [ 4 ] = opcion_5 ; 
strcpy ( versiones [ i ] . preguntas [ j ] . codigo , PQgetvalue ( res_aux , j , 6 ) ) ; 
} 
PQclear ( res_aux ) ; 
} 
PQclear \end{lstlisting}
\end{frame}
\begin{frame}[fragile]
\frametitle{C\'odigo Preprocesado (Sin Pretty Print)}
\begin{lstlisting}[style=CStyle]
( res ) ; 
} 
void Actualiza_Porcentajes ( ) 
{ 
char PG_command [ 2000 ] ; 
PGresult * res , * res_aux ; 
int i , j , k ; 
int N_estudiantes ; 
long double Porcentaje , Porcentaje_ajustado ; 
char examen [ 10 ] ; 
int N_correctas , N_ajustado , M_ajustado ; 
k = ( int ) gtk_spin_button_get_value_as_int ( SP_examen ) ; 
sprintf ( examen , "%05d" , k ) ; 
sprintf \end{lstlisting}
\end{frame}
\begin{frame}[fragile]
\frametitle{C\'odigo Preprocesado (Sin Pretty Print)}
\begin{lstlisting}[style=CStyle]
( PG_command , "SELECT estudiante, version, respuestas from EX_examenes_respuestas where examen = '%s' order by estudiante" , examen ) ; 
res = PQEXEC ( DATABASE , PG_command ) ; 
N_estudiantes = PQntuples ( res ) ; 
for ( i = 0 ; i < N_estudiantes ; i ++ ) 
{ 
Calcula_Notas ( PQgetvalue ( res , i , 1 ) , 
PQgetvalue ( res , i , 2 ) , 
& N_correctas , & N_ajustado , & M_ajustado ) ; 
Porcentaje = ( long double ) N_correctas / N_preguntas * 100.0 ; 
if ( M_ajustado ) 
Porcentaje_ajustado = ( long double ) N_ajustado / M_ajustado * 100.0 ; 
else 
Porcentaje_ajustado = 0.0 ; 
sprintf \end{lstlisting}
\end{frame}
\begin{frame}[fragile]
\frametitle{C\'odigo Preprocesado (Sin Pretty Print)}
\begin{lstlisting}[style=CStyle]
( PG_command , "UPDATE ex_examenes_respuestas set correctas = %d, porcentaje = %Lf, nota_ajustada = %Lf, nota_final = %Lf where examen = '%s' and estudiante = %d" , 
N_correctas , Porcentaje , Porcentaje_ajustado , Porcentaje_ajustado , 
examen , i + 1 ) ; 
res_aux = PQEXEC ( DATABASE , PG_command ) ; 
} 
} 
void Carga_preguntas_examen ( ) 
{ 
int i , j , k ; 
char PG_command [ 3000 ] ; 
PGresult * res , * res_aux ; 
gchar * materia ; 
char examen [ 10 ] ; 
char \end{lstlisting}
\end{frame}
\begin{frame}[fragile]
\frametitle{C\'odigo Preprocesado (Sin Pretty Print)}
\begin{lstlisting}[style=CStyle]
ejercicio_actual [ 10 ] ; 
struct PREGUNTA temporal ; 
unsigned int ajustes_Beamer ; 
k = ( int ) gtk_spin_button_get_value_as_int ( SP_examen ) ; 
sprintf ( examen , "%05d" , k ) ; 
materia = gtk_editable_get_chars ( GTK_EDITABLE ( EN_materia ) , 0 , - 1 ) ; 
sprintf ( PG_command , "SELECT tema, subtema, ejercicio, secuencia, codigo_pregunta, dificultad, BD_texto_preguntas.respuesta, texto_pregunta, nombre from EX_examenes_preguntas, BD_ejercicios, BD_texto_preguntas, BD_estadisticas_preguntas, BD_texto_ejercicios, BD_personas where examen = '%s' and codigo_ejercicio = ejercicio and materia = '%s' and codigo_pregunta = pregunta and codigo_unico_pregunta = pregunta and consecutivo_texto = texto_ejercicio and codigo_persona = autor order by version, tema, subtema, ejercicio, secuencia" , examen , materia ) ; 
res = PQEXEC ( DATABASE , PG_command ) ; 
N_ajustes = 0 ; 
for ( i = 0 ; i < N_preguntas ; i ++ ) 
{ 
strcpy ( preguntas [ i ] . tema , PQgetvalue ( res , i , 0 ) ) ; 
strcpy ( preguntas [ i ] . subtema , PQgetvalue ( res , i , 1 ) ) ; 
strcpy \end{lstlisting}
\end{frame}
\begin{frame}[fragile]
\frametitle{C\'odigo Preprocesado (Sin Pretty Print)}
\begin{lstlisting}[style=CStyle]
( preguntas [ i ] . ejercicio , PQgetvalue ( res , i , 2 ) ) ; 
preguntas [ i ] . secuencia = atoi ( PQgetvalue ( res , i , 3 ) ) ; 
strcpy ( preguntas [ i ] . pregunta , PQgetvalue ( res , i , 4 ) ) ; 
preguntas [ i ] . previo = atof ( PQgetvalue ( res , i , 5 ) ) ; 
preguntas [ i ] . correcta = * PQgetvalue ( res , i , 6 ) ; 
strncpy ( preguntas [ i ] . texto_pregunta , PQgetvalue ( res , i , 7 ) ,  501 - 1 ) ; 
preguntas [ i ] . texto_pregunta [  501 - 1 ] = '\0' ; 
strcpy ( preguntas [ i ] . autor , PQgetvalue ( res , i , 8 ) ) ; 
preguntas [ i ] . desviacion = sqrt ( preguntas [ i ] . previo * ( 1.0 - preguntas [ i ] . previo ) ) ; 
preguntas [ i ] . ajuste = 0 ; 
preguntas [ i ] . revision_especial = 0 ; 
preguntas [ i ] . actualizar = 1 ; 
preguntas [ i ] . excluir = 0 ; 
preguntas \end{lstlisting}
\end{frame}
\begin{frame}[fragile]
\frametitle{C\'odigo Preprocesado (Sin Pretty Print)}
\begin{lstlisting}[style=CStyle]
[ i ] . encoger = 0 ; 
preguntas [ i ] . verbatim = 0 ; 
preguntas [ i ] . header_encoger = 0 ; 
preguntas [ i ] . header_verbatim = 0 ; 
for ( j = 0 ; j < 5 ; j ++ ) 
{ 
preguntas [ i ] . correctas_nuevas [ j ] = 0 ; 
preguntas [ i ] . slide [ j ] = 0 ; 
preguntas [ i ] . encoger_opcion [ j ] = 0 ; 
preguntas [ i ] . verbatim_opcion [ j ] = 0 ; 
} 
sprintf ( PG_command , "SELECT * from EX_examenes_ajustes where examen = '%s' and codigo_pregunta = '%s'" , examen , PQgetvalue ( res , i , 4 ) ) ; 
res_aux = PQEXEC ( DATABASE , PG_command ) ; 
if \end{lstlisting}
\end{frame}
\begin{frame}[fragile]
\frametitle{C\'odigo Preprocesado (Sin Pretty Print)}
\begin{lstlisting}[style=CStyle]
( PQntuples ( res_aux ) ) 
{ 
preguntas [ i ] . ajuste = atoi ( PQgetvalue ( res_aux , 0 , 3 ) ) ; 
preguntas [ i ] . revision_especial = preguntas [ i ] . ajuste & 0xFF ; 
ajustes_Beamer = preguntas [ i ] . ajuste >> 8 ; 
preguntas [ i ] . excluir = ajustes_Beamer & 0x01 ; ajustes_Beamer >>= 1 ; 
preguntas [ i ] . encoger = ajustes_Beamer & 0x01 ; ajustes_Beamer >>= 1 ; 
preguntas [ i ] . verbatim = ajustes_Beamer & 0x01 ; ajustes_Beamer >>= 1 ; 
preguntas [ i ] . header_encoger = ajustes_Beamer & 0x01 ; ajustes_Beamer >>= 1 ; 
preguntas [ i ] . header_verbatim = ajustes_Beamer & 0x01 ; ajustes_Beamer >>= 1 ; 
for ( j = 0 ; j < 5 ; j ++ ) 
{ 
preguntas [ i ] . slide [ j ] = ajustes_Beamer & 0x01 ; ajustes_Beamer >>= 1 ; 
preguntas \end{lstlisting}
\end{frame}
\begin{frame}[fragile]
\frametitle{C\'odigo Preprocesado (Sin Pretty Print)}
\begin{lstlisting}[style=CStyle]
[ i ] . encoger_opcion [ j ] = ajustes_Beamer & 0x01 ; ajustes_Beamer >>= 1 ; 
preguntas [ i ] . verbatim_opcion [ j ] = ajustes_Beamer & 0x01 ; ajustes_Beamer >>= 1 ; 
} 
preguntas [ i ] . correctas_nuevas [ 0 ] = atoi ( PQgetvalue ( res_aux , 0 , 4 ) ) ; 
preguntas [ i ] . correctas_nuevas [ 1 ] = atoi ( PQgetvalue ( res_aux , 0 , 5 ) ) ; 
preguntas [ i ] . correctas_nuevas [ 2 ] = atoi ( PQgetvalue ( res_aux , 0 , 6 ) ) ; 
preguntas [ i ] . correctas_nuevas [ 3 ] = atoi ( PQgetvalue ( res_aux , 0 , 7 ) ) ; 
preguntas [ i ] . correctas_nuevas [ 4 ] = atoi ( PQgetvalue ( res_aux , 0 , 8 ) ) ; 
preguntas [ i ] . actualizar = atoi ( PQgetvalue ( res_aux , 0 , 9 ) ) ; 
if ( preguntas [ i ] . revision_especial ) N_ajustes ++ ; 
} 
PQclear ( res_aux ) ; 
sprintf ( PG_command , "SELECT orden from BD_materias where codigo_materia = '%s' and codigo_tema = '%s' and codigo_subtema = '          '" , materia , PQgetvalue ( res , i , 0 ) ) ; 
res_aux \end{lstlisting}
\end{frame}
\begin{frame}[fragile]
\frametitle{C\'odigo Preprocesado (Sin Pretty Print)}
\begin{lstlisting}[style=CStyle]
= PQEXEC ( DATABASE , PG_command ) ; 
preguntas [ i ] . orden_tema = atoi ( PQgetvalue ( res_aux , 0 , 0 ) ) ; 
PQclear ( res_aux ) ; 
sprintf ( PG_command , "SELECT orden from BD_materias where codigo_materia = '%s' and codigo_tema = '%s' and codigo_subtema = '%s'" , materia , PQgetvalue ( res , i , 0 ) , PQgetvalue ( res , i , 1 ) ) ; 
res_aux = PQEXEC ( DATABASE , PG_command ) ; 
preguntas [ i ] . orden_subtema = atoi ( PQgetvalue ( res_aux , 0 , 0 ) ) ; 
PQclear ( res_aux ) ; 
} 
PQclear ( res ) ; 
for ( i = N_preguntas ; i >= 0 ; i -- ) 
for ( j = 0 ; j < ( i - 1 ) ; j ++ ) 
{ 
if ( ( preguntas [ j ] . orden_tema > preguntas [ j + 1 ] . orden_tema ) || 
( \end{lstlisting}
\end{frame}
\begin{frame}[fragile]
\frametitle{C\'odigo Preprocesado (Sin Pretty Print)}
\begin{lstlisting}[style=CStyle]
( preguntas [ j ] . orden_tema == preguntas [ j + 1 ] . orden_tema ) && 
( preguntas [ j ] . orden_subtema > preguntas [ j + 1 ] . orden_subtema ) ) || 
( ( preguntas [ j ] . orden_tema == preguntas [ j + 1 ] . orden_tema ) && 
( preguntas [ j ] . orden_subtema == preguntas [ j + 1 ] . orden_subtema ) && 
strcmp ( preguntas [ j ] . ejercicio , preguntas [ j + 1 ] . ejercicio ) > 0 ) ) 
{ 
temporal = preguntas [ j + 1 ] ; 
preguntas [ j + 1 ] = preguntas [ j ] ; 
preguntas [ j ] = temporal ; 
} 
} 
g_free ( materia ) ; 
} 
void \end{lstlisting}
\end{frame}
\begin{frame}[fragile]
\frametitle{C\'odigo Preprocesado (Sin Pretty Print)}
\begin{lstlisting}[style=CStyle]
Inicializa_Tabla_estadisticas ( ) 
{ 
int i , j ; 
char ejercicio_actual [ 10 ] ; 
int inicio_actual ; 
inicio_actual = 0 ; 
strcpy ( ejercicio_actual , preguntas [ 0 ] . ejercicio ) ; 
for ( i = 0 ; i < N_preguntas ; i ++ ) 
{ 
if ( strcmp ( ejercicio_actual , preguntas [ i ] . ejercicio ) != 0 ) 
{ 
strcpy ( ejercicio_actual , preguntas [ i ] . ejercicio ) ; 
for ( j = inicio_actual ; j < i ; j ++ ) 
preguntas \end{lstlisting}
\end{frame}
\begin{frame}[fragile]
\frametitle{C\'odigo Preprocesado (Sin Pretty Print)}
\begin{lstlisting}[style=CStyle]
[ j ] . grupo_final = i - 1 ; 
inicio_actual = i ; 
} 
preguntas [ i ] . grupo_inicio = inicio_actual ; 
preguntas [ i ] . buenos = 0 ; 
preguntas [ i ] . malos = 0 ; 
for ( j = 0 ; j < 5 ; j ++ ) 
{ 
preguntas [ i ] . acumulado_opciones [ j ] = 0 ; 
preguntas [ i ] . suma_seleccion [ j ] = 0.0 ; 
preguntas [ i ] . Rpb_opcion [ j ] = 0.0 ; 
} 
preguntas [ i ] . suma_buenos = 0.0 ; 
preguntas \end{lstlisting}
\end{frame}
\begin{frame}[fragile]
\frametitle{C\'odigo Preprocesado (Sin Pretty Print)}
\begin{lstlisting}[style=CStyle]
[ i ] . suma_malos = 0.0 ; 
} 
for ( i = inicio_actual ; i < N_preguntas ; i ++ ) 
preguntas [ i ] . grupo_final = N_preguntas - 1 ; 
} 
void Calcula_estadisticas_examen ( ) 
{ 
int i , j , k , N , version_actual ; 
int buenas , malas ; 
char PG_command [ 3000 ] ; 
PGresult * res , * res_aux , * res_aux_2 ; 
char examen [ 10 ] ; 
char hilera [ 100 ] ; 
long \end{lstlisting}
\end{frame}
\begin{frame}[fragile]
\frametitle{C\'odigo Preprocesado (Sin Pretty Print)}
\begin{lstlisting}[style=CStyle]
double media , diferencia , varianza , desviacion , varianza_pregunta , desviacion_pregunta , suma_varianzas , alfa , Rpb ; 
long double media_buenos , media_malos ; 
long double media_sin , varianza_sin , alfa_sin ; 
long double nota ; 
char tema_actual [ CODIGO_TEMA_SIZE + 1 ] ; 
char tema_subtema_actual [ CODIGO_SUBTEMA_SIZE + 1 ] ; 
int buenas_tema , malas_tema , buenas_subtema , malas_subtema ; 
struct PREGUNTA temporal ; 
int opcion_original [ 5 ] ; 
k = ( int ) gtk_spin_button_get_value_as_int ( SP_examen ) ; 
sprintf ( examen , "%05d" , k ) ; 
for ( i = 0 ; i <  10 ; i ++ ) Frecuencias [ i ] = 0 ; 
sprintf ( PG_command , "SELECT porcentaje, version, respuestas, estudiante, nota_ajustada from EX_examenes_respuestas where examen = '%s' order by estudiante" , examen ) ; 
res \end{lstlisting}
\end{frame}
\begin{frame}[fragile]
\frametitle{C\'odigo Preprocesado (Sin Pretty Print)}
\begin{lstlisting}[style=CStyle]
= PQEXEC ( DATABASE , PG_command ) ; 
N_estudiantes = PQntuples ( res ) ; 
sprintf ( hilera , "%7d" , N_estudiantes ) ; 
gtk_entry_set_text ( GTK_ENTRY ( EN_N_estudiantes ) , hilera ) ; 
if ( N_estudiantes <= 1 ) 
{ 
gtk_widget_set_sensitive ( FR_real , 0 ) ; 
gtk_widget_set_sensitive ( FR_ajustes , 0 ) ; 
gtk_widget_set_sensitive ( BN_save , 0 ) ; 
gtk_widget_set_sensitive ( BN_slides , 0 ) ; 
gtk_widget_set_sensitive ( BN_print , 0 ) ; 
} 
else 
{ \end{lstlisting}
\end{frame}
\begin{frame}[fragile]
\frametitle{C\'odigo Preprocesado (Sin Pretty Print)}
\begin{lstlisting}[style=CStyle]

gtk_range_set_range ( GTK_RANGE ( SC_preguntas ) , ( gdouble ) 1.0 , ( gdouble ) N_preguntas ) ; 
gtk_range_set_value ( GTK_RANGE ( SC_preguntas ) , ( gdouble ) 1.0 ) ; 
media = 0.0 ; 
Nota_minima = 100.0 ; 
Nota_maxima = 0.0 ; 
Nota_minima_ajustada = 100.0 ; 
Nota_maxima_ajustada = 0.0 ; 
for ( i = 0 ; i < N_estudiantes ; i ++ ) 
{ 
nota = atof ( PQgetvalue ( res , i , 0 ) ) ; 
media += nota ; 
if ( nota < Nota_minima ) Nota_minima = nota ; 
if \end{lstlisting}
\end{frame}
\begin{frame}[fragile]
\frametitle{C\'odigo Preprocesado (Sin Pretty Print)}
\begin{lstlisting}[style=CStyle]
( nota > Nota_maxima ) Nota_maxima = nota ; 
j = nota / ( 100.0 / ( long double )  10 ) ; 
Frecuencias [ j ] ++ ; 
nota = atof ( PQgetvalue ( res , i , 4 ) ) ; 
if ( nota < Nota_minima_ajustada ) Nota_minima_ajustada = nota ; 
if ( nota > Nota_maxima_ajustada ) Nota_maxima_ajustada = nota ; 
} 
media /= N_estudiantes ; 
sprintf ( hilera , "%8.3Lf" , media ) ; 
gtk_entry_set_text ( GTK_ENTRY ( EN_media_real ) , hilera ) ; 
varianza = 0.0 ; 
for ( i = 0 ; i < N_estudiantes ; i ++ ) 
{ 
diferencia \end{lstlisting}
\end{frame}
\begin{frame}[fragile]
\frametitle{C\'odigo Preprocesado (Sin Pretty Print)}
\begin{lstlisting}[style=CStyle]
= media - atof ( PQgetvalue ( res , i , 0 ) ) ; 
varianza += ( diferencia * diferencia ) ; 
} 
varianza /= N_estudiantes ; 
desviacion = sqrt ( varianza ) ; 
sprintf ( hilera , "%8.3Lf" , desviacion ) ; 
gtk_entry_set_text ( GTK_ENTRY ( EN_desviacion_real ) , hilera ) ; 
for ( i = 0 ; i < N_estudiantes ; i ++ ) 
{ 
sprintf ( PG_command , "SELECT ejercicio, secuencia, opcion_1, opcion_2, opcion_3, opcion_4, opcion_5 from EX_examenes_preguntas where examen = '%s' and version = '%s' order by posicion" , examen , PQgetvalue ( res , i , 1 ) ) ; 
res_aux = PQEXEC ( DATABASE , PG_command ) ; 
N = PQntuples ( res_aux ) ; 
for ( version_actual = 0 ; ( version_actual < N_versiones ) && strcmp ( versiones [ version_actual ] . codigo , PQgetvalue ( res , i , 1 ) ) != 0 ; version_actual ++ ) ; 
buenas \end{lstlisting}
\end{frame}
\begin{frame}[fragile]
\frametitle{C\'odigo Preprocesado (Sin Pretty Print)}
\begin{lstlisting}[style=CStyle]
= malas = 0 ; 
for ( j = 0 ; j < N_preguntas ; j ++ ) 
{ 
opcion_original [ 0 ] = atoi ( PQgetvalue ( res_aux , j , 2 ) ) ; 
opcion_original [ 1 ] = atoi ( PQgetvalue ( res_aux , j , 3 ) ) ; 
opcion_original [ 2 ] = atoi ( PQgetvalue ( res_aux , j , 4 ) ) ; 
opcion_original [ 3 ] = atoi ( PQgetvalue ( res_aux , j , 5 ) ) ; 
opcion_original [ 4 ] = atoi ( PQgetvalue ( res_aux , j , 6 ) ) ; 
if ( versiones [ version_actual ] . preguntas [ j ] . respuesta == PQgetvalue ( res , i , 2 ) [ j ] ) 
{ 
buenas ++ ; 
} 
else 
{ \end{lstlisting}
\end{frame}
\begin{frame}[fragile]
\frametitle{C\'odigo Preprocesado (Sin Pretty Print)}
\begin{lstlisting}[style=CStyle]

malas ++ ; 
} 
Actualice_Estadisticas ( versiones [ version_actual ] . preguntas [ j ] . respuesta == PQgetvalue ( res , i , 2 ) [ j ] , 
PQgetvalue ( res_aux , j , 0 ) , atoi ( PQgetvalue ( res_aux , j , 1 ) ) , ( long double ) atof ( PQgetvalue ( res , i , 0 ) ) , 
PQgetvalue ( res , i , 2 ) [ j ] , opcion_original ) ; 
} 
PQclear ( res_aux ) ; 
} 
Rpb = 0.0 ; 
suma_varianzas = 0.0 ; 
for ( i = 0 ; i < N_preguntas ; i ++ ) 
{ 
preguntas \end{lstlisting}
\end{frame}
\begin{frame}[fragile]
\frametitle{C\'odigo Preprocesado (Sin Pretty Print)}
\begin{lstlisting}[style=CStyle]
[ i ] . porcentaje = ( long double ) preguntas [ i ] . buenos / N_estudiantes ; 
varianza_pregunta = preguntas [ i ] . porcentaje * ( 1.0 - preguntas [ i ] . porcentaje ) ; 
desviacion_pregunta = sqrt ( varianza_pregunta ) ; 
suma_varianzas += varianza_pregunta ; 
media_buenos = 0.0 ; 
if ( preguntas [ i ] . buenos ) media_buenos = preguntas [ i ] . suma_buenos / preguntas [ i ] . buenos ; 
media_malos = 0.0 ; 
if ( preguntas [ i ] . malos ) media_malos = preguntas [ i ] . suma_malos / preguntas [ i ] . malos ; 
preguntas [ i ] . Rpb = ( media_buenos - media_malos ) * ( desviacion_pregunta / desviacion ) ; 
Rpb += preguntas [ i ] . Rpb ; 
for ( j = 0 ; j < 5 ; j ++ ) 
{ 
if ( ( preguntas [ i ] . acumulado_opciones [ j ] == 0 ) || ( preguntas [ i ] . acumulado_opciones [ j ] == N_estudiantes ) ) 
{ \end{lstlisting}
\end{frame}
\begin{frame}[fragile]
\frametitle{C\'odigo Preprocesado (Sin Pretty Print)}
\begin{lstlisting}[style=CStyle]

preguntas [ i ] . Rpb_opcion [ j ] = 0.0 ; 
} 
else 
{ 
preguntas [ i ] . Rpb_opcion [ j ] = ( ( ( preguntas [ i ] . suma_seleccion [ j ] / preguntas [ i ] . acumulado_opciones [ j ] ) - 
( ( preguntas [ i ] . suma_buenos + preguntas [ i ] . suma_malos - preguntas [ i ] . suma_seleccion [ j ] ) / 
( N_estudiantes - preguntas [ i ] . acumulado_opciones [ j ] ) ) ) / 
desviacion ) * 
sqrt ( ( ( long double ) preguntas [ i ] . acumulado_opciones [ j ] / N_estudiantes ) * 
( 1 - ( ( long double ) preguntas [ i ] . acumulado_opciones [ j ] / N_estudiantes ) ) ) ; 
} 
} 
} \end{lstlisting}
\end{frame}
\begin{frame}[fragile]
\frametitle{C\'odigo Preprocesado (Sin Pretty Print)}
\begin{lstlisting}[style=CStyle]

alfa = ( ( long double ) N_preguntas / ( long double ) ( N_preguntas - 1 ) ) * ( 1.0 - ( suma_varianzas / varianza ) ) ; 
sprintf ( hilera , "%8.3Lf" , alfa ) ; 
gtk_entry_set_text ( GTK_ENTRY ( EN_alfa_real ) , hilera ) ; 
Rpb /= N_preguntas ; 
sprintf ( hilera , "%8.3Lf" , Rpb ) ; 
gtk_entry_set_text ( GTK_ENTRY ( EN_Rpb_real ) , hilera ) ; 
for ( i = 0 ; i < N_preguntas ; i ++ ) 
{ 
media_sin = 0.0 ; 
for ( j = 0 ; j < N_estudiantes ; j ++ ) 
{ 
sprintf ( PG_command , "SELECT version from EX_versiones where examen = '%s' and version <= '%s'" , examen , PQgetvalue ( res , j , 1 ) ) ; 
res_aux_2 \end{lstlisting}
\end{frame}
\begin{frame}[fragile]
\frametitle{C\'odigo Preprocesado (Sin Pretty Print)}
\begin{lstlisting}[style=CStyle]
= PQEXEC ( DATABASE , PG_command ) ; 
version_actual = PQntuples ( res_aux_2 ) - 1 ; 
PQclear ( res_aux_2 ) ; 
N = 0 ; 
for ( k = 0 ; k < N_preguntas ; k ++ ) N += ( versiones [ version_actual ] . preguntas [ k ] . respuesta == PQgetvalue ( res , j , 2 ) [ k ] ) ; 
N -= ( versiones [ version_actual ] . preguntas [ i ] . respuesta == PQgetvalue ( res , j , 2 ) [ i ] ) ; 
media_sin += ( ( long double ) N * 100.0 / ( N_preguntas - 1 ) ) ; 
} 
media_sin /= N_estudiantes ; 
varianza_sin = 0.0 ; 
for ( j = 0 ; j < N_estudiantes ; j ++ ) 
{ 
sprintf ( PG_command , "SELECT version from EX_versiones where examen = '%s' and version <= '%s'" , examen , PQgetvalue ( res , j , 1 ) ) ; 
res_aux_2 \end{lstlisting}
\end{frame}
\begin{frame}[fragile]
\frametitle{C\'odigo Preprocesado (Sin Pretty Print)}
\begin{lstlisting}[style=CStyle]
= PQEXEC ( DATABASE , PG_command ) ; 
version_actual = PQntuples ( res_aux_2 ) - 1 ; 
PQclear ( res_aux_2 ) ; 
N = 0 ; 
for ( k = 0 ; k < N_preguntas ; k ++ ) N += ( versiones [ version_actual ] . preguntas [ k ] . respuesta == PQgetvalue ( res , j , 2 ) [ k ] ) ; 
N -= ( versiones [ version_actual ] . preguntas [ i ] . respuesta == PQgetvalue ( res , j , 2 ) [ i ] ) ; 
varianza_sin += ( ( media_sin - ( ( long double ) N * 100.0 / ( N_preguntas - 1 ) ) ) * ( media_sin - ( ( long double ) N * 100.0 / ( N_preguntas - 1 ) ) ) ) ; 
} 
varianza_sin /= N_estudiantes ; 
varianza_pregunta = ( ( ( long double ) preguntas [ i ] . buenos / N_estudiantes ) * ( ( long double ) preguntas [ i ] . malos / N_estudiantes ) ) ; 
preguntas [ i ] . alfa_sin = ( ( long double ) ( N_preguntas - 1 ) / ( long double ) ( N_preguntas - 2 ) ) * ( 1.0 - ( ( suma_varianzas - varianza_pregunta ) / varianza_sin ) ) ; 
} 
i = 0 ; 
N_temas \end{lstlisting}
\end{frame}
\begin{frame}[fragile]
\frametitle{C\'odigo Preprocesado (Sin Pretty Print)}
\begin{lstlisting}[style=CStyle]
= 0 ; 
N_subtemas = 0 ; 
while ( i < N_preguntas ) 
{ 
strcpy ( resumen_tema [ N_temas ] . tema , preguntas [ i ] . tema ) ; 
strcpy ( resumen_tema [ N_temas ] . subtema , CODIGO_VACIO ) ; 
resumen_tema [ N_temas ] . buenos = 0 ; 
resumen_tema [ N_temas ] . malos = 0 ; 
while ( ( i < N_preguntas ) && ( strcmp ( resumen_tema [ N_temas ] . tema , preguntas [ i ] . tema ) == 0 ) ) 
{ 
strcpy ( resumen_tema [ N_temas ] . subtema , preguntas [ i ] . subtema ) ; 
strcpy ( resumen_tema_subtema [ N_subtemas ] . tema , preguntas [ i ] . tema ) ; 
strcpy ( resumen_tema_subtema [ N_subtemas ] . subtema , preguntas [ i ] . subtema ) ; 
resumen_tema_subtema \end{lstlisting}
\end{frame}
\begin{frame}[fragile]
\frametitle{C\'odigo Preprocesado (Sin Pretty Print)}
\begin{lstlisting}[style=CStyle]
[ N_subtemas ] . buenos = 0 ; 
resumen_tema_subtema [ N_subtemas ] . malos = 0 ; 
while ( ( i < N_preguntas ) && 
( strcmp ( resumen_tema [ N_temas ] . tema , preguntas [ i ] . tema ) == 0 ) && 
( strcmp ( resumen_tema_subtema [ N_subtemas ] . subtema , preguntas [ i ] . subtema ) == 0 ) ) 
{ 
resumen_tema_subtema [ N_subtemas ] . buenos += preguntas [ i ] . buenos ; 
resumen_tema_subtema [ N_subtemas ] . malos += preguntas [ i ] . malos ; 
i ++ ; 
} 
resumen_tema [ N_temas ] . buenos += resumen_tema_subtema [ N_subtemas ] . buenos ; 
resumen_tema [ N_temas ] . malos += resumen_tema_subtema [ N_subtemas ] . malos ; 
N_subtemas ++ ; 
} \end{lstlisting}
\end{frame}
\begin{frame}[fragile]
\frametitle{C\'odigo Preprocesado (Sin Pretty Print)}
\begin{lstlisting}[style=CStyle]

N_temas ++ ; 
} 
for ( i = 0 ; i < N_temas ; i ++ ) resumen_tema [ i ] . porcentaje = ( long double ) resumen_tema [ i ] . buenos / ( resumen_tema [ i ] . buenos + resumen_tema [ i ] . malos ) ; 
for ( i = N_temas ; i >= 0 ; i -- ) 
for ( j = 0 ; j < ( i - 1 ) ; j ++ ) 
{ 
if ( resumen_tema [ j ] . porcentaje > resumen_tema [ j + 1 ] . porcentaje ) 
{ 
temporal = resumen_tema [ j + 1 ] ; 
resumen_tema [ j + 1 ] = resumen_tema [ j ] ; 
resumen_tema [ j ] = temporal ; 
} 
} \end{lstlisting}
\end{frame}
\begin{frame}[fragile]
\frametitle{C\'odigo Preprocesado (Sin Pretty Print)}
\begin{lstlisting}[style=CStyle]

for ( i = 0 ; i < N_subtemas ; i ++ ) resumen_tema_subtema [ i ] . porcentaje = ( long double ) resumen_tema_subtema [ i ] . buenos / ( resumen_tema_subtema [ i ] . buenos + resumen_tema_subtema [ i ] . malos ) ; 
for ( i = N_subtemas ; i >= 0 ; i -- ) 
for ( j = 0 ; j < ( i - 1 ) ; j ++ ) 
{ 
if ( resumen_tema_subtema [ j ] . porcentaje > resumen_tema_subtema [ j + 1 ] . porcentaje ) 
{ 
temporal = resumen_tema_subtema [ j + 1 ] ; 
resumen_tema_subtema [ j + 1 ] = resumen_tema_subtema [ j ] ; 
resumen_tema_subtema [ j ] = temporal ; 
} 
} 
} 
PQclear \end{lstlisting}
\end{frame}
\begin{frame}[fragile]
\frametitle{C\'odigo Preprocesado (Sin Pretty Print)}
\begin{lstlisting}[style=CStyle]
( res ) ; 
} 
void Actualice_Estadisticas ( int buena , char * ejercicio , int secuencia , long double Nota , char respuesta , int opcion_original [ 5 ] ) 
{ 
int i , j ; 
int found ; 
for ( i = 0 ; ( i < N_preguntas ) && ! ( ( strcmp ( ejercicio , preguntas [ i ] . ejercicio ) == 0 ) && ( preguntas [ i ] . secuencia == secuencia ) ) ; i ++ ) ; 
if ( buena ) 
{ 
preguntas [ i ] . buenos ++ ; 
preguntas [ i ] . suma_buenos += Nota ; 
} 
else 
{ \end{lstlisting}
\end{frame}
\begin{frame}[fragile]
\frametitle{C\'odigo Preprocesado (Sin Pretty Print)}
\begin{lstlisting}[style=CStyle]

preguntas [ i ] . malos ++ ; 
preguntas [ i ] . suma_malos += Nota ; 
} 
if ( ( respuesta >= 'A' ) && ( respuesta <= 'E' ) ) 
{ 
preguntas [ i ] . acumulado_opciones [ opcion_original [ respuesta - 'A' ] ] ++ ; 
preguntas [ i ] . suma_seleccion [ opcion_original [ respuesta - 'A' ] ] += Nota ; 
} 
} 
void Imprime_Reporte ( ) 
{ 
int k ; 
FILE \end{lstlisting}
\end{frame}
\begin{frame}[fragile]
\frametitle{C\'odigo Preprocesado (Sin Pretty Print)}
\begin{lstlisting}[style=CStyle]
* Archivo_Latex ; 
long double media_real , desviacion_real , media_prediccion , desviacion_prediccion , alfa , Rpb , width ; 
char hilera_antes [ 3000 ] , hilera_despues [ 3000 ] ; 
gchar * materia , * materia_descripcion , * descripcion , * version , * institucion , * escuela , * programa , * fecha , * profesor ; 
char Directorio [ 2000 ] ; 
char comando [ 200 ] ; 
char codigo [ 10 ] ; 
DIR * pdir ; 
gtk_widget_set_sensitive ( window1 , 0 ) ; 
gtk_widget_show ( window4 ) ; 
gtk_progress_bar_set_fraction ( GTK_PROGRESS_BAR ( PB_analisis ) , 0.0 ) ; 
while ( gtk_events_pending ( ) ) gtk_main_iteration ( ) ; 
Banderas [ 0 ] = Banderas [ 1 ] = Banderas [ 2 ] = Banderas [ 3 ] = Banderas [ 4 ] = Banderas [ 5 ] = 0 ; 
k \end{lstlisting}
\end{frame}
\begin{frame}[fragile]
\frametitle{C\'odigo Preprocesado (Sin Pretty Print)}
\begin{lstlisting}[style=CStyle]
= ( int ) gtk_spin_button_get_value_as_int ( SP_examen ) ; 
sprintf ( codigo , "%05d" , k ) ; 
materia = gtk_editable_get_chars ( GTK_EDITABLE ( EN_materia ) , 0 , - 1 ) ; 
materia_descripcion = gtk_editable_get_chars ( GTK_EDITABLE ( EN_materia_descripcion ) , 0 , - 1 ) ; 
descripcion = gtk_editable_get_chars ( GTK_EDITABLE ( EN_descripcion ) , 0 , - 1 ) ; 
institucion = gtk_editable_get_chars ( GTK_EDITABLE ( EN_institucion ) , 0 , - 1 ) ; 
escuela = gtk_editable_get_chars ( GTK_EDITABLE ( EN_escuela ) , 0 , - 1 ) ; 
programa = gtk_editable_get_chars ( GTK_EDITABLE ( EN_programa ) , 0 , - 1 ) ; 
profesor = gtk_editable_get_chars ( GTK_EDITABLE ( EN_profesor ) , 0 , - 1 ) ; 
fecha = gtk_editable_get_chars ( GTK_EDITABLE ( EN_fecha ) , 0 , - 1 ) ; 
media_real = atof ( gtk_editable_get_chars ( GTK_EDITABLE ( EN_media_real ) , 0 , - 1 ) ) ; 
desviacion_real = atof ( gtk_editable_get_chars ( GTK_EDITABLE ( EN_desviacion_real ) , 0 , - 1 ) ) ; 
media_prediccion = atof ( gtk_editable_get_chars ( GTK_EDITABLE ( EN_media_prediccion ) , 0 , - 1 ) ) ; 
desviacion_prediccion \end{lstlisting}
\end{frame}
\begin{frame}[fragile]
\frametitle{C\'odigo Preprocesado (Sin Pretty Print)}
\begin{lstlisting}[style=CStyle]
= atof ( gtk_editable_get_chars ( GTK_EDITABLE ( EN_desviacion_prediccion ) , 0 , - 1 ) ) ; 
alfa = atof ( gtk_editable_get_chars ( GTK_EDITABLE ( EN_alfa_real ) , 0 , - 1 ) ) ; 
Rpb = atof ( gtk_editable_get_chars ( GTK_EDITABLE ( EN_Rpb_real ) , 0 , - 1 ) ) ; 
Establece_Directorio ( Directorio , materia , fecha + 6 , fecha + 3 , fecha ) ; 
gtk_progress_bar_set_fraction ( GTK_PROGRESS_BAR ( PB_analisis ) , 0.05 ) ; 
while ( gtk_events_pending ( ) ) gtk_main_iteration ( ) ; 
Archivo_Latex = fopen ( "analisis.tex" , "w" ) ; 
fprintf ( Archivo_Latex , "documentclass[9pt,journal, twoside, onecolumn]{EXAMINER}\n" ) ; 
EX_latex_packages ( Archivo_Latex ) ; 
fprintf ( Archivo_Latex , "\n%s\n\n" , parametros . Paquetes ) ; 
fprintf ( Archivo_Latex , "graphicspath{{%s/}}\n\n" , parametros . ruta_figuras ) ; 
fprintf ( Archivo_Latex , "SetWatermarkText{}\n" ) ; 
fprintf ( Archivo_Latex , "pagestyle{fancy}\n" ) ; 
fprintf \end{lstlisting}
\end{frame}
\begin{frame}[fragile]
\frametitle{C\'odigo Preprocesado (Sin Pretty Print)}
\begin{lstlisting}[style=CStyle]
( Archivo_Latex , "fancyhead{}\n" ) ; 
fprintf ( Archivo_Latex , "fancyhead[LO, RE]{%s}\n" , fecha ) ; 
fprintf ( Archivo_Latex , "fancyhead[LE,RO]{thepage}\n" ) ; 
sprintf ( hilera_antes , "fancyhead[CE,CO]{textbf{%s - %s}}" , descripcion , materia_descripcion ) ; 
hilera_LATEX ( hilera_antes , hilera_despues ) ; 
fprintf ( Archivo_Latex , "%s\n" , hilera_despues ) ; 
fprintf ( Archivo_Latex , "fancyfoot{}\n" ) ; 
sprintf ( hilera_antes , "fancyfoot[CE,CO]{%s}" , institucion ) ; 
hilera_LATEX ( hilera_antes , hilera_despues ) ; 
fprintf ( Archivo_Latex , "%s\n" , hilera_despues ) ; 
sprintf ( hilera_antes , "fancyfoot[LE,LO]{textbf{%s}}" , escuela ) ; 
hilera_LATEX ( hilera_antes , hilera_despues ) ; 
fprintf ( Archivo_Latex , "%s\n" , hilera_despues ) ; 
sprintf \end{lstlisting}
\end{frame}
\begin{frame}[fragile]
\frametitle{C\'odigo Preprocesado (Sin Pretty Print)}
\begin{lstlisting}[style=CStyle]
( hilera_antes , "fancyfoot[RE,RO]{%s}" , programa ) ; 
hilera_LATEX ( hilera_antes , hilera_despues ) ; 
fprintf ( Archivo_Latex , "%s\n" , hilera_despues ) ; 
fprintf ( Archivo_Latex , "renewcommand{headrulewidth}{0.4pt}\n" ) ; 
fprintf ( Archivo_Latex , "renewcommand{footrulewidth}{0.4pt}\n" ) ; 
fprintf ( Archivo_Latex , "begin{document}\n" ) ; 
fprintf ( Archivo_Latex , "\n\n" ) ; 
sprintf ( hilera_antes , "title{An'{a}lisis de %s%s (%s)}" , descripcion , materia_descripcion , fecha ) ; 
hilera_LATEX ( hilera_antes , hilera_despues ) ; 
fprintf ( Archivo_Latex , "%s\n" , hilera_despues ) ; 
fprintf ( Archivo_Latex , "maketitle\n" ) ; 
gtk_progress_bar_set_fraction ( GTK_PROGRESS_BAR ( PB_analisis ) , 0.125 ) ; 
while ( gtk_events_pending ( ) ) gtk_main_iteration ( ) ; 
Tabla_Datos_Generales \end{lstlisting}
\end{frame}
\begin{frame}[fragile]
\frametitle{C\'odigo Preprocesado (Sin Pretty Print)}
\begin{lstlisting}[style=CStyle]
( Archivo_Latex , institucion , escuela , programa , materia_descripcion , profesor , descripcion , fecha , codigo , media_real , desviacion_real , alfa , Rpb , 0 ) ; 
gtk_progress_bar_set_fraction ( GTK_PROGRESS_BAR ( PB_analisis ) , 0.15 ) ; 
while ( gtk_events_pending ( ) ) gtk_main_iteration ( ) ; 
Analisis_General ( Archivo_Latex , alfa , Rpb , 0 ) ; 
gtk_progress_bar_set_fraction ( GTK_PROGRESS_BAR ( PB_analisis ) , 0.175 ) ; 
while ( gtk_events_pending ( ) ) gtk_main_iteration ( ) ; 
width = media_real - Nota_minima ; 
if ( ( Nota_maxima - media_real ) > width ) width = Nota_maxima - media_real ; 
Prepara_Histograma_Notas ( ) ; 
gtk_progress_bar_set_fraction ( GTK_PROGRESS_BAR ( PB_analisis ) , 0.22 ) ; 
while ( gtk_events_pending ( ) ) gtk_main_iteration ( ) ; 
Prepara_Grafico_Normal ( media_real , desviacion_real , media_prediccion , desviacion_prediccion , width ) ; 
gtk_progress_bar_set_fraction ( GTK_PROGRESS_BAR ( PB_analisis ) , 0.25 ) ; 
while \end{lstlisting}
\end{frame}
\begin{frame}[fragile]
\frametitle{C\'odigo Preprocesado (Sin Pretty Print)}
\begin{lstlisting}[style=CStyle]
( gtk_events_pending ( ) ) gtk_main_iteration ( ) ; 
fprintf ( Archivo_Latex , "begin{figure}[H]\n" ) ; 
fprintf ( Archivo_Latex , "centering\n" ) ; 
fprintf ( Archivo_Latex , "begin{minipage}[c]{0.49textwidth}\n" ) ; 
fprintf ( Archivo_Latex , "includegraphics[scale=0.8]{EX4010-h.pdf}\n" ) ; 
fprintf ( Archivo_Latex , "caption*{small {textbf{Histograma de Notas}}}\n" ) ; 
fprintf ( Archivo_Latex , "end{minipage}\n" ) ; 
fprintf ( Archivo_Latex , "begin{minipage}[c]{0.49textwidth}\n" ) ; 
fprintf ( Archivo_Latex , "includegraphics[scale=0.8]{EX4010-n.pdf}\n" ) ; 
fprintf ( Archivo_Latex , "caption*{small {textbf{Distribuci'{o}n Normal (real y predicci'{o}n)}}}\n" ) ; 
fprintf ( Archivo_Latex , "end{minipage}\n" ) ; 
fprintf ( Archivo_Latex , "end{figure}\n" ) ; 
fprintf ( Archivo_Latex , "clearpage\n" ) ; 
Dificultad_vs_Discriminacion \end{lstlisting}
\end{frame}
\begin{frame}[fragile]
\frametitle{C\'odigo Preprocesado (Sin Pretty Print)}
\begin{lstlisting}[style=CStyle]
( ) ; 
gtk_progress_bar_set_fraction ( GTK_PROGRESS_BAR ( PB_analisis ) , 0.35 ) ; 
while ( gtk_events_pending ( ) ) gtk_main_iteration ( ) ; 
fprintf ( Archivo_Latex , "begin{figure}[H]\n" ) ; 
fprintf ( Archivo_Latex , "centering\n" ) ; 
fprintf ( Archivo_Latex , "includegraphics[scale=1.0]{EX4010p.pdf}\n" ) ; 
hilera_LATEX ( "caption*{textbf{Cruce entre Coeficiente de Discriminación ($r_{pb}$) y Dificultad ($p$)}}" , hilera_despues ) ; 
fprintf ( Archivo_Latex , "%s\n\n" , hilera_despues ) ; 
fprintf ( Archivo_Latex , "end{figure}\n\n" ) ; 
fprintf ( Archivo_Latex , "begin{figure}[H]\n" ) ; 
fprintf ( Archivo_Latex , "centering\n" ) ; 
fprintf ( Archivo_Latex , "includegraphics[scale=1.0]{EX4010c.pdf}\n" ) ; 
hilera_LATEX ( "caption*{textbf{Líneas de Contorno del Cruce entre Coeficiente de Discriminación ($r_{pb}$) y Dificultad ($p$)}}" , hilera_despues ) ; 
fprintf \end{lstlisting}
\end{frame}
\begin{frame}[fragile]
\frametitle{C\'odigo Preprocesado (Sin Pretty Print)}
\begin{lstlisting}[style=CStyle]
( Archivo_Latex , "%s\n\n" , hilera_despues ) ; 
fprintf ( Archivo_Latex , "end{figure}\n\n" ) ; 
fprintf ( Archivo_Latex , "clearpage\n" ) ; 
if ( N_temas > 1 ) Prepara_Grafico_Pastel ( Archivo_Latex ) ; 
Prepara_Histograma_Temas ( ) ; 
fprintf ( Archivo_Latex , "begin{figure}[H]\n" ) ; 
fprintf ( Archivo_Latex , "centering\n" ) ; 
fprintf ( Archivo_Latex , "includegraphics[scale=1.0, angle=-90]{EX4010-t.pdf}\n" ) ; 
fprintf ( Archivo_Latex , "caption*{small {textbf{An'{a}lisis por Temas (ordenado por rendimiento)}}}\n" ) ; 
fprintf ( Archivo_Latex , "end{figure}\n" ) ; 
gtk_progress_bar_set_fraction ( GTK_PROGRESS_BAR ( PB_analisis ) , 0.4 ) ; 
while ( gtk_events_pending ( ) ) gtk_main_iteration ( ) ; 
Prepara_Histograma_Subtemas ( ) ; 
fprintf \end{lstlisting}
\end{frame}
\begin{frame}[fragile]
\frametitle{C\'odigo Preprocesado (Sin Pretty Print)}
\begin{lstlisting}[style=CStyle]
( Archivo_Latex , "begin{figure}[H]\n" ) ; 
fprintf ( Archivo_Latex , "centering\n" ) ; 
fprintf ( Archivo_Latex , "includegraphics[scale=1.0, angle=-90]{EX4010-s.pdf}\n" ) ; 
fprintf ( Archivo_Latex , "caption*{small {textbf{An'{a}lisis por Subtemas (ordenado por rendimiento)}}}\n" ) ; 
fprintf ( Archivo_Latex , "end{figure}\n" ) ; 
gtk_progress_bar_set_fraction ( GTK_PROGRESS_BAR ( PB_analisis ) , 0.45 ) ; 
while ( gtk_events_pending ( ) ) gtk_main_iteration ( ) ; 
Asigna_Banderas ( ) ; 
gtk_progress_bar_set_fraction ( GTK_PROGRESS_BAR ( PB_analisis ) , 0.5 ) ; 
while ( gtk_events_pending ( ) ) gtk_main_iteration ( ) ; 
fprintf ( Archivo_Latex , "twocolumn\n" ) ; 
fprintf ( Archivo_Latex , "clearpage\n" ) ; 
fprintf ( Archivo_Latex , "SetWatermarkText{Confidencial}\n" ) ; 
Lista_de_Preguntas \end{lstlisting}
\end{frame}
\begin{frame}[fragile]
\frametitle{C\'odigo Preprocesado (Sin Pretty Print)}
\begin{lstlisting}[style=CStyle]
( Archivo_Latex , PB_analisis , 0.5 , 0.29 ) ; 
gtk_progress_bar_set_fraction ( GTK_PROGRESS_BAR ( PB_analisis ) , 0.8 ) ; 
while ( gtk_events_pending ( ) ) gtk_main_iteration ( ) ; 
fprintf ( Archivo_Latex , "onecolumn\n" ) ; 
fprintf ( Archivo_Latex , "clearpage\n" ) ; 
fprintf ( Archivo_Latex , "SetWatermarkText{}\n" ) ; 
Resumen_de_Banderas ( Archivo_Latex , 0 ) ; 
gtk_progress_bar_set_fraction ( GTK_PROGRESS_BAR ( PB_analisis ) , 0.82 ) ; 
while ( gtk_events_pending ( ) ) gtk_main_iteration ( ) ; 
fprintf ( Archivo_Latex , "clearpage\n" ) ; 
fprintf ( Archivo_Latex , "SetWatermarkText{Confidencial}\n" ) ; 
Lista_de_Notas ( Archivo_Latex ) ; 
fprintf ( Archivo_Latex , "end{document}\n" ) ; 
fclose \end{lstlisting}
\end{frame}
\begin{frame}[fragile]
\frametitle{C\'odigo Preprocesado (Sin Pretty Print)}
\begin{lstlisting}[style=CStyle]
( Archivo_Latex ) ; 
gtk_progress_bar_set_fraction ( GTK_PROGRESS_BAR ( PB_analisis ) , 0.85 ) ; 
while ( gtk_events_pending ( ) ) gtk_main_iteration ( ) ; 
latex_2_pdf ( & parametros , Directorio , parametros . ruta_latex , "analisis" , 1 , PB_analisis , 0.85 , 0.1 , NULL , NULL ) ; 
gtk_progress_bar_set_fraction ( GTK_PROGRESS_BAR ( PB_analisis ) , 0.95 ) ; 
while ( gtk_events_pending ( ) ) gtk_main_iteration ( ) ; 
system ( "rm EX4010-h.pdf" ) ; 
system ( "rm EX4010-n.pdf" ) ; 
system ( "rm EX4010-t.pdf" ) ; 
system ( "rm EX4010-s.pdf" ) ; 
system ( "rm EX4010p.pdf" ) ; 
system ( "rm EX4010c.pdf" ) ; 
system ( "rm EX4010.dat" ) ; 
g_free \end{lstlisting}
\end{frame}
\begin{frame}[fragile]
\frametitle{C\'odigo Preprocesado (Sin Pretty Print)}
\begin{lstlisting}[style=CStyle]
( materia ) ; 
g_free ( materia_descripcion ) ; 
g_free ( descripcion ) ; 
g_free ( institucion ) ; 
g_free ( escuela ) ; 
g_free ( programa ) ; 
g_free ( fecha ) ; 
gtk_progress_bar_set_fraction ( GTK_PROGRESS_BAR ( PB_analisis ) , 1.0 ) ; 
while ( gtk_events_pending ( ) ) gtk_main_iteration ( ) ; 
gtk_widget_hide ( window4 ) ; 
gtk_widget_set_sensitive ( window1 , 1 ) ; 
} 
void Prepara_Grafico_Pastel ( FILE * Archivo_Latex ) 
{ \end{lstlisting}
\end{frame}
\begin{frame}[fragile]
\frametitle{C\'odigo Preprocesado (Sin Pretty Print)}
\begin{lstlisting}[style=CStyle]

int Empates ; 
Crea_archivo_datos_pastel ( & Empates ) ; 
fprintf ( Archivo_Latex , "DTLloaddb{datosesquema}{EX4010.dat}\n" ) ; 
fprintf ( Archivo_Latex , "begin{figure}[htpb]\n" ) ; 
fprintf ( Archivo_Latex , "centering\n" ) ; 
fprintf ( Archivo_Latex , "setlength{DTLpieoutlinewidth}{1pt}\n" ) ; 
colores_pastel ( Archivo_Latex ) ; 
fprintf ( Archivo_Latex , "renewcommand*{DTLdisplayinnerlabel}[1]{textsf{#1}}\n" ) ; 
fprintf ( Archivo_Latex , "renewcommand*{DTLdisplayouterlabel}[1]{textsf{#1}}\n" ) ; 
fprintf ( Archivo_Latex , "DTLpiechart{variable=quantity,%%\n" ) ; 
fprintf ( Archivo_Latex , "radius=4.1cm,%%\n" ) ; 
fprintf ( Archivo_Latex , "outerratio=1.05,%%\n" ) ; 
if \end{lstlisting}
\end{frame}
\begin{frame}[fragile]
\frametitle{C\'odigo Preprocesado (Sin Pretty Print)}
\begin{lstlisting}[style=CStyle]
( Empates <  18 ) 
if ( Empates == 1 ) 
fprintf ( Archivo_Latex , "cutaway={1},%%\n" ) ; 
else 
fprintf ( Archivo_Latex , "cutaway={1-%d},%%\n" , Empates ) ; 
fprintf ( Archivo_Latex , "innerlabel={},%%\n" ) ; 
fprintf ( Archivo_Latex , "outerlabel={DTLpievariable}}{datosesquema}{%%\n" ) ; 
fprintf ( Archivo_Latex , "name=Name,quantity=Quantity}\n" ) ; 
fprintf ( Archivo_Latex , "begin{tabular}[b]{ll}\n" ) ; 
fprintf ( Archivo_Latex , "DTLforeach{datosesquema}{name=Name,quantity=Quantity}{DTLiffirstrow{}{}%%\n" ) ; 
fprintf ( Archivo_Latex , "DTLdocurrentpiesegmentcolorrule{10pt}{10pt}&\n" ) ; 
fprintf ( Archivo_Latex , "textsf{name}\n" ) ; 
fprintf ( Archivo_Latex , "}\n" ) ; 
fprintf \end{lstlisting}
\end{frame}
\begin{frame}[fragile]
\frametitle{C\'odigo Preprocesado (Sin Pretty Print)}
\begin{lstlisting}[style=CStyle]
( Archivo_Latex , "end{tabular}\n" ) ; 
fprintf ( Archivo_Latex , "caption*{small {textbf{Material Evaluado en el Examen}}}\n" ) ; 
fprintf ( Archivo_Latex , "end{figure}\n" ) ; 
} 
void Prepara_Grafico_Pastel_Beamer ( FILE * Archivo_Latex ) 
{ 
int Empates ; 
int aspecto ; 
aspecto = gtk_combo_box_get_active ( CB_aspecto ) ; 
Crea_archivo_datos_pastel ( & Empates ) ; 
fprintf ( Archivo_Latex , "DTLloaddb{datosesquema}{EX4010.dat}\n" ) ; 
fprintf ( Archivo_Latex , "begin{columns}\n" ) ; 
fprintf ( Archivo_Latex , "begin{column}{0.5textwidth}\n" ) ; 
fprintf \end{lstlisting}
\end{frame}
\begin{frame}[fragile]
\frametitle{C\'odigo Preprocesado (Sin Pretty Print)}
\begin{lstlisting}[style=CStyle]
( Archivo_Latex , "begin{figure}[H]\n" ) ; 
fprintf ( Archivo_Latex , "centering\n" ) ; 
fprintf ( Archivo_Latex , "setlength{DTLpieoutlinewidth}{1pt}\n" ) ; 
colores_pastel ( Archivo_Latex ) ; 
fprintf ( Archivo_Latex , "renewcommand*{DTLdisplayinnerlabel}[1]{textsf{#1}}\n" ) ; 
fprintf ( Archivo_Latex , "renewcommand*{DTLdisplayouterlabel}[1]{textsf{#1}}\n" ) ; 
fprintf ( Archivo_Latex , "DTLpiechart{variable=quantity,%%\n" ) ; 
if ( ( aspecto == 1 ) || ( aspecto == 2 ) ) 
fprintf ( Archivo_Latex , "radius=3.0cm,%%\n" ) ; 
else 
fprintf ( Archivo_Latex , "radius=2.25cm,%%\n" ) ; 
fprintf ( Archivo_Latex , "outerratio=1.05,%%\n" ) ; 
if ( Empates <  18 ) 
if \end{lstlisting}
\end{frame}
\begin{frame}[fragile]
\frametitle{C\'odigo Preprocesado (Sin Pretty Print)}
\begin{lstlisting}[style=CStyle]
( Empates == 1 ) 
fprintf ( Archivo_Latex , "cutaway={1},%%\n" ) ; 
else 
fprintf ( Archivo_Latex , "cutaway={1-%d},%%\n" , Empates ) ; 
fprintf ( Archivo_Latex , "innerlabel={},%%\n" ) ; 
fprintf ( Archivo_Latex , "outerlabel={DTLpievariable}}{datosesquema}{%%\n" ) ; 
fprintf ( Archivo_Latex , "name=Name,quantity=Quantity}\n" ) ; 
fprintf ( Archivo_Latex , "end{figure}\n" ) ; 
fprintf ( Archivo_Latex , "end{column}\n" ) ; 
fprintf ( Archivo_Latex , "begin{column}{0.5textwidth}\n" ) ; 
fprintf ( Archivo_Latex , "begin{figure}[htpb]\n" ) ; 
fprintf ( Archivo_Latex , "centering\n" ) ; 
colores_pastel ( Archivo_Latex ) ; 
fprintf \end{lstlisting}
\end{frame}
\begin{frame}[fragile]
\frametitle{C\'odigo Preprocesado (Sin Pretty Print)}
\begin{lstlisting}[style=CStyle]
( Archivo_Latex , "begin{tabular}[b]{ll}\n" ) ; 
fprintf ( Archivo_Latex , "DTLforeach{datosesquema}{name=Name,quantity=Quantity}{DTLiffirstrow{}{}%%\n" ) ; 
fprintf ( Archivo_Latex , "DTLdocurrentpiesegmentcolorrule{10pt}{10pt}&\n" ) ; 
fprintf ( Archivo_Latex , "textsf{name}\n" ) ; 
fprintf ( Archivo_Latex , "}\n" ) ; 
fprintf ( Archivo_Latex , "end{tabular}\n" ) ; 
fprintf ( Archivo_Latex , "end{figure}\n" ) ; 
fprintf ( Archivo_Latex , "end{column}\n" ) ; 
fprintf ( Archivo_Latex , "end{columns}\n" ) ; 
} 
void Crea_archivo_datos_pastel ( int * Empates ) 
{ 
gchar * materia ; 
FILE \end{lstlisting}
\end{frame}
\begin{frame}[fragile]
\frametitle{C\'odigo Preprocesado (Sin Pretty Print)}
\begin{lstlisting}[style=CStyle]
* Archivo_Datos ; 
int i , j , N , N_otros ; 
char hilera_antes [ 2000 ] ; 
char hilera_despues [ 2000 ] ; 
struct { 
char tema [ CODIGO_TEMA_SIZE + 1 ] ; 
int cantidad ; 
} temporal , tabla [ N_temas ] ; 
char PG_command [ 2000 ] ; 
PGresult * res_tema ; 
char Descripcion [ 300 ] ; 
int MAX_DESCRIPCION ; 
int aspecto ; 
aspecto \end{lstlisting}
\end{frame}
\begin{frame}[fragile]
\frametitle{C\'odigo Preprocesado (Sin Pretty Print)}
\begin{lstlisting}[style=CStyle]
= gtk_combo_box_get_active ( CB_aspecto ) ; 
if ( ( aspecto == 1 ) || ( aspecto == 2 ) ) 
MAX_DESCRIPCION = 43 ; 
else 
MAX_DESCRIPCION = 35 ; 
materia = gtk_editable_get_chars ( GTK_EDITABLE ( EN_materia ) , 0 , - 1 ) ; 
Archivo_Datos = fopen ( "EX4010.dat" , "w" ) ; 
fprintf ( Archivo_Datos , "Name, Quantity\n" ) ; 
for ( i = 0 ; i < N_temas ; i ++ ) 
{ 
strcpy ( tabla [ i ] . tema , resumen_tema [ i ] . tema ) ; 
tabla [ i ] . cantidad = ( resumen_tema [ i ] . buenos + resumen_tema [ i ] . malos ) / N_estudiantes ; 
} 
for \end{lstlisting}
\end{frame}
\begin{frame}[fragile]
\frametitle{C\'odigo Preprocesado (Sin Pretty Print)}
\begin{lstlisting}[style=CStyle]
( i = N_temas ; i >= 0 ; i -- ) 
for ( j = 0 ; j < ( i - 1 ) ; j ++ ) 
{ 
if ( tabla [ j ] . cantidad < tabla [ j + 1 ] . cantidad ) 
{ 
temporal = tabla [ j + 1 ] ; 
tabla [ j + 1 ] = tabla [ j ] ; 
tabla [ j ] = temporal ; 
} 
} 
N = N_temas ; 
if ( N >  18 ) N =  18 - 1 ; 
for ( i = 0 ; i < N ; i ++ ) 
{ \end{lstlisting}
\end{frame}
\begin{frame}[fragile]
\frametitle{C\'odigo Preprocesado (Sin Pretty Print)}
\begin{lstlisting}[style=CStyle]

sprintf ( PG_command , "SELECT descripcion_materia from bd_materias where codigo_materia = '%s' and codigo_tema = '%s' and codigo_subtema = '          '" , 
materia , tabla [ i ] . tema ) ; 
res_tema = PQEXEC ( DATABASE , PG_command ) ; 
strcpy ( Descripcion , PQgetvalue ( res_tema , 0 , 0 ) ) ; 
if ( strlen ( Descripcion ) > MAX_DESCRIPCION ) 
{ 
Descripcion [ MAX_DESCRIPCION - 3 ] = '.' ; 
Descripcion [ MAX_DESCRIPCION - 2 ] = '.' ; 
Descripcion [ MAX_DESCRIPCION - 1 ] = '.' ; 
Descripcion [ MAX_DESCRIPCION ] = '\0' ; 
} 
sprintf ( hilera_antes , "\"textbf{%s}\", %d" , Descripcion , tabla [ i ] . cantidad ) ; 
hilera_LATEX \end{lstlisting}
\end{frame}
\begin{frame}[fragile]
\frametitle{C\'odigo Preprocesado (Sin Pretty Print)}
\begin{lstlisting}[style=CStyle]
( hilera_antes , hilera_despues ) ; 
fprintf ( Archivo_Datos , "%s\n" , hilera_despues ) ; 
} 
N_otros = 0 ; 
for ( i = N ; i < N_temas ; i ++ ) N_otros += tabla [ i ] . cantidad ; 
if ( N_otros ) 
fprintf ( Archivo_Datos , "\"textbf{Otros} (%d temas)\", %d\n" , N_temas - N , N_otros ) ; 
fclose ( Archivo_Datos ) ; 
i = 1 ; 
while ( ( i < N_temas ) && tabla [ 0 ] . cantidad == tabla [ i ++ ] . cantidad ) ; 
* Empates = i - 1 ; 
g_free ( materia ) ; 
} 
void \end{lstlisting}
\end{frame}
\begin{frame}[fragile]
\frametitle{C\'odigo Preprocesado (Sin Pretty Print)}
\begin{lstlisting}[style=CStyle]
colores_pastel ( FILE * Archivo_Latex ) 
{ 
fprintf ( Archivo_Latex , "DTLsetpiesegmentcolor{1}{green}\n" ) ; 
fprintf ( Archivo_Latex , "DTLsetpiesegmentcolor{2}{blue}\n" ) ; 
fprintf ( Archivo_Latex , "DTLsetpiesegmentcolor{3}{red}\n" ) ; 
fprintf ( Archivo_Latex , "DTLsetpiesegmentcolor{4}{yellow}\n" ) ; 
fprintf ( Archivo_Latex , "DTLsetpiesegmentcolor{5}{magenta}\n" ) ; 
fprintf ( Archivo_Latex , "DTLsetpiesegmentcolor{6}{cyan}\n" ) ; 
fprintf ( Archivo_Latex , "DTLsetpiesegmentcolor{7}{orange}\n" ) ; 
fprintf ( Archivo_Latex , "DTLsetpiesegmentcolor{8}{violet}\n" ) ; 
fprintf ( Archivo_Latex , "DTLsetpiesegmentcolor{9}{gray}\n" ) ; 
fprintf ( Archivo_Latex , "DTLsetpiesegmentcolor{10}{lime}\n" ) ; 
fprintf ( Archivo_Latex , "DTLsetpiesegmentcolor{11}{teal}\n" ) ; 
fprintf \end{lstlisting}
\end{frame}
\begin{frame}[fragile]
\frametitle{C\'odigo Preprocesado (Sin Pretty Print)}
\begin{lstlisting}[style=CStyle]
( Archivo_Latex , "DTLsetpiesegmentcolor{12}{pink}\n" ) ; 
fprintf ( Archivo_Latex , "DTLsetpiesegmentcolor{13}{brown}\n" ) ; 
fprintf ( Archivo_Latex , "DTLsetpiesegmentcolor{14}{purple}\n" ) ; 
fprintf ( Archivo_Latex , "DTLsetpiesegmentcolor{15}{darkgray}\n" ) ; 
fprintf ( Archivo_Latex , "DTLsetpiesegmentcolor{16}{olive}\n" ) ; 
fprintf ( Archivo_Latex , "DTLsetpiesegmentcolor{17}{black}\n" ) ; 
fprintf ( Archivo_Latex , "DTLsetpiesegmentcolor{18}{lightgray}\n" ) ; 
} 
void Establece_Directorio ( char * Directorio , gchar * materia , char * year , char * month , char * day ) 
{ 
char ruta_final [ 200 ] ; 
DIR * pdir ; 
char Directorio_materia [ 600 ] ; 
char \end{lstlisting}
\end{frame}
\begin{frame}[fragile]
\frametitle{C\'odigo Preprocesado (Sin Pretty Print)}
\begin{lstlisting}[style=CStyle]
Directorio_materia_fecha [ 600 ] ; 
int i , n ; 
sprintf ( Directorio_materia , "%s/%s" , parametros . ruta_examenes , materia ) ; 
n = strlen ( Directorio_materia ) ; 
for ( i = n - 1 ; Directorio_materia [ i ] == ' ' ; i -- ) ; 
Directorio_materia [ i + 1 ] = '\0' ; 
sprintf ( Directorio , "%s/%s-%.2s-%.2s" , Directorio_materia , year , month , day ) ; 
pdir = opendir ( Directorio_materia ) ; 
if ( ! pdir ) 
{ 
mkdir ( Directorio_materia , S_IRWXU | S_IRWXG | S_IROTH | S_IXOTH ) ; 
} 
else 
closedir \end{lstlisting}
\end{frame}
\begin{frame}[fragile]
\frametitle{C\'odigo Preprocesado (Sin Pretty Print)}
\begin{lstlisting}[style=CStyle]
( pdir ) ; 
pdir = opendir ( Directorio ) ; 
if ( ! pdir ) 
{ 
mkdir ( Directorio , S_IRWXU | S_IRWXG | S_IROTH | S_IXOTH ) ; 
} 
else 
closedir ( pdir ) ; 
} 
void Quita_espacios ( char * hilera ) 
{ 
int i , j ; 
i = j = 0 ; 
while \end{lstlisting}
\end{frame}
\begin{frame}[fragile]
\frametitle{C\'odigo Preprocesado (Sin Pretty Print)}
\begin{lstlisting}[style=CStyle]
( hilera [ i ] ) 
{ 
if ( hilera [ i ] != ' ' ) hilera [ j ++ ] = hilera [ i ] ; 
i ++ ; 
} 
hilera [ j ] = '\0' ; 
} 
void Genera_Beamer ( ) 
{ 
int k ; 
char codigo [ 10 ] ; 
gchar * size , * font , * color , * estilo , * materia , * materia_descripcion , * descripcion , * version , * institucion , * escuela , * programa , * fecha , * profesor ; 
long double media_real , desviacion_real , media_prediccion , desviacion_prediccion , alfa , Rpb , width ; 
FILE \end{lstlisting}
\end{frame}
\begin{frame}[fragile]
\frametitle{C\'odigo Preprocesado (Sin Pretty Print)}
\begin{lstlisting}[style=CStyle]
* Archivo_Latex ; 
char Directorio [ 3000 ] ; 
char hilera_antes [ 3000 ] , hilera_despues [ 3000 ] ; 
int aspecto ; 
int resultado_OK ; 
gtk_widget_set_sensitive ( window1 , 0 ) ; 
gtk_widget_show ( window3 ) ; 
Update_PB ( PB_beamer , 0.0 ) ; 
Banderas [ 0 ] = Banderas [ 1 ] = Banderas [ 2 ] = Banderas [ 3 ] = Banderas [ 4 ] = Banderas [ 5 ] = 0 ; 
k = ( int ) gtk_spin_button_get_value_as_int ( SP_examen ) ; 
sprintf ( codigo , "%05d" , k ) ; 
materia = gtk_editable_get_chars ( GTK_EDITABLE ( EN_materia ) , 0 , - 1 ) ; 
materia_descripcion = gtk_editable_get_chars ( GTK_EDITABLE ( EN_materia_descripcion ) , 0 , - 1 ) ; 
descripcion \end{lstlisting}
\end{frame}
\begin{frame}[fragile]
\frametitle{C\'odigo Preprocesado (Sin Pretty Print)}
\begin{lstlisting}[style=CStyle]
= gtk_editable_get_chars ( GTK_EDITABLE ( EN_descripcion ) , 0 , - 1 ) ; 
institucion = gtk_editable_get_chars ( GTK_EDITABLE ( EN_institucion ) , 0 , - 1 ) ; 
escuela = gtk_editable_get_chars ( GTK_EDITABLE ( EN_escuela ) , 0 , - 1 ) ; 
programa = gtk_editable_get_chars ( GTK_EDITABLE ( EN_programa ) , 0 , - 1 ) ; 
profesor = gtk_editable_get_chars ( GTK_EDITABLE ( EN_profesor ) , 0 , - 1 ) ; 
fecha = gtk_editable_get_chars ( GTK_EDITABLE ( EN_fecha ) , 0 , - 1 ) ; 
media_real = atof ( gtk_editable_get_chars ( GTK_EDITABLE ( EN_media_real ) , 0 , - 1 ) ) ; 
desviacion_real = atof ( gtk_editable_get_chars ( GTK_EDITABLE ( EN_desviacion_real ) , 0 , - 1 ) ) ; 
media_prediccion = atof ( gtk_editable_get_chars ( GTK_EDITABLE ( EN_media_prediccion ) , 0 , - 1 ) ) ; 
desviacion_prediccion = atof ( gtk_editable_get_chars ( GTK_EDITABLE ( EN_desviacion_prediccion ) , 0 , - 1 ) ) ; 
alfa = atof ( gtk_editable_get_chars ( GTK_EDITABLE ( EN_alfa_real ) , 0 , - 1 ) ) ; 
Rpb = atof ( gtk_editable_get_chars ( GTK_EDITABLE ( EN_Rpb_real ) , 0 , - 1 ) ) ; 
estilo = gtk_combo_box_get_active_text ( CB_estilo ) ; 
color \end{lstlisting}
\end{frame}
\begin{frame}[fragile]
\frametitle{C\'odigo Preprocesado (Sin Pretty Print)}
\begin{lstlisting}[style=CStyle]
= gtk_combo_box_get_active_text ( CB_color ) ; 
font = gtk_combo_box_get_active_text ( CB_font ) ; 
size = gtk_combo_box_get_active_text ( CB_size ) ; 
aspecto = gtk_combo_box_get_active ( CB_aspecto ) ; 
Establece_Directorio ( Directorio , materia , fecha + 6 , fecha + 3 , fecha ) ; 
Update_PB ( PB_beamer , 0.05 ) ; 
Archivo_Latex = fopen ( "analisis-beamer.tex" , "w" ) ; 
Beamer_Preamble ( Archivo_Latex , aspecto , size , estilo , color , font , materia_descripcion , descripcion , profesor , programa , escuela , institucion , fecha ) ; 
Update_PB ( PB_beamer , 0.08 ) ; 
Beamer_Cover ( Archivo_Latex ) ; 
Update_PB ( PB_beamer , 0.09 ) ; 
Beamer_TOC ( Archivo_Latex ) ; 
Update_PB ( PB_beamer , 0.10 ) ; 
Beamer_Datos_Generales \end{lstlisting}
\end{frame}
\begin{frame}[fragile]
\frametitle{C\'odigo Preprocesado (Sin Pretty Print)}
\begin{lstlisting}[style=CStyle]
( Archivo_Latex , institucion , escuela , programa , materia_descripcion , profesor , descripcion , fecha , codigo , media_real , desviacion_real , alfa , Rpb ) ; 
Update_PB ( PB_beamer , 0.12 ) ; 
if ( N_temas > 1 ) Beamer_Grafico_Pastel ( Archivo_Latex ) ; 
Update_PB ( PB_beamer , 0.15 ) ; 
Beamer_Histograma_Notas ( Archivo_Latex , media_real , desviacion_real , media_prediccion , desviacion_prediccion ) ; 
Update_PB ( PB_beamer , 0.17 ) ; 
Beamer_Dificultad_vs_Discriminacion ( Archivo_Latex ) ; 
Update_PB ( PB_beamer , 0.19 ) ; 
Beamer_Histograma_Temas ( Archivo_Latex ) ; 
Update_PB ( PB_beamer , 0.21 ) ; 
Beamer_Preguntas ( Archivo_Latex , PB_beamer , 0.21 , 0.33 ) ; 
Beamer_Gracias ( Archivo_Latex ) ; 
Beamer_Cierre ( Archivo_Latex ) ; 
fclose \end{lstlisting}
\end{frame}
\begin{frame}[fragile]
\frametitle{C\'odigo Preprocesado (Sin Pretty Print)}
\begin{lstlisting}[style=CStyle]
( Archivo_Latex ) ; 
Update_PB ( PB_beamer , 0.55 ) ; 
resultado_OK = latex_2_pdf ( & parametros , Directorio , parametros . ruta_latex , "analisis-beamer" , 1 , PB_beamer , 0.55 , 0.43 , NULL , NULL ) ; 
system ( "rm EX4010-h.pdf" ) ; 
system ( "rm EX4010-n.pdf" ) ; 
system ( "rm EX4010-t.pdf" ) ; 
system ( "rm EX4010p.pdf" ) ; 
system ( "rm EX4010c.pdf" ) ; 
system ( "rm EX4010.dat" ) ; 
system ( "rm analisis-beamer.*" ) ; 
Update_PB ( PB_beamer , 0.99 ) ; 
g_free ( estilo ) ; 
g_free ( color ) ; 
g_free \end{lstlisting}
\end{frame}
\begin{frame}[fragile]
\frametitle{C\'odigo Preprocesado (Sin Pretty Print)}
\begin{lstlisting}[style=CStyle]
( font ) ; 
g_free ( size ) ; 
g_free ( materia ) ; 
g_free ( materia_descripcion ) ; 
g_free ( descripcion ) ; 
g_free ( institucion ) ; 
g_free ( escuela ) ; 
g_free ( programa ) ; 
g_free ( fecha ) ; 
Update_PB ( PB_beamer , 1.0 ) ; 
gtk_widget_hide ( window3 ) ; 
gtk_widget_set_sensitive ( window1 , 1 ) ; 
if ( ! resultado_OK ) 
{ \end{lstlisting}
\end{frame}
\begin{frame}[fragile]
\frametitle{C\'odigo Preprocesado (Sin Pretty Print)}
\begin{lstlisting}[style=CStyle]

Beamer_Failure ( ) ; 
} 
} 
void Beamer_Preamble ( FILE * Archivo_Latex , int aspecto , gchar * size , gchar * estilo , gchar * color , gchar * font , gchar * materia_descripcion , gchar * descripcion , gchar * profesor , gchar * programa , gchar * escuela , gchar * institucion , gchar * fecha ) 
{ 
char hilera_antes [ 3000 ] , hilera_despues [ 3000 ] ; 
fprintf ( Archivo_Latex , "documentclass[aspectratio=%s,%s]{beamer}\n" , Beamer_aspectratio [ aspecto ] , size ) ; 
fprintf ( Archivo_Latex , "def BEAMER {}\n" ) ; 
EX_latex_packages ( Archivo_Latex ) ; 
fprintf ( Archivo_Latex , "\n%s\n\n" , parametros . Paquetes ) ; 
fprintf ( Archivo_Latex , "newcommand*circled[1]{tikz[baseline=(char.base)]{node[shape=circle,draw,inner sep=2pt] (char) {#1};}}\n" ) ; 
fprintf ( Archivo_Latex , "newcommand*correcta[1]{tikz[baseline=(char.base)]{node[shape=circle,fill=gray,draw,inner sep=2pt] (char) {#1};}}\n" ) ; 
fprintf \end{lstlisting}
\end{frame}
\begin{frame}[fragile]
\frametitle{C\'odigo Preprocesado (Sin Pretty Print)}
\begin{lstlisting}[style=CStyle]
( Archivo_Latex , "graphicspath{{%s/}}\n\n" , parametros . ruta_figuras ) ; 
fprintf ( Archivo_Latex , "usetheme{%s}\n" , estilo ) ; 
fprintf ( Archivo_Latex , "usecolortheme{%s}\n" , color ) ; 
fprintf ( Archivo_Latex , "usefonttheme{%s}\n" , font ) ; 
fprintf ( Archivo_Latex , "useoutertheme{shadow}\n" ) ; 
fprintf ( Archivo_Latex , "setbeamertemplate{navigation symbols}{}\n" ) ; 
fprintf ( Archivo_Latex , "setbeamertemplate{caption}[numbered]\n" ) ; 
fprintf ( Archivo_Latex , "captionsetup{labelsep = colon,figureposition = bottom}\n" ) ; 
fprintf ( Archivo_Latex , "setbeamercolor{caption name}{fg=black}\n" ) ; 
fprintf ( Archivo_Latex , "setbeamertemplate{headline}{}\n" ) ; 
fprintf ( Archivo_Latex , "definecolor{DoradoPalido}{RGB}{238,232,170}\n" ) ; 
fprintf ( Archivo_Latex , "setbeamercolor{pregunta} {bg=DoradoPalido, fg=black}\n" ) ; 
fprintf ( Archivo_Latex , "setbeamertemplate{blocks}[rounded][shadow=true]\n" ) ; 
fprintf \end{lstlisting}
\end{frame}
\begin{frame}[fragile]
\frametitle{C\'odigo Preprocesado (Sin Pretty Print)}
\begin{lstlisting}[style=CStyle]
( Archivo_Latex , "SetWatermarkText{}\n" ) ; 
sprintf ( hilera_antes , "title{%s}" , materia_descripcion ) ; 
hilera_LATEX ( hilera_antes , hilera_despues ) ; 
fprintf ( Archivo_Latex , "%s\n" , hilera_despues ) ; 
sprintf ( hilera_antes , "subtitle{Análisis de %s}" , descripcion ) ; 
hilera_LATEX ( hilera_antes , hilera_despues ) ; 
fprintf ( Archivo_Latex , "%s\n" , hilera_despues ) ; 
sprintf ( hilera_antes , "author[%s]{Prof. %s}" , profesor , profesor ) ; 
hilera_LATEX ( hilera_antes , hilera_despues ) ; 
fprintf ( Archivo_Latex , "%s\n" , hilera_despues ) ; 
fprintf ( Archivo_Latex , "institute{\n" ) ; 
sprintf ( hilera_antes , "%s" , programa ) ; 
hilera_LATEX ( hilera_antes , hilera_despues ) ; 
fprintf \end{lstlisting}
\end{frame}
\begin{frame}[fragile]
\frametitle{C\'odigo Preprocesado (Sin Pretty Print)}
\begin{lstlisting}[style=CStyle]
( Archivo_Latex , "%s\n" , hilera_despues ) ; 
sprintf ( hilera_antes , "%s" , escuela ) ; 
hilera_LATEX ( hilera_antes , hilera_despues ) ; 
fprintf ( Archivo_Latex , "%s\n" , hilera_despues ) ; 
sprintf ( hilera_antes , "%s" , institucion ) ; 
hilera_LATEX ( hilera_antes , hilera_despues ) ; 
fprintf ( Archivo_Latex , "%s\n" , hilera_despues ) ; 
fprintf ( Archivo_Latex , "}\n" ) ; 
fprintf ( Archivo_Latex , "date{%s}\n" , fecha ) ; 
fprintf ( Archivo_Latex , "begin{document}\n\n" ) ; 
} 
void Beamer_Cover ( FILE * Archivo_Latex ) 
{ 
fprintf \end{lstlisting}
\end{frame}
\begin{frame}[fragile]
\frametitle{C\'odigo Preprocesado (Sin Pretty Print)}
\begin{lstlisting}[style=CStyle]
( Archivo_Latex , "begin{frame}[plain]\n" ) ; 
fprintf ( Archivo_Latex , "titlepage\n" ) ; 
fprintf ( Archivo_Latex , "begin{figure}\n" ) ; 
fprintf ( Archivo_Latex , "includegraphics[height=1.3cm]{.imagenes/EX.png}\n" ) ; 
fprintf ( Archivo_Latex , "end{figure}\n" ) ; 
fprintf ( Archivo_Latex , "end{frame}\n" ) ; 
} 
void Beamer_TOC ( FILE * Archivo_Latex ) 
{ 
fprintf ( Archivo_Latex , "begin{frame}[plain]{Contenido}\n" ) ; 
fprintf ( Archivo_Latex , "tableofcontents\n" ) ; 
fprintf ( Archivo_Latex , "end{frame}\n" ) ; 
} 
void \end{lstlisting}
\end{frame}
\begin{frame}[fragile]
\frametitle{C\'odigo Preprocesado (Sin Pretty Print)}
\begin{lstlisting}[style=CStyle]
Beamer_Datos_Generales ( FILE * Archivo_Latex , gchar * institucion , gchar * escuela , gchar * programa , gchar * materia_descripcion , gchar * profesor , gchar * descripcion , gchar * fecha , char * codigo , long double media_real , long double desviacion_real , long double alfa , long double Rpb ) 
{ 
fprintf ( Archivo_Latex , "section{Estad'isticas B'asicas}\n" ) ; 
fprintf ( Archivo_Latex , "begin{frame}{Datos Generales}\n" ) ; 
Tabla_Datos_Generales ( Archivo_Latex , institucion , escuela , programa , materia_descripcion , profesor , descripcion , fecha , codigo , media_real , desviacion_real , alfa , Rpb , 1 ) ; 
fprintf ( Archivo_Latex , "end{frame}\n" ) ; 
fprintf ( Archivo_Latex , "begin{frame}{An'alisis General}\n" ) ; 
Analisis_General ( Archivo_Latex , alfa , Rpb , 1 ) ; 
fprintf ( Archivo_Latex , "end{frame}\n" ) ; 
} 
void Beamer_Grafico_Pastel ( FILE * Archivo_Latex ) 
{ 
fprintf ( Archivo_Latex , "begin{frame}[shrink,fragile]{Material Evaluado en el Examen}\n" ) ; 
Prepara_Grafico_Pastel_Beamer \end{lstlisting}
\end{frame}
\begin{frame}[fragile]
\frametitle{C\'odigo Preprocesado (Sin Pretty Print)}
\begin{lstlisting}[style=CStyle]
( Archivo_Latex ) ; 
fprintf ( Archivo_Latex , "end{frame}\n" ) ; 
} 
void Beamer_Histograma_Notas ( FILE * Archivo_Latex , long double media_real , long double desviacion_real , long double media_prediccion , long double desviacion_prediccion ) 
{ 
long double width ; 
width = media_real - Nota_minima ; 
if ( ( Nota_maxima - media_real ) > width ) width = Nota_maxima - media_real ; 
Prepara_Histograma_Notas ( ) ; 
Prepara_Grafico_Normal ( media_real , desviacion_real , media_prediccion , desviacion_prediccion , width ) ; 
fprintf ( Archivo_Latex , "begin{frame}{Histograma de Notas}\n" ) ; 
fprintf ( Archivo_Latex , "begin{figure}[H]\n" ) ; 
fprintf ( Archivo_Latex , "centering\n" ) ; 
fprintf \end{lstlisting}
\end{frame}
\begin{frame}[fragile]
\frametitle{C\'odigo Preprocesado (Sin Pretty Print)}
\begin{lstlisting}[style=CStyle]
( Archivo_Latex , "includegraphics[scale=0.8]{EX4010-h.pdf}\n" ) ; 
fprintf ( Archivo_Latex , "end{figure}\n" ) ; 
fprintf ( Archivo_Latex , "end{frame}\n" ) ; 
fprintf ( Archivo_Latex , "begin{frame}{Distribuci'{o}n Normal (Predicci'on y Real)}\n" ) ; 
fprintf ( Archivo_Latex , "begin{figure}[H]\n" ) ; 
fprintf ( Archivo_Latex , "centering\n" ) ; 
fprintf ( Archivo_Latex , "includegraphics[scale=0.8]{EX4010-n.pdf}\n" ) ; 
fprintf ( Archivo_Latex , "end{figure}\n" ) ; 
fprintf ( Archivo_Latex , "end{frame}\n" ) ; 
} 
void Beamer_Dificultad_vs_Discriminacion ( FILE * Archivo_Latex ) 
{ 
Dificultad_vs_Discriminacion ( ) ; 
fprintf \end{lstlisting}
\end{frame}
\begin{frame}[fragile]
\frametitle{C\'odigo Preprocesado (Sin Pretty Print)}
\begin{lstlisting}[style=CStyle]
( Archivo_Latex , "begin{frame}{Cruce de Discriminaci'on ($r_{pb}$) y Dificultad ($p$)}" ) ; 
fprintf ( Archivo_Latex , "begin{figure}[H]\n" ) ; 
fprintf ( Archivo_Latex , "includegraphics[scale=0.65]{EX4010p.pdf}\n" ) ; 
fprintf ( Archivo_Latex , "end{figure}\n\n" ) ; 
fprintf ( Archivo_Latex , "end{frame}\n" ) ; 
fprintf ( Archivo_Latex , "begin{frame}{L'ineas de Contorno del Cruce entre Coeficiente de Discriminaci'on ($r_{pb}$) y Dificultad ($p$)}" ) ; 
fprintf ( Archivo_Latex , "begin{figure}[H]\n" ) ; 
fprintf ( Archivo_Latex , "includegraphics[scale=0.65]{EX4010c.pdf}\n" ) ; 
fprintf ( Archivo_Latex , "end{figure}\n\n" ) ; 
fprintf ( Archivo_Latex , "end{frame}\n" ) ; 
} 
void Beamer_Histograma_Temas ( FILE * Archivo_Latex ) 
{ 
Prepara_Histograma_Temas \end{lstlisting}
\end{frame}
\begin{frame}[fragile]
\frametitle{C\'odigo Preprocesado (Sin Pretty Print)}
\begin{lstlisting}[style=CStyle]
( ) ; 
fprintf ( Archivo_Latex , "begin{frame}{An'{a}lisis por Temas (ordenado por rendimiento)}\n" ) ; 
fprintf ( Archivo_Latex , "begin{figure}[H]\n" ) ; 
fprintf ( Archivo_Latex , "includegraphics[scale=0.67, angle=-90]{EX4010-t.pdf}\n" ) ; 
fprintf ( Archivo_Latex , "end{figure}\n" ) ; 
fprintf ( Archivo_Latex , "end{frame}\n" ) ; 
} 
void Beamer_Preguntas ( FILE * Archivo_Latex , GtkWidget * PB , long double base , long double limite ) 
{ 
if ( ! gtk_toggle_button_get_active ( CK_general ) ) 
{ 
Asigna_Banderas ( ) ; 
Lista_de_Preguntas_Beamer ( Archivo_Latex , PB , base , limite ) ; 
} \end{lstlisting}
\end{frame}
\begin{frame}[fragile]
\frametitle{C\'odigo Preprocesado (Sin Pretty Print)}
\begin{lstlisting}[style=CStyle]

} 
void Beamer_Gracias ( FILE * Archivo_Latex ) 
{ 
fprintf ( Archivo_Latex , "begin{frame}[plain]\n" ) ; 
fprintf ( Archivo_Latex , "centering\n" ) ; 
fprintf ( Archivo_Latex , "{Huge Muchas gracias}\n" ) ; 
fprintf ( Archivo_Latex , "end{frame}\n" ) ; 
} 
void Beamer_Cierre ( FILE * Archivo_Latex ) 
{ 
fprintf ( Archivo_Latex , "end{document}\n" ) ; 
} 
void \end{lstlisting}
\end{frame}
\begin{frame}[fragile]
\frametitle{C\'odigo Preprocesado (Sin Pretty Print)}
\begin{lstlisting}[style=CStyle]
Beamer_Failure ( ) 
{ 
gtk_widget_set_sensitive ( window1 , 0 ) ; 
gtk_widget_show ( window5 ) ; 
} 
void Graba_Ajustes ( ) 
{ 
int i , k ; 
char PG_command [ 3000 ] ; 
PGresult * res ; 
char examen [ 10 ] ; 
k = ( int ) gtk_spin_button_get_value_as_int ( SP_examen ) ; 
sprintf ( examen , "%05d" , k ) ; 
res \end{lstlisting}
\end{frame}
\begin{frame}[fragile]
\frametitle{C\'odigo Preprocesado (Sin Pretty Print)}
\begin{lstlisting}[style=CStyle]
= PQEXEC ( DATABASE , "BEGIN" ) ; PQclear ( res ) ; 
sprintf ( PG_command , "DELETE from EX_examenes_ajustes where examen = '%s'" , examen ) ; 
res = PQEXEC ( DATABASE , PG_command ) ; PQclear ( res ) ; 
for ( i = 0 ; i < N_preguntas ; i ++ ) 
{ 
Calcula_ajuste ( i ) ; 
if ( preguntas [ i ] . ajuste !=  0 ) 
{ 
sprintf ( PG_command , "INSERT into EX_examenes_ajustes values (\'%s\', %d, \'%s\', %d, %d, %d, %d, %d, %d, %d)" , 
examen , i + 1 , preguntas [ i ] . pregunta , preguntas [ i ] . ajuste , 
preguntas [ i ] . correctas_nuevas [ 0 ] , preguntas [ i ] . correctas_nuevas [ 1 ] , preguntas [ i ] . correctas_nuevas [ 2 ] , 
preguntas [ i ] . correctas_nuevas [ 3 ] , preguntas [ i ] . correctas_nuevas [ 4 ] , 
preguntas [ i ] . actualizar ) ; 
res \end{lstlisting}
\end{frame}
\begin{frame}[fragile]
\frametitle{C\'odigo Preprocesado (Sin Pretty Print)}
\begin{lstlisting}[style=CStyle]
= PQEXEC ( DATABASE , PG_command ) ; PQclear ( res ) ; 
} 
} 
res = PQEXEC ( DATABASE , "END" ) ; PQclear ( res ) ; 
} 
void Calcula_ajuste ( int i ) 
{ 
preguntas [ i ] . ajuste = preguntas [ i ] . revision_especial + 
( preguntas [ i ] . excluir << 8 ) + 
( preguntas [ i ] . encoger << 9 ) + 
( preguntas [ i ] . verbatim << 10 ) + 
( preguntas [ i ] . header_encoger << 11 ) + 
( preguntas [ i ] . header_verbatim << 12 ) + 
( \end{lstlisting}
\end{frame}
\begin{frame}[fragile]
\frametitle{C\'odigo Preprocesado (Sin Pretty Print)}
\begin{lstlisting}[style=CStyle]
preguntas [ i ] . slide [ 0 ] << 13 ) + 
( preguntas [ i ] . encoger_opcion [ 0 ] << 14 ) + 
( preguntas [ i ] . verbatim_opcion [ 0 ] << 15 ) + 
( preguntas [ i ] . slide [ 1 ] << 16 ) + 
( preguntas [ i ] . encoger_opcion [ 1 ] << 17 ) + 
( preguntas [ i ] . verbatim_opcion [ 1 ] << 18 ) + 
( preguntas [ i ] . slide [ 2 ] << 19 ) + 
( preguntas [ i ] . encoger_opcion [ 2 ] << 20 ) + 
( preguntas [ i ] . verbatim_opcion [ 2 ] << 21 ) + 
( preguntas [ i ] . slide [ 3 ] << 22 ) + 
( preguntas [ i ] . encoger_opcion [ 3 ] << 23 ) + 
( preguntas [ i ] . verbatim_opcion [ 3 ] << 24 ) + 
( preguntas [ i ] . slide [ 4 ] << 25 ) + 
( \end{lstlisting}
\end{frame}
\begin{frame}[fragile]
\frametitle{C\'odigo Preprocesado (Sin Pretty Print)}
\begin{lstlisting}[style=CStyle]
preguntas [ i ] . encoger_opcion [ 4 ] << 26 ) + 
( preguntas [ i ] . verbatim_opcion [ 4 ] << 27 ) ; 
} 
void Analisis_General ( FILE * Archivo_Latex , long double alfa , long double Rpb , int Beamer_o_reporte ) 
{ 
char mensaje [ 2000 ] ; 
if ( alfa > 0.9 ) 
{ 
sprintf ( mensaje , "El examen muestra una textbf{excelente consistencia interna} ($alpha$ de Cronbach = textbf{%Lf}). Diversos 'itemes que miden la misma caracter'istica muestran un comportamiento bastante similar." , alfa ) ; 
Cajita_con_bandera ( Archivo_Latex , mensaje ,  0 , Beamer_o_reporte ) ; 
} 
else 
if ( alfa > 0.8 ) 
{ \end{lstlisting}
\end{frame}
\begin{frame}[fragile]
\frametitle{C\'odigo Preprocesado (Sin Pretty Print)}
\begin{lstlisting}[style=CStyle]

sprintf ( mensaje , "El examen muestra una textbf{buena consistencia interna} ($alpha$ de Cronbach = textbf{%Lf}). Diversos 'itemes que miden la misma caracter'istica muestran comportamientos similares." , alfa ) ; 
Cajita_con_bandera ( Archivo_Latex , mensaje ,  1 , Beamer_o_reporte ) ; 
} 
else 
if ( alfa > 0.7 ) 
{ 
sprintf ( mensaje , "La textbf{consistencia interna} del examen es aceptable, sin ser muy buena ($alpha$ de Cronbach = textbf{%Lf}). Se nota que el comportamiento de diversos 'itemes del examen es divergente." , alfa ) ; 
Cajita_con_bandera ( Archivo_Latex , mensaje ,  3 , Beamer_o_reporte ) ; 
} 
else 
{ 
sprintf ( mensaje , "La textbf{consistencia interna} de este examen es muy pobre ($alpha$ de Cronbach = textbf{%Lf}). Los 'itemes del examen muestran comportamientos divergente." , alfa ) ; 
Cajita_con_bandera \end{lstlisting}
\end{frame}
\begin{frame}[fragile]
\frametitle{C\'odigo Preprocesado (Sin Pretty Print)}
\begin{lstlisting}[style=CStyle]
( Archivo_Latex , mensaje ,  4 , Beamer_o_reporte ) ; 
} 
if ( Rpb > 0.3 ) 
{ 
sprintf ( mensaje , "La discriminaci'on promedio de los 'itemes es bastante alta (textbf{%Lf}). El examen es muy preciso en distinguir entre estudiantes de buen rendimiento y bajo rendimiento." , Rpb ) ; 
Cajita_con_bandera ( Archivo_Latex , mensaje ,  0 , Beamer_o_reporte ) ; 
} 
else 
if ( Rpb > 0.15 ) 
{ 
sprintf ( mensaje , "El examen muestra una buena discriminaci'on promedio (textbf{%Lf}). Distingue aceptablemente entre estudiantes de alto rendimiento y bajo rendimiento." , Rpb ) ; 
Cajita_con_bandera ( Archivo_Latex , mensaje ,  1 , Beamer_o_reporte ) ; 
} 
else \end{lstlisting}
\end{frame}
\begin{frame}[fragile]
\frametitle{C\'odigo Preprocesado (Sin Pretty Print)}
\begin{lstlisting}[style=CStyle]

if ( Rpb > 0.0 ) 
{ 
sprintf ( mensaje , "La discriminaci'on promedio del examen es muy baja (textbf{%Lf}) aunque positiva. Se recomienda una revisi'on general de los enunciados y las opciones." , Rpb ) ; 
Cajita_con_bandera ( Archivo_Latex , mensaje ,  3 , Beamer_o_reporte ) ; 
} 
else 
{ 
sprintf ( mensaje , "El promedio de discriminaci'on de las preguntas de este examen es negativo (textbf{%Lf}). Es indispensable hacer una revisión detallada de todos los enunciados y sus opciones." , Rpb ) ; 
Cajita_con_bandera ( Archivo_Latex , mensaje ,  4 , Beamer_o_reporte ) ; 
} 
} 
void Asigna_Banderas ( ) 
{ \end{lstlisting}
\end{frame}
\begin{frame}[fragile]
\frametitle{C\'odigo Preprocesado (Sin Pretty Print)}
\begin{lstlisting}[style=CStyle]

int i , j , N_distractores_usados ; 
long double alfa ; 
alfa = atof ( gtk_editable_get_chars ( GTK_EDITABLE ( EN_alfa_real ) , 0 , - 1 ) ) ; 
for ( i = 0 ; i < N_preguntas ; i ++ ) 
{ 
for ( j = 0 ; j <  16 ; j ++ ) preguntas [ i ] . flags [ j ] = 0 ; 
if ( ! preguntas [ i ] . buenos ) preguntas [ i ] . flags [  0 ] = 1 ; 
if ( ! preguntas [ i ] . malos ) preguntas [ i ] . flags [  1 ] = 1 ; 
if ( ( preguntas [ i ] . porcentaje < 0.31 ) && preguntas [ i ] . buenos ) preguntas [ i ] . flags [  2 ] = 1 ; 
if ( preguntas [ i ] . Rpb < 0.0 ) preguntas [ i ] . flags [  5 ] = 1 ; 
if ( preguntas [ i ] . Rpb >= 0.3 ) preguntas [ i ] . flags [  3 ] = 1 ; 
if ( ( preguntas [ i ] . porcentaje - preguntas [ i ] . previo ) > preguntas [ i ] . desviacion ) preguntas [ i ] . flags [  6 ] = 1 ; 
if \end{lstlisting}
\end{frame}
\begin{frame}[fragile]
\frametitle{C\'odigo Preprocesado (Sin Pretty Print)}
\begin{lstlisting}[style=CStyle]
( ( preguntas [ i ] . previo - preguntas [ i ] . porcentaje ) > preguntas [ i ] . desviacion ) preguntas [ i ] . flags [  7 ] = 1 ; 
if ( ( preguntas [ i ] . alfa_sin - alfa ) > 0.1 ) preguntas [ i ] . flags [  8 ] = 1 ; 
else if ( ( preguntas [ i ] . alfa_sin - alfa ) > 0.04 ) preguntas [ i ] . flags [  9 ] = 1 ; 
N_distractores_usados = 0 ; 
for ( j = 0 ; j < 5 ; j ++ ) 
{ 
if ( ( preguntas [ i ] . Rpb_opcion [ j ] > 0.3 ) && ( preguntas [ i ] . correcta != ( 'A' + j ) ) ) 
preguntas [ i ] . flags [  10 ] = 1 ; 
else 
if ( ( preguntas [ i ] . Rpb_opcion [ j ] > 0.0 ) && ( preguntas [ i ] . correcta != ( 'A' + j ) ) ) 
preguntas [ i ] . flags [  11 ] = 1 ; 
else 
if ( ( preguntas [ i ] . Rpb_opcion [ j ] < - 0.29 ) && ( preguntas [ i ] . correcta != ( 'A' + j ) ) ) 
preguntas \end{lstlisting}
\end{frame}
\begin{frame}[fragile]
\frametitle{C\'odigo Preprocesado (Sin Pretty Print)}
\begin{lstlisting}[style=CStyle]
[ i ] . flags [  12 ] = 1 ; 
if ( preguntas [ i ] . acumulado_opciones [ j ] > 0 ) N_distractores_usados ++ ; 
} 
if ( N_distractores_usados > 3 ) preguntas [ i ] . flags [  13 ] = 1 ; 
if ( ( preguntas [ i ] . Rpb >= 0.3 ) && ( preguntas [ i ] . porcentaje <= 0.75 ) && N_distractores_usados <= 3 ) preguntas [ i ] . flags [  14 ] = 1 ; 
if ( ( preguntas [ i ] . Rpb >= 0.3 ) && ( preguntas [ i ] . porcentaje <= 0.75 ) && N_distractores_usados > 3 ) preguntas [ i ] . flags [  15 ] = 1 ; 
} 
} 
void Color_Fila ( FILE * Archivo_Latex , int flags [  16 ] ) 
{ 
if ( flags [  0 ] || flags [  5 ] || flags [  8 ] || flags [  7 ] ) 
fprintf ( Archivo_Latex , "rowcolor{red}\n" ) ; 
else 
if \end{lstlisting}
\end{frame}
\begin{frame}[fragile]
\frametitle{C\'odigo Preprocesado (Sin Pretty Print)}
\begin{lstlisting}[style=CStyle]
( flags [  2 ] || flags [  9 ] ) 
fprintf ( Archivo_Latex , "rowcolor{yellow}\n" ) ; 
else 
if ( flags [  1 ] || flags [  6 ] || flags [  3 ] ) 
fprintf ( Archivo_Latex , "rowcolor{blue}\n" ) ; 
} 
void Lista_de_Preguntas ( FILE * Archivo_Latex , GtkWidget * PB , long double base , long double limite ) 
{ 
int i , actual ; 
char ejercicio_actual [  7 ] = "00000" ; 
char PG_command [ 3000 ] ; 
PGresult * res ; 
char hilera_antes [ 3000 ] , hilera_despues [ 3000 ] ; 
int \end{lstlisting}
\end{frame}
\begin{frame}[fragile]
\frametitle{C\'odigo Preprocesado (Sin Pretty Print)}
\begin{lstlisting}[style=CStyle]
N_preguntas_ejercicio ; 
actual = 0 ; 
strcpy ( ejercicio_actual , preguntas [ actual ] . ejercicio ) ; 
N_preguntas_ejercicio = 0 ; 
for ( i = actual ; ( i < N_preguntas ) && ( strcmp ( ejercicio_actual , preguntas [ i ] . ejercicio ) == 0 ) ; i ++ ) N_preguntas_ejercicio ++ ; 
sprintf ( PG_command , "SELECT usa_header, texto_header from bd_texto_ejercicios, bd_ejercicios where codigo_ejercicio = '%s' and texto_ejercicio = consecutivo_texto" , 
ejercicio_actual ) ; 
res = PQEXEC ( DATABASE , PG_command ) ; 
if ( N_preguntas_ejercicio > 1 ) fprintf ( Archivo_Latex , "\n\nrule{8.1cm}{5pt}\n\n" ) ; 
if ( * PQgetvalue ( res , 0 , 0 ) == 't' ) 
{ 
if ( N_preguntas_ejercicio > 1 ) 
{ 
if \end{lstlisting}
\end{frame}
\begin{frame}[fragile]
\frametitle{C\'odigo Preprocesado (Sin Pretty Print)}
\begin{lstlisting}[style=CStyle]
( N_preguntas_ejercicio == 2 ) 
fprintf ( Archivo_Latex , "textbf{Las preguntas %d y %d requieren la siguiente informaci'{o}n:}\n\n \n\n" , actual + 1 , actual + 2 ) ; 
else 
fprintf ( Archivo_Latex , "textbf{Las preguntas %d a %d requieren la siguiente informaci'{o}n:}\n\n \n\n" , actual + 1 , actual + N_preguntas_ejercicio ) ; 
strcpy ( hilera_antes , PQgetvalue ( res , 0 , 1 ) ) ; 
hilera_LATEX ( hilera_antes , hilera_despues ) ; 
fprintf ( Archivo_Latex , "%s\n" , hilera_despues ) ; 
fprintf ( Archivo_Latex , "rule{8.9cm}{1pt}\n" ) ; 
fprintf ( Archivo_Latex , "begin{questions}\n" ) ; 
} 
else 
{ 
fprintf ( Archivo_Latex , "begin{questions}\n" ) ; 
Imprime_pregunta \end{lstlisting}
\end{frame}
\begin{frame}[fragile]
\frametitle{C\'odigo Preprocesado (Sin Pretty Print)}
\begin{lstlisting}[style=CStyle]
( actual , Archivo_Latex , PQgetvalue ( res , 0 , 1 ) ) ; 
actual ++ ; 
N_preguntas_ejercicio = 0 ; 
} 
} 
else 
{ 
fprintf ( Archivo_Latex , "begin{questions}\n" ) ; 
} 
PQclear ( res ) ; 
for ( i = 0 ; i < N_preguntas_ejercicio ; i ++ ) 
{ 
Imprime_pregunta ( actual + i , Archivo_Latex , " " ) ; 
} \end{lstlisting}
\end{frame}
\begin{frame}[fragile]
\frametitle{C\'odigo Preprocesado (Sin Pretty Print)}
\begin{lstlisting}[style=CStyle]

if ( N_preguntas_ejercicio > 1 ) fprintf ( Archivo_Latex , "\n\nrule{8.1cm}{5pt}\n\n" ) ; 
actual += N_preguntas_ejercicio ; 
gtk_progress_bar_set_fraction ( GTK_PROGRESS_BAR ( PB_analisis ) , 0.5 + ( ( long double ) actual / N_preguntas * 0.3 ) ) ; 
while ( gtk_events_pending ( ) ) gtk_main_iteration ( ) ; 
while ( actual < N_preguntas ) 
{ 
strcpy ( ejercicio_actual , preguntas [ actual ] . ejercicio ) ; 
N_preguntas_ejercicio = 0 ; 
for ( i = actual ; ( i < N_preguntas ) && ( strcmp ( ejercicio_actual , preguntas [ i ] . ejercicio ) == 0 ) ; i ++ ) N_preguntas_ejercicio ++ ; 
if ( N_preguntas_ejercicio > 1 ) fprintf ( Archivo_Latex , "\n\nrule{8.1cm}{5pt}\n\n" ) ; 
sprintf ( PG_command , "SELECT usa_header, texto_header from bd_texto_ejercicios, bd_ejercicios where codigo_ejercicio = '%s' and texto_ejercicio = consecutivo_texto" , 
ejercicio_actual ) ; 
res \end{lstlisting}
\end{frame}
\begin{frame}[fragile]
\frametitle{C\'odigo Preprocesado (Sin Pretty Print)}
\begin{lstlisting}[style=CStyle]
= PQEXEC ( DATABASE , PG_command ) ; 
if ( * PQgetvalue ( res , 0 , 0 ) == 't' ) 
{ 
if ( N_preguntas_ejercicio > 1 ) 
{ 
if ( N_preguntas_ejercicio == 2 ) 
fprintf ( Archivo_Latex , "textbf{Las preguntas %d y %d requieren la siguiente informaci'{o}n:}\n\n \n\n" , actual + 1 , actual + 2 ) ; 
else 
fprintf ( Archivo_Latex , "textbf{Las preguntas %d a %d requieren la siguiente informaci'{o}n:}\n\n \n\n" , actual + 1 , actual + N_preguntas_ejercicio ) ; 
strcpy ( hilera_antes , PQgetvalue ( res , 0 , 1 ) ) ; 
hilera_LATEX ( hilera_antes , hilera_despues ) ; 
fprintf ( Archivo_Latex , "%s\n" , hilera_despues ) ; 
fprintf ( Archivo_Latex , "rule{8cm}{1pt}\n" ) ; 
} \end{lstlisting}
\end{frame}
\begin{frame}[fragile]
\frametitle{C\'odigo Preprocesado (Sin Pretty Print)}
\begin{lstlisting}[style=CStyle]

else 
{ 
Imprime_pregunta ( actual , Archivo_Latex , PQgetvalue ( res , 0 , 1 ) ) ; 
N_preguntas_ejercicio = 0 ; 
actual ++ ; 
} 
} 
for ( i = 0 ; i < N_preguntas_ejercicio ; i ++ ) 
{ 
Imprime_pregunta ( actual + i , Archivo_Latex , " " ) ; 
} 
if ( N_preguntas_ejercicio > 1 ) fprintf ( Archivo_Latex , "\n\nrule{8.1cm}{5pt}\n\n" ) ; 
actual \end{lstlisting}
\end{frame}
\begin{frame}[fragile]
\frametitle{C\'odigo Preprocesado (Sin Pretty Print)}
\begin{lstlisting}[style=CStyle]
+= N_preguntas_ejercicio ; 
gtk_progress_bar_set_fraction ( GTK_PROGRESS_BAR ( PB_analisis ) , 0.5 + ( ( long double ) actual / N_preguntas * 0.3 ) ) ; 
while ( gtk_events_pending ( ) ) gtk_main_iteration ( ) ; 
} 
fprintf ( Archivo_Latex , "end{questions}\n\n" ) ; 
} 
void Imprime_pregunta ( int i , FILE * Archivo_Latex , char * prefijo ) 
{ 
long double cota_inferior ; 
char hilera_antes [ 4000 ] , hilera_despues [ 4000 ] ; 
char PG_command [ 2000 ] ; 
PGresult * res ; 
long double Por_A , Por_B , Por_C , Por_D , Por_E , Total ; 
gchar \end{lstlisting}
\end{frame}
\begin{frame}[fragile]
\frametitle{C\'odigo Preprocesado (Sin Pretty Print)}
\begin{lstlisting}[style=CStyle]
* materia ; 
materia = gtk_editable_get_chars ( GTK_EDITABLE ( EN_materia ) , 0 , - 1 ) ; 
cota_inferior =  1.15  /  6.65  * 100.0 ; 
sprintf ( PG_command , "SELECT texto_pregunta, texto_opcion_A, texto_opcion_B, texto_opcion_C, texto_opcion_D, texto_opcion_E from bd_texto_preguntas where codigo_unico_pregunta = '%s'" , preguntas [ i ] . pregunta ) ; 
res = PQEXEC ( DATABASE , PG_command ) ; 
sprintf ( hilera_antes , "question %s\n%s" , prefijo , PQgetvalue ( res , 0 , 0 ) ) ; 
hilera_LATEX ( hilera_antes , hilera_despues ) ; 
fprintf ( Archivo_Latex , "%s\n" , hilera_despues ) ; 
Por_A = ( long double ) preguntas [ i ] . acumulado_opciones [ 0 ] / N_estudiantes * 100.0 ; 
Por_B = ( long double ) preguntas [ i ] . acumulado_opciones [ 1 ] / N_estudiantes * 100.0 ; 
Por_C = ( long double ) preguntas [ i ] . acumulado_opciones [ 2 ] / N_estudiantes * 100.0 ; 
Por_D = ( long double ) preguntas [ i ] . acumulado_opciones [ 3 ] / N_estudiantes * 100.0 ; 
Por_E = ( long double ) preguntas [ i ] . acumulado_opciones [ 4 ] / N_estudiantes * 100.0 ; 
fprintf \end{lstlisting}
\end{frame}
\begin{frame}[fragile]
\frametitle{C\'odigo Preprocesado (Sin Pretty Print)}
\begin{lstlisting}[style=CStyle]
( Archivo_Latex , "begin{answers}\n" ) ; 
Imprime_Opcion ( Archivo_Latex , res , Por_A , i , 0 ) ; 
Imprime_Opcion ( Archivo_Latex , res , Por_B , i , 1 ) ; 
Imprime_Opcion ( Archivo_Latex , res , Por_C , i , 2 ) ; 
Imprime_Opcion ( Archivo_Latex , res , Por_D , i , 3 ) ; 
Imprime_Opcion ( Archivo_Latex , res , Por_E , i , 4 ) ; 
fprintf ( Archivo_Latex , "end{answers}\n" ) ; 
Analiza_Banderas ( Archivo_Latex , preguntas [ i ] , 0 , 0 , 0 , NULL ) ; 
Analiza_Ajuste ( Archivo_Latex , preguntas [ i ] , 0 ) ; 
fprintf ( Archivo_Latex , "framebox[7.5cm][l]{%.6s.%d - %.6s - %s %s %s}\n\n" , 
preguntas [ i ] . ejercicio , preguntas [ i ] . secuencia , preguntas [ i ] . pregunta , materia , preguntas [ i ] . tema , preguntas [ i ] . subtema ) ; 
sprintf ( hilera_antes , "framebox[7.5cm][l]{Autor: %s}" , preguntas [ i ] . autor ) ; 
hilera_LATEX ( hilera_antes , hilera_despues ) ; 
fprintf \end{lstlisting}
\end{frame}
\begin{frame}[fragile]
\frametitle{C\'odigo Preprocesado (Sin Pretty Print)}
\begin{lstlisting}[style=CStyle]
( Archivo_Latex , "%s\n\n" , hilera_despues ) ; 
fprintf ( Archivo_Latex , "rule{8cm}{1pt}\n" ) ; 
fprintf ( Archivo_Latex , "\n\n" ) ; 
g_free ( materia ) ; 
} 
void Imprime_Opcion ( FILE * Archivo_Latex , PGresult * res , long double Porcentaje , int pregunta , int opcion ) 
{ 
char hilera_antes [ 3000 ] , hilera_despues [ 3000 ] ; 
fprintf ( Archivo_Latex , "item " ) ; 
strcpy ( hilera_antes , PQgetvalue ( res , 0 , 1 + opcion ) ) ; 
hilera_LATEX ( hilera_antes , hilera_despues ) ; 
fprintf ( Archivo_Latex , "%s" , hilera_despues ) ; 
fprintf ( Archivo_Latex , "\n\n{color{green} rule{%Lfcm}{5pt}\n\n{footnotesize textbf{texttt{%6.2Lf %%}}}} (textbf{%d})\n\n" , 
0.05 \end{lstlisting}
\end{frame}
\begin{frame}[fragile]
\frametitle{C\'odigo Preprocesado (Sin Pretty Print)}
\begin{lstlisting}[style=CStyle]
+  6.65  * Porcentaje / 100.0 , Porcentaje , 
preguntas [ pregunta ] . acumulado_opciones [ opcion ] ) ; 
fprintf ( Archivo_Latex , " $r_{pb}$ = textbf{%Lf}\n\n" , preguntas [ pregunta ] . Rpb_opcion [ opcion ] ) ; 
if ( preguntas [ pregunta ] . correcta == ( 'A' + opcion ) ) 
{ 
fprintf ( Archivo_Latex , "setlength{fboxrule}{2fboxrule}\n" ) ; 
fprintf ( Archivo_Latex , "\n\nfcolorbox{black}{blue}{color{white} $starstarstar$ textbf{CORRECTA} $starstarstar$}" ) ; 
fprintf ( Archivo_Latex , "setlength{fboxrule}{0.5fboxrule}\n" ) ; 
} 
fprintf ( Archivo_Latex , "\n\n" ) ; 
} 
void Analiza_Banderas ( FILE * Archivo_Latex , struct PREGUNTA item , int beamer , int N_flags , int i , char * Descripcion ) 
{ 
int \end{lstlisting}
\end{frame}
\begin{frame}[fragile]
\frametitle{C\'odigo Preprocesado (Sin Pretty Print)}
\begin{lstlisting}[style=CStyle]
N ; 
int Ajuste ; 
long double alfa ; 
char mensaje [ 1000 ] ; 
Ajuste = ( item . revision_especial !=  0 ) ; 
N = Ajuste ; 
alfa = atof ( gtk_editable_get_chars ( GTK_EDITABLE ( EN_alfa_real ) , 0 , - 1 ) ) ; 
if ( item . flags [  3 ] ) 
{ 
sprintf ( mensaje , "Esta pregunta muestra un buen '{i}ndice de discrimaci'{o}n ($r_{pb}$ = textbf{%6.4Lf}). Separa muy bien a los estudiantes de buen rendimiento de los de bajo rendimiento." , 
item . Rpb ) ; 
Cajita_con_bandera ( Archivo_Latex , mensaje ,  1 , beamer ) ; 
N ++ ; 
if \end{lstlisting}
\end{frame}
\begin{frame}[fragile]
\frametitle{C\'odigo Preprocesado (Sin Pretty Print)}
\begin{lstlisting}[style=CStyle]
( beamer && ( ( N % 3 ) == 0 ) && N < ( N_flags + Ajuste ) ) 
Continuar_banderas ( Archivo_Latex , i , Descripcion ) ; 
} 
if ( item . flags [  12 ] ) 
{ 
sprintf ( mensaje , "Hay al menos un textbf{distractor} con un $r_{pb}$ muy negativo, que atrajo a los estudiantes de bajo rendimiento y no fue considerado por los estudiantes de buen rendimiento. textbf{Buen distractor}." ) ; 
Cajita_con_bandera ( Archivo_Latex , mensaje ,  2 , beamer ) ; 
N ++ ; 
if ( beamer && ( ( N % 3 ) == 0 ) && N < ( N_flags + Ajuste ) ) 
Continuar_banderas ( Archivo_Latex , i , Descripcion ) ; 
} 
if ( item . flags [  13 ] ) 
{ 
sprintf \end{lstlisting}
\end{frame}
\begin{frame}[fragile]
\frametitle{C\'odigo Preprocesado (Sin Pretty Print)}
\begin{lstlisting}[style=CStyle]
( mensaje , "Por lo menos 4 opciones diferentes fueron escogidas por los estudiantes." ) ; 
Cajita_con_bandera ( Archivo_Latex , mensaje ,  2 , beamer ) ; 
N ++ ; 
if ( beamer && ( ( N % 3 ) == 0 ) && N < ( N_flags + Ajuste ) ) 
Continuar_banderas ( Archivo_Latex , i , Descripcion ) ; 
} 
if ( item . flags [  6 ] ) 
{ 
sprintf ( mensaje , "La media de respuestas correctas a esta pregunta (textbf{%6.4Lf}) textbf{supera} en m'{a}s de una desviaci'{o}n est'{a}ndar (textbf{%6.4Lf}) a la media hist'{o}rica o estimada (textbf{%6.4Lf})." , 
item . porcentaje , item . desviacion , item . previo ) ; 
Cajita_con_bandera ( Archivo_Latex , mensaje ,  2 , beamer ) ; 
N ++ ; 
if ( beamer && ( ( N % 3 ) == 0 ) && N < ( N_flags + Ajuste ) ) 
Continuar_banderas \end{lstlisting}
\end{frame}
\begin{frame}[fragile]
\frametitle{C\'odigo Preprocesado (Sin Pretty Print)}
\begin{lstlisting}[style=CStyle]
( Archivo_Latex , i , Descripcion ) ; 
} 
if ( item . flags [  1 ] ) 
{ 
sprintf ( mensaje , "textbf{Todos} los estudiantes contestaron correctamente, por lo que la pregunta no discrimina de manera efectiva." ) ; 
Cajita_con_bandera ( Archivo_Latex , mensaje ,  3 , beamer ) ; 
N ++ ; 
if ( beamer && ( ( N % 3 ) == 0 ) && N < ( N_flags + Ajuste ) ) 
Continuar_banderas ( Archivo_Latex , i , Descripcion ) ; 
} 
if ( item . flags [  2 ] ) 
{ 
sprintf ( mensaje , "S'{o}lo el textbf{%5.2Lf}%% de los estudiantes contestaron correctamente esta pregunta." , 
item \end{lstlisting}
\end{frame}
\begin{frame}[fragile]
\frametitle{C\'odigo Preprocesado (Sin Pretty Print)}
\begin{lstlisting}[style=CStyle]
. porcentaje * 100 ) ; 
Cajita_con_bandera ( Archivo_Latex , mensaje ,  3 , beamer ) ; 
N ++ ; 
if ( beamer && ( ( N % 3 ) == 0 ) && N < ( N_flags + Ajuste ) ) 
Continuar_banderas ( Archivo_Latex , i , Descripcion ) ; 
} 
if ( item . flags [  7 ] ) 
{ 
sprintf ( mensaje , "La media de respuestas correctas a esta pregunta (textbf{%6.4Lf}) es textbf{menor} en m'{a}s de una desviaci'{o}n est'{a}ndar (textbf{%6.4Lf}) a la media hist'{o}rica o estimada (textbf{%6.4Lf})." , 
item . porcentaje , item . desviacion , item . previo ) ; 
Cajita_con_bandera ( Archivo_Latex , mensaje ,  3 , beamer ) ; 
N ++ ; 
if ( beamer && ( ( N % 3 ) == 0 ) && N < ( N_flags + Ajuste ) ) 
Continuar_banderas \end{lstlisting}
\end{frame}
\begin{frame}[fragile]
\frametitle{C\'odigo Preprocesado (Sin Pretty Print)}
\begin{lstlisting}[style=CStyle]
( Archivo_Latex , i , Descripcion ) ; 
} 
if ( item . flags [  9 ] ) 
{ 
sprintf ( mensaje , "Si esta pregunta se elimina del examen, el $alpha$ de Cronbach subir'{i}a de textbf{%6.4Lf} a textbf{%6.4Lf}." , 
alfa , item . alfa_sin ) ; 
Cajita_con_bandera ( Archivo_Latex , mensaje ,  3 , beamer ) ; 
N ++ ; 
if ( beamer && ( ( N % 3 ) == 0 ) && N < ( N_flags + Ajuste ) ) 
Continuar_banderas ( Archivo_Latex , i , Descripcion ) ; 
} 
if ( item . flags [  11 ] ) 
{ 
sprintf \end{lstlisting}
\end{frame}
\begin{frame}[fragile]
\frametitle{C\'odigo Preprocesado (Sin Pretty Print)}
\begin{lstlisting}[style=CStyle]
( mensaje , "Hay al menos un textbf{distractor} con un $r_{pb}$ ligeramente positivo. Revisar enunciado y opciones." ) ; 
Cajita_con_bandera ( Archivo_Latex , mensaje ,  3 , beamer ) ; 
N ++ ; 
if ( beamer && ( ( N % 3 ) == 0 ) && N < ( N_flags + Ajuste ) ) 
Continuar_banderas ( Archivo_Latex , i , Descripcion ) ; 
} 
if ( item . flags [  5 ] ) 
{ 
sprintf ( mensaje , "Esta pregunta muestra un '{i}ndice de discrimaci'{o}n ($r_{pb}$ = textbf{%6.4Lf}) negativo. Los estudiantes de buen rendimiento en este examen tendieron a equivocarse, mientras que los de bajo rendimiento tendieron a contestarla bien. textbf{Revisar muy bien el enunciado y las opciones de la pregunta}" , 
item . Rpb ) ; 
Cajita_con_bandera ( Archivo_Latex , mensaje ,  4 , beamer ) ; 
N ++ ; 
if ( beamer && ( ( N % 3 ) == 0 ) && N < ( N_flags + Ajuste ) ) 
Continuar_banderas \end{lstlisting}
\end{frame}
\begin{frame}[fragile]
\frametitle{C\'odigo Preprocesado (Sin Pretty Print)}
\begin{lstlisting}[style=CStyle]
( Archivo_Latex , i , Descripcion ) ; 
} 
if ( item . flags [  0 ] ) 
{ 
sprintf ( mensaje , "textbf{Todos} los estudiantes contestaron equivocadamente esta pregunta. La pregunta no discrimina de manera efectiva y posiblemente est'a mal redactada." ) ; 
Cajita_con_bandera ( Archivo_Latex , mensaje ,  4 , beamer ) ; 
N ++ ; 
if ( beamer && ( ( N % 3 ) == 0 ) && N < ( N_flags + Ajuste ) ) 
Continuar_banderas ( Archivo_Latex , i , Descripcion ) ; 
} 
if ( item . flags [  4 ] ) 
{ 
sprintf ( mensaje , "Esta pregunta muestra un '{i}ndice de discrimaci'{o}n ($r_{pb}$ = textbf{%6.4Lf}) muy negativo. Los estudiantes de buen rendimiento en este examen tendieron a equivocarse, mientras que los de bajo rendimiento tendieron a contestarla bien. textbf{Revisar muy bien el enunciado y las opciones de la pregunta}" , 
item \end{lstlisting}
\end{frame}
\begin{frame}[fragile]
\frametitle{C\'odigo Preprocesado (Sin Pretty Print)}
\begin{lstlisting}[style=CStyle]
. Rpb ) ; 
Cajita_con_bandera ( Archivo_Latex , mensaje ,  4 , beamer ) ; 
N ++ ; 
if ( beamer && ( ( N % 3 ) == 0 ) && N < ( N_flags + Ajuste ) ) 
Continuar_banderas ( Archivo_Latex , i , Descripcion ) ; 
} 
if ( item . flags [  8 ] ) 
{ 
sprintf ( mensaje , "Si esta pregunta se elimina del examen, el $alpha$ de Cronbach subir'{i}a textbf{considerablemente} (de textbf{%6.4Lf} a textbf{%6.4Lf})." , 
alfa , item . alfa_sin ) ; 
Cajita_con_bandera ( Archivo_Latex , mensaje ,  4 , beamer ) ; 
N ++ ; 
if ( beamer && ( ( N % 3 ) == 0 ) && N < ( N_flags + Ajuste ) ) 
Continuar_banderas \end{lstlisting}
\end{frame}
\begin{frame}[fragile]
\frametitle{C\'odigo Preprocesado (Sin Pretty Print)}
\begin{lstlisting}[style=CStyle]
( Archivo_Latex , i , Descripcion ) ; 
} 
if ( item . flags [  10 ] ) 
{ 
sprintf ( mensaje , "Hay al menos un textbf{distractor} con un $r_{pb}$ muy positivo, esto significa que, pese a ser incorrecto, atrajo a los estudiantes de mejor rendimiento. textbf{Revisar cuidadosamente enunciado y opciones}." ) ; 
Cajita_con_bandera ( Archivo_Latex , mensaje ,  4 , beamer ) ; 
N ++ ; 
if ( beamer && ( ( N % 3 ) == 0 ) && N < ( N_flags + Ajuste ) ) 
Continuar_banderas ( Archivo_Latex , i , Descripcion ) ; 
} 
if ( item . flags [  14 ] ) 
{ 
sprintf ( mensaje , "Esta pregunta no es f'acil (textit{p} = textbf{%6.4Lf}) y tiene un buen 'indice de discriminaci'on ($r_{pb}$ = textbf{%6.4Lf}). textbf{Buena pregunta}\n" , 
item \end{lstlisting}
\end{frame}
\begin{frame}[fragile]
\frametitle{C\'odigo Preprocesado (Sin Pretty Print)}
\begin{lstlisting}[style=CStyle]
. porcentaje , item . Rpb ) ; 
Cajita_con_bandera ( Archivo_Latex , mensaje ,  1 , beamer ) ; 
N ++ ; 
if ( beamer && ( ( N % 3 ) == 0 ) && N < ( N_flags + Ajuste ) ) 
Continuar_banderas ( Archivo_Latex , i , Descripcion ) ; 
} 
if ( item . flags [  15 ] ) 
{ 
sprintf ( mensaje , "Esta pregunta no es f'acil (textit{p} = textbf{%6.4Lf}), muestra un buen 'indice de discriminaci'on ($r_{pb}$ = textbf{%6.4Lf}), y los estudiantes usaron al menos 4 de las opciones. textbf{Excelente pregunta}\n" , 
item . porcentaje , item . Rpb ) ; 
Cajita_con_bandera ( Archivo_Latex , mensaje ,  0 , beamer ) ; 
N ++ ; 
if ( beamer && ( ( N % 3 ) == 0 ) && N < ( N_flags + Ajuste ) ) 
Continuar_banderas \end{lstlisting}
\end{frame}
\begin{frame}[fragile]
\frametitle{C\'odigo Preprocesado (Sin Pretty Print)}
\begin{lstlisting}[style=CStyle]
( Archivo_Latex , i , Descripcion ) ; 
} 
} 
void Continuar_banderas ( FILE * Archivo_Latex , int i , char * Descripcion ) 
{ 
char hilera_antes [ 1000 ] ; 
char hilera_despues [ 1000 ] ; 
fprintf ( Archivo_Latex , "end{frame}\n" ) ; 
fprintf ( Archivo_Latex , "}\n" ) ; 
fprintf ( Archivo_Latex , "{\n" ) ; 
sprintf ( hilera_antes , "begin{frame}{Análisis de Pregunta %d - cont. hfill {small %s}}" , i , Descripcion ) ; 
hilera_LATEX ( hilera_antes , hilera_despues ) ; 
fprintf ( Archivo_Latex , "%s\n" , hilera_despues ) ; 
} \end{lstlisting}
\end{frame}
\begin{frame}[fragile]
\frametitle{C\'odigo Preprocesado (Sin Pretty Print)}
\begin{lstlisting}[style=CStyle]

void Analiza_Ajuste ( FILE * Archivo_Latex , struct PREGUNTA item , int modo ) 
{ 
if ( item . revision_especial ==  1 ) 
Cajita_con_bandera ( Archivo_Latex , "textbf{Ajuste:} Pregunta no ser'a tomada en cuenta para la evaluaci'on." ,  5 , modo ) ; 
if ( item . revision_especial ==  2 ) 
Cajita_con_bandera ( Archivo_Latex , "textbf{Ajuste:} Se considerar'an varias opciones como correctas." ,  5 , modo ) ; 
if ( item . revision_especial ==  3 ) 
Cajita_con_bandera ( Archivo_Latex , "textbf{Ajuste:} Debido a problemas en su formulaci'on, esta pregunta se da como textbf{correcta} a todos los estudiantes." ,  5 , modo ) ; 
if ( item . revision_especial ==  4 ) 
Cajita_con_bandera ( Archivo_Latex , "textbf{Ajuste:} Debido a situaciones especiales, esta pregunta se da como textbf{incorrecta} a todos los estudiantes." ,  5 , modo ) ; 
if ( item . revision_especial ==  5 ) 
Cajita_con_bandera ( Archivo_Latex , "textbf{Ajuste:} Por caracter'isticas especiales (dificultad, ambig\"uedad, falta de informaci'on, falta de tiempo, etc.), esta pregunta ser'a tomada en cuenta 'unicamente para los estudiantes que la contestaron correctamente." ,  5 , modo ) ; 
if \end{lstlisting}
\end{frame}
\begin{frame}[fragile]
\frametitle{C\'odigo Preprocesado (Sin Pretty Print)}
\begin{lstlisting}[style=CStyle]
( item . revision_especial ==  6 ) 
Cajita_con_bandera ( Archivo_Latex , "textbf{Ajuste:} Esta pregunta otorga un bono a los estudiantes que la contestaron correctamente." ,  5 , modo ) ; 
if ( item . revision_especial ==  7 ) 
Cajita_con_bandera ( Archivo_Latex , "textbf{Ajuste:} Esta pregunta es textbf{extra} al examen, dando puntos adicionales sobre 100 a los que la contesten correctamente." ,  5 , modo ) ; 
} 
void Cajita_con_bandera ( FILE * Archivo_Latex , char * mensaje , int color , int modo ) 
{ 
int i ; 
Banderas [ color ] ++ ; 
fprintf ( Archivo_Latex , "begin{figure}[H]\n" ) ; 
fprintf ( Archivo_Latex , "centering\n" ) ; 
fprintf ( Archivo_Latex , "setlength{fboxrule}{4fboxrule}\n" ) ; 
fprintf ( Archivo_Latex , "fcolorbox{%s}{white}{\n" , colores [ color ] ) ; 
fprintf \end{lstlisting}
\end{frame}
\begin{frame}[fragile]
\frametitle{C\'odigo Preprocesado (Sin Pretty Print)}
\begin{lstlisting}[style=CStyle]
( Archivo_Latex , "begin{minipage}{1.4 cm}\n" ) ; 
if ( color ==  5 ) 
fprintf ( Archivo_Latex , "includegraphics[scale=0.27]{.imagenes/%s}\n" , banderas [ color ] ) ; 
else 
fprintf ( Archivo_Latex , "includegraphics[scale=0.07, angle=45]{.imagenes/%s}\n" , banderas [ color ] ) ; 
fprintf ( Archivo_Latex , "end{minipage}\n" ) ; 
if ( modo == 1 ) 
fprintf ( Archivo_Latex , "begin{minipage}{8.6 cm}\n" ) ; 
else 
fprintf ( Archivo_Latex , "begin{minipage}{6.1 cm}\n" ) ; 
fprintf ( Archivo_Latex , "small{%s}\n" , mensaje ) ; 
fprintf ( Archivo_Latex , "end{minipage}\n" ) ; 
fprintf ( Archivo_Latex , "}\n" ) ; 
fprintf \end{lstlisting}
\end{frame}
\begin{frame}[fragile]
\frametitle{C\'odigo Preprocesado (Sin Pretty Print)}
\begin{lstlisting}[style=CStyle]
( Archivo_Latex , "setlength{fboxrule}{0.25fboxrule}\n" ) ; 
fprintf ( Archivo_Latex , "end{figure}\n" ) ; 
} 
void Resumen_de_Banderas ( FILE * Archivo_Latex , int modo ) 
{ 
int color ; 
if ( ! modo ) 
fprintf ( Archivo_Latex , "center{textbf{Resumen de Observaciones}}\n\n" ) ; 
for ( color = 0 ; color < 5 ; color ++ ) 
{ 
fprintf ( Archivo_Latex , "begin{minipage}{3.50 cm}\n" ) ; 
fprintf ( Archivo_Latex , "begin{figure}[H]\n" ) ; 
fprintf ( Archivo_Latex , "centering\n" ) ; 
fprintf \end{lstlisting}
\end{frame}
\begin{frame}[fragile]
\frametitle{C\'odigo Preprocesado (Sin Pretty Print)}
\begin{lstlisting}[style=CStyle]
( Archivo_Latex , "setlength{fboxrule}{4fboxrule}\n" ) ; 
fprintf ( Archivo_Latex , "fcolorbox{%s}{white}{\n" , colores [ color ] ) ; 
fprintf ( Archivo_Latex , "begin{minipage}{1.5 cm}\n" ) ; 
fprintf ( Archivo_Latex , "includegraphics[scale=0.07, angle=45]{.imagenes/%s}\n" , banderas [ color ] ) ; 
fprintf ( Archivo_Latex , "end{minipage}\n" ) ; 
fprintf ( Archivo_Latex , "begin{minipage}{0.5 cm}\n" ) ; 
fprintf ( Archivo_Latex , "large{textbf{%d}}\n" , Banderas [ color ] ) ; 
fprintf ( Archivo_Latex , "end{minipage}\n" ) ; 
fprintf ( Archivo_Latex , "}\n" ) ; 
fprintf ( Archivo_Latex , "setlength{fboxrule}{0.25fboxrule}\n" ) ; 
fprintf ( Archivo_Latex , "end{figure}\n" ) ; 
fprintf ( Archivo_Latex , "end{minipage}\n" ) ; 
} 
fprintf \end{lstlisting}
\end{frame}
\begin{frame}[fragile]
\frametitle{C\'odigo Preprocesado (Sin Pretty Print)}
\begin{lstlisting}[style=CStyle]
( Archivo_Latex , "begin{minipage}{3.50 cm}\n" ) ; 
fprintf ( Archivo_Latex , "begin{figure}[H]\n" ) ; 
fprintf ( Archivo_Latex , "centering\n" ) ; 
fprintf ( Archivo_Latex , "setlength{fboxrule}{4fboxrule}\n" ) ; 
fprintf ( Archivo_Latex , "fcolorbox{%s}{white}{\n" , colores [  5 ] ) ; 
fprintf ( Archivo_Latex , "begin{minipage}{1.5 cm}\n" ) ; 
fprintf ( Archivo_Latex , "includegraphics[scale=0.27]{.imagenes/%s}\n" , banderas [  5 ] ) ; 
fprintf ( Archivo_Latex , "end{minipage}\n" ) ; 
fprintf ( Archivo_Latex , "begin{minipage}{0.5 cm}\n" ) ; 
fprintf ( Archivo_Latex , "large{textbf{%d}}\n" , Banderas [  5 ] ) ; 
fprintf ( Archivo_Latex , "end{minipage}\n" ) ; 
fprintf ( Archivo_Latex , "}\n" ) ; 
fprintf ( Archivo_Latex , "setlength{fboxrule}{0.25fboxrule}\n" ) ; 
fprintf \end{lstlisting}
\end{frame}
\begin{frame}[fragile]
\frametitle{C\'odigo Preprocesado (Sin Pretty Print)}
\begin{lstlisting}[style=CStyle]
( Archivo_Latex , "end{figure}\n" ) ; 
fprintf ( Archivo_Latex , "end{minipage}\n" ) ; 
} 
void Lista_de_Notas ( FILE * Archivo_Latex ) 
{ 
int i , k , N_estudiantes ; 
char examen [ 10 ] ; 
char PG_command [ 2000 ] ; 
PGresult * res ; 
long double Porcentaje , Porcentaje_ajustado ; 
int N_correctas , N_ajustado , M_ajustado ; 
char hilera_antes [ 2000 ] , hilera_despues [ 2000 ] ; 
k = ( int ) gtk_spin_button_get_value_as_int ( SP_examen ) ; 
sprintf \end{lstlisting}
\end{frame}
\begin{frame}[fragile]
\frametitle{C\'odigo Preprocesado (Sin Pretty Print)}
\begin{lstlisting}[style=CStyle]
( examen , "%05d" , k ) ; 
sprintf ( PG_command , "SELECT nombre, version, respuestas, correctas, porcentaje from EX_examenes_respuestas where examen = '%s' order by nombre" , examen ) ; 
res = PQEXEC ( DATABASE , PG_command ) ; 
N_estudiantes = PQntuples ( res ) ; 
fprintf ( Archivo_Latex , "\n\n" ) ; 
fprintf ( Archivo_Latex , "begin{center}\n" ) ; 
fprintf ( Archivo_Latex , "begin{longtable}{|l|c|c|c|c|c|}\n" ) ; 
fprintf ( Archivo_Latex , "hline\n" ) ; 
fprintf ( Archivo_Latex , "textbf{Nombre} & textbf{Versi'{o}n} & textbf{Correctas} & textbf{Porcentaje} & textbf{Ajuste} & textbf{Nota Ajustada} hline hline\n" ) ; 
fprintf ( Archivo_Latex , "endfirsthead\n" ) ; 
fprintf ( Archivo_Latex , "hline\n" ) ; 
fprintf ( Archivo_Latex , "textbf{Nombre} & textbf{Versi'{o}n} & textbf{Correctas} & textbf{Porcentaje} & textbf{Ajuste} & textbf{Nota Ajustada} hline hline\n" ) ; 
fprintf ( Archivo_Latex , "endhead\n" ) ; 
for \end{lstlisting}
\end{frame}
\begin{frame}[fragile]
\frametitle{C\'odigo Preprocesado (Sin Pretty Print)}
\begin{lstlisting}[style=CStyle]
( i = 0 ; i < N_estudiantes ; i ++ ) 
{ 
Calcula_Notas ( PQgetvalue ( res , i , 1 ) , 
PQgetvalue ( res , i , 2 ) , 
& N_correctas , & N_ajustado , & M_ajustado ) ; 
Porcentaje = ( long double ) N_correctas / N_preguntas * 100.0 ; 
if ( M_ajustado ) 
Porcentaje_ajustado = ( long double ) N_ajustado / M_ajustado * 100.0 ; 
else 
Porcentaje_ajustado = 0.0 ; 
sprintf ( hilera_antes , "%s & %s & %d/%d & %7.2Lf & %d/%d & textbf{%7.2Lf}  hline" , 
PQgetvalue ( res , i , 0 ) , PQgetvalue ( res , i , 1 ) , 
N_correctas , N_preguntas , Porcentaje , 
N_ajustado \end{lstlisting}
\end{frame}
\begin{frame}[fragile]
\frametitle{C\'odigo Preprocesado (Sin Pretty Print)}
\begin{lstlisting}[style=CStyle]
, M_ajustado , Porcentaje_ajustado ) ; 
hilera_LATEX ( hilera_antes , hilera_despues ) ; 
fprintf ( Archivo_Latex , "%s\n" , hilera_despues ) ; 
} 
fprintf ( Archivo_Latex , "end{longtable}\n" ) ; 
fprintf ( Archivo_Latex , "end{center}\n" ) ; 
} 
void Calcula_Notas ( char * version , char * respuestas , int * n_buenas , int * n_ajustado , int * m_ajustado ) 
{ 
int i , k_version ; 
int N , K , M ; 
for ( k_version = 0 ; ( k_version < N_versiones ) && strcmp ( versiones [ k_version ] . codigo , version ) != 0 ; k_version ++ ) ; 
N = 0 ; 
for \end{lstlisting}
\end{frame}
\begin{frame}[fragile]
\frametitle{C\'odigo Preprocesado (Sin Pretty Print)}
\begin{lstlisting}[style=CStyle]
( i = 0 ; i < N_preguntas ; i ++ ) 
if ( versiones [ k_version ] . preguntas [ i ] . respuesta == respuestas [ i ] ) N ++ ; 
if ( N_ajustes == 0 ) 
{ 
K = N ; 
M = N_preguntas ; 
} 
else 
Calcula_nota_ajustada ( k_version , respuestas , & K , & M ) ; 
* n_buenas = N ; 
* n_ajustado = K ; 
* m_ajustado = M ; 
} 
void \end{lstlisting}
\end{frame}
\begin{frame}[fragile]
\frametitle{C\'odigo Preprocesado (Sin Pretty Print)}
\begin{lstlisting}[style=CStyle]
Calcula_nota_ajustada ( int k_version , char * respuestas , int * n , int * m ) 
{ 
int i , k ; 
int N_correctas , N_evaluadas ; 
N_correctas = N_evaluadas = 0 ; 
for ( i = 0 ; i < N_preguntas ; i ++ ) 
{ 
for ( k = 0 ; ( k < N_preguntas ) && strcmp ( versiones [ k_version ] . preguntas [ i ] . codigo , preguntas [ k ] . pregunta ) != 0 ; k ++ ) ; 
switch ( preguntas [ k ] . revision_especial ) 
{ 
case  0 : 
if ( versiones [ k_version ] . preguntas [ i ] . respuesta == respuestas [ i ] ) N_correctas ++ ; 
N_evaluadas ++ ; 
break \end{lstlisting}
\end{frame}
\begin{frame}[fragile]
\frametitle{C\'odigo Preprocesado (Sin Pretty Print)}
\begin{lstlisting}[style=CStyle]
; 
case  1 : 
break ; 
case  3 : 
N_correctas ++ ; 
N_evaluadas ++ ; 
break ; 
case  2 : 
break ; 
case  4 : 
N_evaluadas ++ ; 
break ; 
case  5 : 
if \end{lstlisting}
\end{frame}
\begin{frame}[fragile]
\frametitle{C\'odigo Preprocesado (Sin Pretty Print)}
\begin{lstlisting}[style=CStyle]
( versiones [ k_version ] . preguntas [ i ] . respuesta == respuestas [ i ] ) 
{ 
N_correctas ++ ; 
N_evaluadas ++ ; 
} 
break ; 
case  6 : 
if ( versiones [ k_version ] . preguntas [ i ] . respuesta == respuestas [ i ] ) N_correctas += 2 ; 
N_evaluadas ++ ; 
break ; 
case  7 : 
if ( versiones [ k_version ] . preguntas [ i ] . respuesta == respuestas [ i ] ) N_correctas ++ ; 
break ; 
} \end{lstlisting}
\end{frame}
\begin{frame}[fragile]
\frametitle{C\'odigo Preprocesado (Sin Pretty Print)}
\begin{lstlisting}[style=CStyle]

} 
* n = N_correctas ; 
* m = N_evaluadas ; 
} 
void Cambia_Pregunta ( ) 
{ 
int i , k ; 
GdkColor color ; 
k = ( int ) gtk_range_get_value ( GTK_RANGE ( SC_preguntas ) ) ; 
if ( k && preguntas ) 
{ 
Color_ajustes ( k - 1 ) ; 
gtk_combo_box_set_active \end{lstlisting}
\end{frame}
\begin{frame}[fragile]
\frametitle{C\'odigo Preprocesado (Sin Pretty Print)}
\begin{lstlisting}[style=CStyle]
( CB_ajuste , preguntas [ k - 1 ] . revision_especial ) ; 
if ( preguntas [ k - 1 ] . revision_especial !=  0 ) 
{ 
if ( preguntas [ k - 1 ] . actualizar ) 
gtk_toggle_button_set_active ( CK_no_actualiza , FALSE ) ; 
else 
gtk_toggle_button_set_active ( CK_no_actualiza , TRUE ) ; 
if ( preguntas [ k - 1 ] . revision_especial ==  2 ) 
{ 
if ( preguntas [ k - 1 ] . correctas_nuevas [ 0 ] ) gtk_toggle_button_set_active ( TG_A , TRUE ) ; 
if ( preguntas [ k - 1 ] . correctas_nuevas [ 1 ] ) gtk_toggle_button_set_active ( TG_B , TRUE ) ; 
if ( preguntas [ k - 1 ] . correctas_nuevas [ 2 ] ) gtk_toggle_button_set_active ( TG_C , TRUE ) ; 
if ( preguntas [ k - 1 ] . correctas_nuevas [ 3 ] ) gtk_toggle_button_set_active ( TG_D , TRUE ) ; 
if \end{lstlisting}
\end{frame}
\begin{frame}[fragile]
\frametitle{C\'odigo Preprocesado (Sin Pretty Print)}
\begin{lstlisting}[style=CStyle]
( preguntas [ k - 1 ] . correctas_nuevas [ 4 ] ) gtk_toggle_button_set_active ( TG_E , TRUE ) ; 
} 
} 
gtk_text_buffer_set_text ( buffer_TV_pregunta , preguntas [ k - 1 ] . texto_pregunta , - 1 ) ; 
if ( preguntas [ k - 1 ] . grupo_inicio == preguntas [ k - 1 ] . grupo_final ) 
{ 
gtk_toggle_button_set_active ( CK_header_encoger , FALSE ) ; 
gtk_toggle_button_set_active ( CK_header_verbatim , FALSE ) ; 
gtk_widget_set_sensitive ( GTK_WIDGET ( CK_header_encoger ) , 0 ) ; 
gtk_widget_set_sensitive ( GTK_WIDGET ( CK_header_verbatim ) , 0 ) ; 
} 
else 
{ 
gtk_widget_set_sensitive \end{lstlisting}
\end{frame}
\begin{frame}[fragile]
\frametitle{C\'odigo Preprocesado (Sin Pretty Print)}
\begin{lstlisting}[style=CStyle]
( GTK_WIDGET ( CK_header_encoger ) , 1 ) ; 
gtk_widget_set_sensitive ( GTK_WIDGET ( CK_header_verbatim ) , 1 ) ; 
if ( preguntas [ k - 1 ] . header_encoger ) 
gtk_toggle_button_set_active ( CK_header_encoger , TRUE ) ; 
else 
gtk_toggle_button_set_active ( CK_header_encoger , FALSE ) ; 
if ( preguntas [ k - 1 ] . header_verbatim ) 
gtk_toggle_button_set_active ( CK_header_verbatim , TRUE ) ; 
else 
gtk_toggle_button_set_active ( CK_header_verbatim , FALSE ) ; 
} 
if ( preguntas [ k - 1 ] . excluir ) 
gtk_toggle_button_set_active ( CK_excluir , TRUE ) ; 
else \end{lstlisting}
\end{frame}
\begin{frame}[fragile]
\frametitle{C\'odigo Preprocesado (Sin Pretty Print)}
\begin{lstlisting}[style=CStyle]

gtk_toggle_button_set_active ( CK_excluir , FALSE ) ; 
if ( preguntas [ k - 1 ] . encoger ) 
gtk_toggle_button_set_active ( CK_encoger , TRUE ) ; 
else 
gtk_toggle_button_set_active ( CK_encoger , FALSE ) ; 
if ( preguntas [ k - 1 ] . verbatim ) 
gtk_toggle_button_set_active ( CK_verbatim , TRUE ) ; 
else 
gtk_toggle_button_set_active ( CK_verbatim , FALSE ) ; 
for ( i = 0 ; i < 5 ; i ++ ) 
{ 
if ( preguntas [ k - 1 ] . slide [ i ] ) 
gtk_toggle_button_set_active \end{lstlisting}
\end{frame}
\begin{frame}[fragile]
\frametitle{C\'odigo Preprocesado (Sin Pretty Print)}
\begin{lstlisting}[style=CStyle]
( CK_slide [ i ] , TRUE ) ; 
else 
gtk_toggle_button_set_active ( CK_slide [ i ] , FALSE ) ; 
} 
for ( i = 0 ; i < 5 ; i ++ ) 
{ 
if ( preguntas [ k - 1 ] . encoger_opcion [ i ] ) 
gtk_toggle_button_set_active ( CK_encoger_opcion [ i ] , TRUE ) ; 
else 
gtk_toggle_button_set_active ( CK_encoger_opcion [ i ] , FALSE ) ; 
} 
for ( i = 0 ; i < 5 ; i ++ ) 
{ 
if \end{lstlisting}
\end{frame}
\begin{frame}[fragile]
\frametitle{C\'odigo Preprocesado (Sin Pretty Print)}
\begin{lstlisting}[style=CStyle]
( preguntas [ k - 1 ] . verbatim_opcion [ i ] ) 
gtk_toggle_button_set_active ( CK_verbatim_opcion [ i ] , TRUE ) ; 
else 
gtk_toggle_button_set_active ( CK_verbatim_opcion [ i ] , FALSE ) ; 
} 
} 
} 
void Cambio_Ajuste ( GtkWidget * widget , gpointer user_data ) 
{ 
int j , k , previo ; 
GdkColor color ; 
k = ( int ) gtk_range_get_value ( GTK_RANGE ( SC_preguntas ) ) ; 
j = gtk_combo_box_get_active ( CB_ajuste ) ; 
if \end{lstlisting}
\end{frame}
\begin{frame}[fragile]
\frametitle{C\'odigo Preprocesado (Sin Pretty Print)}
\begin{lstlisting}[style=CStyle]
( k && preguntas ) 
{ 
previo = preguntas [ k - 1 ] . revision_especial ; 
preguntas [ k - 1 ] . revision_especial = j ; 
if ( j !=  2 ) 
{ 
gtk_toggle_button_set_active ( TG_A , FALSE ) ; 
gtk_toggle_button_set_active ( TG_B , FALSE ) ; 
gtk_toggle_button_set_active ( TG_C , FALSE ) ; 
gtk_toggle_button_set_active ( TG_D , FALSE ) ; 
gtk_toggle_button_set_active ( TG_E , FALSE ) ; 
gtk_widget_set_sensitive ( GTK_WIDGET ( TG_A ) , 0 ) ; 
gtk_widget_set_sensitive ( GTK_WIDGET ( TG_B ) , 0 ) ; 
gtk_widget_set_sensitive \end{lstlisting}
\end{frame}
\begin{frame}[fragile]
\frametitle{C\'odigo Preprocesado (Sin Pretty Print)}
\begin{lstlisting}[style=CStyle]
( GTK_WIDGET ( TG_C ) , 0 ) ; 
gtk_widget_set_sensitive ( GTK_WIDGET ( TG_D ) , 0 ) ; 
gtk_widget_set_sensitive ( GTK_WIDGET ( TG_E ) , 0 ) ; 
} 
else 
{ 
gtk_widget_set_sensitive ( GTK_WIDGET ( TG_A ) , 1 ) ; 
gtk_widget_set_sensitive ( GTK_WIDGET ( TG_B ) , 1 ) ; 
gtk_widget_set_sensitive ( GTK_WIDGET ( TG_C ) , 1 ) ; 
gtk_widget_set_sensitive ( GTK_WIDGET ( TG_D ) , 1 ) ; 
gtk_widget_set_sensitive ( GTK_WIDGET ( TG_E ) , 1 ) ; 
} 
if ( j ==  0 ) 
{ \end{lstlisting}
\end{frame}
\begin{frame}[fragile]
\frametitle{C\'odigo Preprocesado (Sin Pretty Print)}
\begin{lstlisting}[style=CStyle]

gtk_toggle_button_set_active ( CK_no_actualiza , FALSE ) ; 
gtk_widget_set_sensitive ( GTK_WIDGET ( CK_no_actualiza ) , 0 ) ; 
preguntas [ k - 1 ] . actualizar = 1 ; 
} 
else 
{ 
gtk_widget_set_sensitive ( GTK_WIDGET ( CK_no_actualiza ) , 1 ) ; 
if ( previo ==  0 ) 
gtk_toggle_button_set_active ( CK_no_actualiza , TRUE ) ; 
else 
{ 
if ( preguntas [ k - 1 ] . actualizar ) 
gtk_toggle_button_set_active \end{lstlisting}
\end{frame}
\begin{frame}[fragile]
\frametitle{C\'odigo Preprocesado (Sin Pretty Print)}
\begin{lstlisting}[style=CStyle]
( CK_no_actualiza , FALSE ) ; 
else 
gtk_toggle_button_set_active ( CK_no_actualiza , TRUE ) ; 
} 
} 
Color_ajustes ( k - 1 ) ; 
} 
N_ajustes = 0 ; 
for ( j = 0 ; j < N_preguntas ; j ++ ) if ( preguntas [ j ] . revision_especial !=  0 ) N_ajustes ++ ; 
} 
void Cambio_no_actualizar ( GtkWidget * widget , gpointer user_data ) 
{ 
int k ; 
k \end{lstlisting}
\end{frame}
\begin{frame}[fragile]
\frametitle{C\'odigo Preprocesado (Sin Pretty Print)}
\begin{lstlisting}[style=CStyle]
= ( int ) gtk_range_get_value ( GTK_RANGE ( SC_preguntas ) ) ; 
if ( k ) 
{ 
if ( gtk_toggle_button_get_active ( CK_no_actualiza ) ) 
preguntas [ k - 1 ] . actualizar = 0 ; 
else 
preguntas [ k - 1 ] . actualizar = 1 ; 
} 
} 
void Cambio_excluir ( GtkWidget * widget , gpointer user_data ) 
{ 
int k ; 
k = ( int ) gtk_range_get_value ( GTK_RANGE ( SC_preguntas ) ) ; 
if \end{lstlisting}
\end{frame}
\begin{frame}[fragile]
\frametitle{C\'odigo Preprocesado (Sin Pretty Print)}
\begin{lstlisting}[style=CStyle]
( k ) 
{ 
if ( gtk_toggle_button_get_active ( CK_excluir ) ) 
preguntas [ k - 1 ] . excluir = 1 ; 
else 
preguntas [ k - 1 ] . excluir = 0 ; 
Color_ajustes ( k - 1 ) ; 
} 
} 
void Cambio_encoger ( GtkWidget * widget , gpointer user_data ) 
{ 
int k ; 
k = ( int ) gtk_range_get_value ( GTK_RANGE ( SC_preguntas ) ) ; 
if \end{lstlisting}
\end{frame}
\begin{frame}[fragile]
\frametitle{C\'odigo Preprocesado (Sin Pretty Print)}
\begin{lstlisting}[style=CStyle]
( k ) 
{ 
if ( gtk_toggle_button_get_active ( CK_encoger ) ) 
preguntas [ k - 1 ] . encoger = 1 ; 
else 
preguntas [ k - 1 ] . encoger = 0 ; 
Color_ajustes ( k - 1 ) ; 
} 
} 
void Cambio_verbatim ( GtkWidget * widget , gpointer user_data ) 
{ 
int k ; 
k = ( int ) gtk_range_get_value ( GTK_RANGE ( SC_preguntas ) ) ; 
if \end{lstlisting}
\end{frame}
\begin{frame}[fragile]
\frametitle{C\'odigo Preprocesado (Sin Pretty Print)}
\begin{lstlisting}[style=CStyle]
( k ) 
{ 
if ( gtk_toggle_button_get_active ( CK_verbatim ) ) 
preguntas [ k - 1 ] . verbatim = 1 ; 
else 
preguntas [ k - 1 ] . verbatim = 0 ; 
Color_ajustes ( k - 1 ) ; 
} 
} 
void Cambio_header_encoger ( GtkWidget * widget , gpointer user_data ) 
{ 
int k , j ; 
k = ( int ) gtk_range_get_value ( GTK_RANGE ( SC_preguntas ) ) ; 
if \end{lstlisting}
\end{frame}
\begin{frame}[fragile]
\frametitle{C\'odigo Preprocesado (Sin Pretty Print)}
\begin{lstlisting}[style=CStyle]
( k ) 
{ 
if ( gtk_toggle_button_get_active ( CK_header_encoger ) ) 
for ( j = preguntas [ k - 1 ] . grupo_inicio ; j <= preguntas [ k - 1 ] . grupo_final ; j ++ ) 
preguntas [ j ] . header_encoger = 1 ; 
else 
for ( j = preguntas [ k - 1 ] . grupo_inicio ; j <= preguntas [ k - 1 ] . grupo_final ; j ++ ) 
preguntas [ j ] . header_encoger = 0 ; 
Color_ajustes ( k - 1 ) ; 
} 
} 
void Cambio_header_verbatim ( GtkWidget * widget , gpointer user_data ) 
{ 
int \end{lstlisting}
\end{frame}
\begin{frame}[fragile]
\frametitle{C\'odigo Preprocesado (Sin Pretty Print)}
\begin{lstlisting}[style=CStyle]
k , j ; 
k = ( int ) gtk_range_get_value ( GTK_RANGE ( SC_preguntas ) ) ; 
if ( k ) 
{ 
if ( gtk_toggle_button_get_active ( CK_header_verbatim ) ) 
for ( j = preguntas [ k - 1 ] . grupo_inicio ; j <= preguntas [ k - 1 ] . grupo_final ; j ++ ) 
preguntas [ j ] . header_verbatim = 1 ; 
else 
for ( j = preguntas [ k - 1 ] . grupo_inicio ; j <= preguntas [ k - 1 ] . grupo_final ; j ++ ) 
preguntas [ j ] . header_verbatim = 0 ; 
Color_ajustes ( k - 1 ) ; 
} 
} 
void \end{lstlisting}
\end{frame}
\begin{frame}[fragile]
\frametitle{C\'odigo Preprocesado (Sin Pretty Print)}
\begin{lstlisting}[style=CStyle]
Cambio_slide ( int i ) 
{ 
int k ; 
k = ( int ) gtk_range_get_value ( GTK_RANGE ( SC_preguntas ) ) ; 
if ( k ) 
{ 
if ( gtk_toggle_button_get_active ( CK_slide [ i ] ) ) 
{ 
preguntas [ k - 1 ] . slide [ i ] = 1 ; 
gtk_widget_set_sensitive ( GTK_WIDGET ( CK_encoger_opcion [ i ] ) , 1 ) ; 
gtk_widget_set_sensitive ( GTK_WIDGET ( CK_verbatim_opcion [ i ] ) , 1 ) ; 
} 
else 
{ \end{lstlisting}
\end{frame}
\begin{frame}[fragile]
\frametitle{C\'odigo Preprocesado (Sin Pretty Print)}
\begin{lstlisting}[style=CStyle]

preguntas [ k - 1 ] . slide [ i ] = 0 ; 
preguntas [ k - 1 ] . encoger_opcion [ i ] = 0 ; 
preguntas [ k - 1 ] . verbatim_opcion [ i ] = 0 ; 
gtk_toggle_button_set_active ( CK_encoger_opcion [ i ] , FALSE ) ; 
gtk_toggle_button_set_active ( CK_verbatim_opcion [ i ] , FALSE ) ; 
gtk_widget_set_sensitive ( GTK_WIDGET ( CK_encoger_opcion [ i ] ) , 0 ) ; 
gtk_widget_set_sensitive ( GTK_WIDGET ( CK_verbatim_opcion [ i ] ) , 0 ) ; 
} 
Color_ajustes ( k - 1 ) ; 
} 
} 
void Cambio_encoger_opcion ( int i ) 
{ \end{lstlisting}
\end{frame}
\begin{frame}[fragile]
\frametitle{C\'odigo Preprocesado (Sin Pretty Print)}
\begin{lstlisting}[style=CStyle]

int k ; 
k = ( int ) gtk_range_get_value ( GTK_RANGE ( SC_preguntas ) ) ; 
if ( k ) 
{ 
if ( gtk_toggle_button_get_active ( CK_encoger_opcion [ i ] ) ) 
preguntas [ k - 1 ] . encoger_opcion [ i ] = 1 ; 
else 
preguntas [ k - 1 ] . encoger_opcion [ i ] = 0 ; 
Color_ajustes ( k - 1 ) ; 
} 
} 
void Cambio_verbatim_opcion ( int i ) 
{ \end{lstlisting}
\end{frame}
\begin{frame}[fragile]
\frametitle{C\'odigo Preprocesado (Sin Pretty Print)}
\begin{lstlisting}[style=CStyle]

int k ; 
k = ( int ) gtk_range_get_value ( GTK_RANGE ( SC_preguntas ) ) ; 
if ( k ) 
{ 
if ( gtk_toggle_button_get_active ( CK_verbatim_opcion [ i ] ) ) 
preguntas [ k - 1 ] . verbatim_opcion [ i ] = 1 ; 
else 
preguntas [ k - 1 ] . verbatim_opcion [ i ] = 0 ; 
Color_ajustes ( k - 1 ) ; 
} 
} 
void Cambio_A ( GtkWidget * widget , gpointer user_data ) 
{ \end{lstlisting}
\end{frame}
\begin{frame}[fragile]
\frametitle{C\'odigo Preprocesado (Sin Pretty Print)}
\begin{lstlisting}[style=CStyle]

int k ; 
k = ( int ) gtk_range_get_value ( GTK_RANGE ( SC_preguntas ) ) ; 
preguntas [ k - 1 ] . correctas_nuevas [ 0 ] = gtk_toggle_button_get_active ( TG_A ) ? 1 : 0 ; 
Color_ajustes ( k - 1 ) ; 
} 
void Cambio_B ( GtkWidget * widget , gpointer user_data ) 
{ 
int k ; 
k = ( int ) gtk_range_get_value ( GTK_RANGE ( SC_preguntas ) ) ; 
preguntas [ k - 1 ] . correctas_nuevas [ 1 ] = gtk_toggle_button_get_active ( TG_B ) ? 1 : 0 ; 
Color_ajustes ( k - 1 ) ; 
} 
void \end{lstlisting}
\end{frame}
\begin{frame}[fragile]
\frametitle{C\'odigo Preprocesado (Sin Pretty Print)}
\begin{lstlisting}[style=CStyle]
Cambio_C ( GtkWidget * widget , gpointer user_data ) 
{ 
int k ; 
k = ( int ) gtk_range_get_value ( GTK_RANGE ( SC_preguntas ) ) ; 
preguntas [ k - 1 ] . correctas_nuevas [ 2 ] = gtk_toggle_button_get_active ( TG_C ) ? 1 : 0 ; 
Color_ajustes ( k - 1 ) ; 
} 
void Cambio_D ( GtkWidget * widget , gpointer user_data ) 
{ 
int k ; 
k = ( int ) gtk_range_get_value ( GTK_RANGE ( SC_preguntas ) ) ; 
preguntas [ k - 1 ] . correctas_nuevas [ 3 ] = gtk_toggle_button_get_active ( TG_D ) ? 1 : 0 ; 
Color_ajustes ( k - 1 ) ; 
} \end{lstlisting}
\end{frame}
\begin{frame}[fragile]
\frametitle{C\'odigo Preprocesado (Sin Pretty Print)}
\begin{lstlisting}[style=CStyle]

void Cambio_E ( GtkWidget * widget , gpointer user_data ) 
{ 
int k ; 
k = ( int ) gtk_range_get_value ( GTK_RANGE ( SC_preguntas ) ) ; 
preguntas [ k - 1 ] . correctas_nuevas [ 4 ] = gtk_toggle_button_get_active ( TG_E ) ? 1 : 0 ; 
Color_ajustes ( k - 1 ) ; 
} 
void Color_ajustes ( int k ) 
{ 
GdkColor color ; 
Calcula_ajuste ( k ) ; 
if ( preguntas [ k ] . ajuste - preguntas [ k ] . revision_especial > 0 ) 
{ \end{lstlisting}
\end{frame}
\begin{frame}[fragile]
\frametitle{C\'odigo Preprocesado (Sin Pretty Print)}
\begin{lstlisting}[style=CStyle]

gdk_color_parse ( RARE_AREA , & color ) ; 
gtk_widget_modify_bg ( EB_formato , GTK_STATE_NORMAL , & color ) ; 
} 
else 
{ 
gdk_color_parse ( SECONDARY_AREA , & color ) ; 
gtk_widget_modify_bg ( EB_formato , GTK_STATE_NORMAL , & color ) ; 
} 
if ( preguntas [ k ] . revision_especial !=  0 ) 
{ 
gdk_color_parse ( RARE_AREA , & color ) ; 
gtk_widget_modify_bg ( EB_ajustes , GTK_STATE_NORMAL , & color ) ; 
} \end{lstlisting}
\end{frame}
\begin{frame}[fragile]
\frametitle{C\'odigo Preprocesado (Sin Pretty Print)}
\begin{lstlisting}[style=CStyle]

else 
{ 
gdk_color_parse ( SECONDARY_AREA , & color ) ; 
gtk_widget_modify_bg ( EB_ajustes , GTK_STATE_NORMAL , & color ) ; 
} 
} 
void Fin_de_Programa ( GtkWidget * widget , gpointer user_data ) 
{ 
PQfinish ( DATABASE ) ; 
gtk_main_quit ( ) ; 
exit ( 0 ) ; 
} 
void \end{lstlisting}
\end{frame}
\begin{frame}[fragile]
\frametitle{C\'odigo Preprocesado (Sin Pretty Print)}
\begin{lstlisting}[style=CStyle]
Fin_Ventana ( GtkWidget * widget , gpointer user_data ) 
{ 
PQfinish ( DATABASE ) ; 
gtk_main_quit ( ) ; 
exit ( 0 ) ; 
} 
void Prepara_Grafico_Normal ( long double media , long double desviacion , long double media_pred , long double desviacion_pred , long double width ) 
{ 
int i ; 
FILE * Archivo_gnuplot ; 
char Hilera_Antes [ 2000 ] , Hilera_Despues [ 2000 ] ; 
char comando [ 2000 ] ; 
Archivo_gnuplot = fopen ( "EX4010-n.gp" , "w" ) ; 
fprintf \end{lstlisting}
\end{frame}
\begin{frame}[fragile]
\frametitle{C\'odigo Preprocesado (Sin Pretty Print)}
\begin{lstlisting}[style=CStyle]
( Archivo_gnuplot , "set term postscript eps enhanced color \"Times\" 12\n" ) ; 
fprintf ( Archivo_gnuplot , "set encoding iso_8859_1\n" ) ; 
fprintf ( Archivo_gnuplot , "set size 0.9, 0.9\n" ) ; 
fprintf ( Archivo_gnuplot , "set grid xtics\n" ) ; 
fprintf ( Archivo_gnuplot , "set output \"EX4010-n.eps\"\n" ) ; 
hilera_GNUPLOT ( "set ylabel \"Proporción\"\n" , Hilera_Despues ) ; 
fprintf ( Archivo_gnuplot , "%s" , Hilera_Despues ) ; 
hilera_GNUPLOT ( "set xlabel \"Nota\"\n" , Hilera_Despues ) ; 
fprintf ( Archivo_gnuplot , "%s" , Hilera_Despues ) ; 
fprintf ( Archivo_gnuplot , "set xrange [%Lf:%Lf]\n" , ( media - width ) - 3.0 , ( media + width ) + 3.0 ) ; 
fprintf ( Archivo_gnuplot , "load \".scripts/stat.inc\"\n" ) ; 
fprintf ( Archivo_gnuplot , "set style fill solid 0.3 border 1\n" ) ; 
fprintf ( Archivo_gnuplot , "set boxwidth 1.0\n" ) ; 
fprintf \end{lstlisting}
\end{frame}
\begin{frame}[fragile]
\frametitle{C\'odigo Preprocesado (Sin Pretty Print)}
\begin{lstlisting}[style=CStyle]
( Archivo_gnuplot , "set arrow from %Lf,0.0 to %Lf,graph(1, 1) linetype 2 lw 4 lc 1 nohead front\n" , media , media ) ; 
fprintf ( Archivo_gnuplot , "set arrow from %Lf,0.0 to %Lf,graph(1, 1) linetype 2 lw 4 lc 1 nohead front\n" , media_pred , media_pred ) ; 
sprintf ( Hilera_Antes , "plot normal (x, %Lf, %Lf) with lines linetype 1 lw 7 lc 3 title \"Predicción\", normal (x, %Lf, %Lf) with lines linetype 1 lw 7 lc 2 title \"Real\"" , 
media_pred , desviacion_pred , media , desviacion ) ; 
hilera_GNUPLOT ( Hilera_Antes , Hilera_Despues ) ; 
fprintf ( Archivo_gnuplot , "%s\n" , Hilera_Despues ) ; 
fclose ( Archivo_gnuplot ) ; 
sprintf ( comando , "%s EX4010-n.gp" , parametros . gnuplot ) ; 
system ( comando ) ; 
sprintf ( comando , "mv EX4010-n.gp %s" , parametros . ruta_gnuplot ) ; 
system ( comando ) ; 
sprintf ( comando , "%s EX4010-n.eps" , parametros . epstopdf ) ; 
system ( comando ) ; 
system \end{lstlisting}
\end{frame}
\begin{frame}[fragile]
\frametitle{C\'odigo Preprocesado (Sin Pretty Print)}
\begin{lstlisting}[style=CStyle]
( "rm EX4010-n.eps" ) ; 
} 
void Prepara_Histograma_Notas ( ) 
{ 
int i , maximo ; 
FILE * Archivo_gnuplot , * Archivo_Datos ; 
char Hilera_Antes [ 2000 ] , Hilera_Despues [ 2000 ] ; 
char comando [ 2000 ] ; 
maximo = 0 ; 
Archivo_Datos = fopen ( "EX4010y.dat" , "w" ) ; 
for ( i = 0 ; i <  10 ; i ++ ) 
{ 
if ( Frecuencias [ i ] ) fprintf ( Archivo_Datos , "%Lf %Lf\n" , ( long double ) i * ( 100.0 / ( long double )  10 ) , ( long double ) Frecuencias [ i ] ) ; 
if \end{lstlisting}
\end{frame}
\begin{frame}[fragile]
\frametitle{C\'odigo Preprocesado (Sin Pretty Print)}
\begin{lstlisting}[style=CStyle]
( Frecuencias [ i ] > maximo ) maximo = Frecuencias [ i ] ; 
} 
fclose ( Archivo_Datos ) ; 
Archivo_gnuplot = fopen ( "EX4010-h.gp" , "w" ) ; 
fprintf ( Archivo_gnuplot , "set term postscript eps enhanced color \"Times\" 12\n" ) ; 
fprintf ( Archivo_gnuplot , "set encoding iso_8859_1\n" ) ; 
fprintf ( Archivo_gnuplot , "set size 0.9, 0.9\n" ) ; 
fprintf ( Archivo_gnuplot , "set grid xtics\n" ) ; 
fprintf ( Archivo_gnuplot , "set output \"EX4010-h.eps\"\n" ) ; 
hilera_GNUPLOT ( "set ylabel \"Cantidad\"\n" , Hilera_Despues ) ; 
fprintf ( Archivo_gnuplot , "%s" , Hilera_Despues ) ; 
hilera_GNUPLOT ( "set xlabel \"Nota\"\n" , Hilera_Despues ) ; 
fprintf ( Archivo_gnuplot , "%s" , Hilera_Despues ) ; 
fprintf \end{lstlisting}
\end{frame}
\begin{frame}[fragile]
\frametitle{C\'odigo Preprocesado (Sin Pretty Print)}
\begin{lstlisting}[style=CStyle]
( Archivo_gnuplot , "set xrange [0.0:100.0]\n" ) ; 
fprintf ( Archivo_gnuplot , "set xtics 10\n" ) ; 
fprintf ( Archivo_gnuplot , "set yrange [0.0:%d]\n" , maximo + 2 ) ; 
fprintf ( Archivo_gnuplot , "set style fill solid 1.0 border -1\n" ) ; 
fprintf ( Archivo_gnuplot , "set boxwidth 8\n" ) ; 
fprintf ( Archivo_gnuplot , "plot \"EX4010y.dat\" with boxes fill lw 3 lc 2 notitle" ) ; 
fclose ( Archivo_gnuplot ) ; 
sprintf ( comando , "%s EX4010-h.gp" , parametros . gnuplot ) ; 
system ( comando ) ; 
sprintf ( comando , "mv EX4010-h.gp %s" , parametros . ruta_gnuplot ) ; 
system ( comando ) ; 
sprintf ( comando , "%s EX4010-h.eps" , parametros . epstopdf ) ; 
system ( comando ) ; 
system \end{lstlisting}
\end{frame}
\begin{frame}[fragile]
\frametitle{C\'odigo Preprocesado (Sin Pretty Print)}
\begin{lstlisting}[style=CStyle]
( "rm EX4010-h.eps" ) ; 
system ( "rm EX4010y.dat" ) ; 
} 
void Prepara_Histograma_Temas ( ) 
{ 
int i , maximo ; 
FILE * Archivo_gnuplot , * Archivo_Datos ; 
char Hilera_Antes [ 2000 ] , Hilera_Despues [ 2000 ] ; 
char PG_command [ 1000 ] ; 
gchar * materia ; 
PGresult * res ; 
char comando [ 2000 ] ; 
materia = gtk_editable_get_chars ( GTK_EDITABLE ( EN_materia ) , 0 , - 1 ) ; 
Archivo_Datos \end{lstlisting}
\end{frame}
\begin{frame}[fragile]
\frametitle{C\'odigo Preprocesado (Sin Pretty Print)}
\begin{lstlisting}[style=CStyle]
= fopen ( "EX4010z.dat" , "w" ) ; 
for ( i = 0 ; i < N_temas ; i ++ ) 
{ 
fprintf ( Archivo_Datos , "%d %Lf %Lf\n" , i , ( long double ) ( ( resumen_tema [ i ] . buenos + resumen_tema [ i ] . malos ) / N_estudiantes ) / N_preguntas * 100.0 , 
( long double ) resumen_tema [ i ] . buenos / ( resumen_tema [ i ] . buenos + resumen_tema [ i ] . malos ) * 100.0 ) ; 
} 
fclose ( Archivo_Datos ) ; 
Archivo_gnuplot = fopen ( "EX4010-t.gp" , "w" ) ; 
fprintf ( Archivo_gnuplot , "set term postscript eps enhanced color \"Times\" 12\n" ) ; 
fprintf ( Archivo_gnuplot , "set encoding iso_8859_1\n" ) ; 
fprintf ( Archivo_gnuplot , "set grid y2tics ytics\n" ) ; 
fprintf ( Archivo_gnuplot , "set output \"EX4010-t.eps\"\n" ) ; 
fprintf ( Archivo_gnuplot , "set key at graph 0.24, 0.85 horizontal samplen 0.1\n" ) ; 
fprintf \end{lstlisting}
\end{frame}
\begin{frame}[fragile]
\frametitle{C\'odigo Preprocesado (Sin Pretty Print)}
\begin{lstlisting}[style=CStyle]
( Archivo_gnuplot , "set style data histogram\n" ) ; 
fprintf ( Archivo_gnuplot , "set style histogram cluster gap 1\n" ) ; 
fprintf ( Archivo_gnuplot , "set style fill solid border -1\n" ) ; 
fprintf ( Archivo_gnuplot , "set boxwidth 0.8\n" ) ; 
fprintf ( Archivo_gnuplot , "set xtics (" ) ; 
for ( i = 0 ; i < N_temas ; i ++ ) 
{ 
sprintf ( PG_command , "SELECT descripcion_materia from BD_materias where codigo_materia = '%s' and codigo_tema = '%s' and codigo_subtema = '          '" , materia , resumen_tema [ i ] . tema ) ; 
res = PQEXEC ( DATABASE , PG_command ) ; 
sprintf ( Hilera_Antes , "\"%.44s (%3d)\" %d" , PQgetvalue ( res , 0 , 0 ) , ( resumen_tema [ i ] . buenos + resumen_tema [ i ] . malos ) / N_estudiantes , i ) ; 
hilera_GNUPLOT ( Hilera_Antes , Hilera_Despues ) ; 
fprintf ( Archivo_gnuplot , "%s" , Hilera_Despues ) ; 
if ( i == ( N_temas - 1 ) ) 
fprintf \end{lstlisting}
\end{frame}
\begin{frame}[fragile]
\frametitle{C\'odigo Preprocesado (Sin Pretty Print)}
\begin{lstlisting}[style=CStyle]
( Archivo_gnuplot , ")\n" ) ; 
else 
fprintf ( Archivo_gnuplot , ", " ) ; 
PQclear ( res ) ; 
} 
fprintf ( Archivo_gnuplot , "set xtics rotate by -270 scale 0\n" ) ; 
fprintf ( Archivo_gnuplot , "set ytics rotate by 90\n" ) ; 
fprintf ( Archivo_gnuplot , "set ytics 5,10\n" ) ; 
fprintf ( Archivo_gnuplot , "set y2tics rotate by 90\n" ) ; 
fprintf ( Archivo_gnuplot , "set y2tics 0,10\n" ) ; 
fprintf ( Archivo_gnuplot , "set yrange [0:100.0]\n" ) ; 
fprintf ( Archivo_gnuplot , "set xlabel \" \"\n" ) ; 
fprintf ( Archivo_gnuplot , "set size 0.63, 2.0\n" ) ; 
fprintf \end{lstlisting}
\end{frame}
\begin{frame}[fragile]
\frametitle{C\'odigo Preprocesado (Sin Pretty Print)}
\begin{lstlisting}[style=CStyle]
( Archivo_gnuplot , "set label 2 \"%% preguntas\" at graph 0.13, 0.85 left rotate by 90\n" ) ; 
fprintf ( Archivo_gnuplot , "set label 3 \"%% correctas\" at graph 0.21, 0.85 left rotate by 90\n" ) ; 
fprintf ( Archivo_gnuplot , "plot \"EX4010z.dat\" using 2 title \" \" lc 3 lt 1, \"\" using 3 title \" \" lc 2 lt 1\n" ) ; 
fclose ( Archivo_gnuplot ) ; 
sprintf ( comando , "%s EX4010-t.gp" , parametros . gnuplot ) ; 
system ( comando ) ; 
sprintf ( comando , "mv EX4010-t.gp %s" , parametros . ruta_gnuplot ) ; 
system ( comando ) ; 
sprintf ( comando , "%s EX4010-t.eps" , parametros . epstopdf ) ; 
system ( comando ) ; 
system ( "rm EX4010-t.eps" ) ; 
system ( "rm EX4010z.dat" ) ; 
g_free ( materia ) ; 
} \end{lstlisting}
\end{frame}
\begin{frame}[fragile]
\frametitle{C\'odigo Preprocesado (Sin Pretty Print)}
\begin{lstlisting}[style=CStyle]

void Prepara_Histograma_Subtemas ( ) 
{ 
int i , maximo ; 
FILE * Archivo_gnuplot , * Archivo_Datos ; 
char Hilera_Antes [ 2000 ] , Hilera_Despues [ 2000 ] ; 
char PG_command [ 1000 ] ; 
gchar * materia ; 
PGresult * res ; 
char comando [ 2000 ] ; 
materia = gtk_editable_get_chars ( GTK_EDITABLE ( EN_materia ) , 0 , - 1 ) ; 
Archivo_Datos = fopen ( "EX4010y.dat" , "w" ) ; 
for ( i = 0 ; i < N_subtemas ; i ++ ) 
{ \end{lstlisting}
\end{frame}
\begin{frame}[fragile]
\frametitle{C\'odigo Preprocesado (Sin Pretty Print)}
\begin{lstlisting}[style=CStyle]

fprintf ( Archivo_Datos , "%d %Lf %Lf\n" , i , ( long double ) ( ( resumen_tema_subtema [ i ] . buenos + resumen_tema_subtema [ i ] . malos ) / N_estudiantes ) / N_preguntas * 100.0 , 
( long double ) resumen_tema_subtema [ i ] . buenos / ( resumen_tema_subtema [ i ] . buenos + resumen_tema_subtema [ i ] . malos ) * 100.0 ) ; 
} 
fclose ( Archivo_Datos ) ; 
Archivo_gnuplot = fopen ( "EX4010-s.gp" , "w" ) ; 
fprintf ( Archivo_gnuplot , "set term postscript eps enhanced color \"Times\" 12\n" ) ; 
fprintf ( Archivo_gnuplot , "set encoding iso_8859_1\n" ) ; 
fprintf ( Archivo_gnuplot , "set grid y2tics ytics\n" ) ; 
fprintf ( Archivo_gnuplot , "set output \"EX4010-s.eps\"\n" ) ; 
fprintf ( Archivo_gnuplot , "set key at graph 0.24, 0.85 horizontal samplen 0.1\n" ) ; 
fprintf ( Archivo_gnuplot , "set style data histogram\n" ) ; 
fprintf ( Archivo_gnuplot , "set style histogram cluster gap 1\n" ) ; 
fprintf \end{lstlisting}
\end{frame}
\begin{frame}[fragile]
\frametitle{C\'odigo Preprocesado (Sin Pretty Print)}
\begin{lstlisting}[style=CStyle]
( Archivo_gnuplot , "set style fill solid border -1\n" ) ; 
fprintf ( Archivo_gnuplot , "set boxwidth 0.8\n" ) ; 
fprintf ( Archivo_gnuplot , "set xtics (" ) ; 
for ( i = 0 ; i < N_subtemas ; i ++ ) 
{ 
sprintf ( PG_command , "SELECT descripcion_materia from BD_materias where codigo_materia = '%s' and codigo_tema = '%s' and codigo_subtema = '%s'" , 
materia , resumen_tema_subtema [ i ] . tema , resumen_tema_subtema [ i ] . subtema ) ; 
res = PQEXEC ( DATABASE , PG_command ) ; 
sprintf ( Hilera_Antes , "\"%.44s (%3d)\" %d" , PQgetvalue ( res , 0 , 0 ) , ( resumen_tema_subtema [ i ] . buenos + resumen_tema_subtema [ i ] . malos ) / N_estudiantes , i ) ; 
hilera_GNUPLOT ( Hilera_Antes , Hilera_Despues ) ; 
fprintf ( Archivo_gnuplot , "%s" , Hilera_Despues ) ; 
if ( i == ( N_subtemas - 1 ) ) 
fprintf ( Archivo_gnuplot , ")\n" ) ; 
else \end{lstlisting}
\end{frame}
\begin{frame}[fragile]
\frametitle{C\'odigo Preprocesado (Sin Pretty Print)}
\begin{lstlisting}[style=CStyle]

fprintf ( Archivo_gnuplot , ", " ) ; 
PQclear ( res ) ; 
} 
fprintf ( Archivo_gnuplot , "set xtics rotate by -270 scale 0\n" ) ; 
fprintf ( Archivo_gnuplot , "set ytics rotate by 90\n" ) ; 
fprintf ( Archivo_gnuplot , "set ytics 5,10\n" ) ; 
fprintf ( Archivo_gnuplot , "set y2tics rotate by 90\n" ) ; 
fprintf ( Archivo_gnuplot , "set y2tics 0,10\n" ) ; 
fprintf ( Archivo_gnuplot , "set yrange [0:100.0]\n" ) ; 
fprintf ( Archivo_gnuplot , "set xlabel \" \"\n" ) ; 
fprintf ( Archivo_gnuplot , "set size 1.15, 2.0\n" ) ; 
fprintf ( Archivo_gnuplot , "set label 2 \"%% preguntas\" at graph 0.17, 0.86 left rotate by 90\n" ) ; 
fprintf \end{lstlisting}
\end{frame}
\begin{frame}[fragile]
\frametitle{C\'odigo Preprocesado (Sin Pretty Print)}
\begin{lstlisting}[style=CStyle]
( Archivo_gnuplot , "set label 3 \"%% correctas\" at graph 0.22, 0.86 left rotate by 90\n" ) ; 
fprintf ( Archivo_gnuplot , "plot \"EX4010y.dat\" using 2 title \"  \" lc 3 lt 1, \"\" using 3 title \" \" lc 2 lt 1\n" ) ; 
fclose ( Archivo_gnuplot ) ; 
sprintf ( comando , "%s EX4010-s.gp" , parametros . gnuplot ) ; 
system ( comando ) ; 
sprintf ( comando , "mv EX4010-s.gp %s" , parametros . ruta_gnuplot ) ; 
system ( comando ) ; 
sprintf ( comando , "%s EX4010-s.eps" , parametros . epstopdf ) ; 
system ( comando ) ; 
system ( "rm EX4010-s.eps" ) ; 
system ( "rm EX4010y.dat" ) ; 
g_free ( materia ) ; 
} 
long \end{lstlisting}
\end{frame}
\begin{frame}[fragile]
\frametitle{C\'odigo Preprocesado (Sin Pretty Print)}
\begin{lstlisting}[style=CStyle]
double CDF ( long double X , long double Media , long double Desv ) 
{ 
return ( 0.5 * ( 1 + ( long double ) erf ( ( X - Media ) / ( Desv * M_SQRT2 ) ) ) ) ; 
} 
void Dificultad_vs_Discriminacion ( ) 
{ 
int i ; 
FILE * Archivo_gnuplot ; 
int Color , Smooth , Rotacion ; 
int Total_preguntas ; 
long double dificultad , discriminacion ; 
int frecuencia ; 
char comando [ 1000 ] ; 
char \end{lstlisting}
\end{frame}
\begin{frame}[fragile]
\frametitle{C\'odigo Preprocesado (Sin Pretty Print)}
\begin{lstlisting}[style=CStyle]
Hilera_Antes [ 2000 ] , Hilera_Despues [ 2000 ] ; 
char * Colores [ ] = { "unset pm3d" , 
"set palette gray" , 
"set palette gray negative" , 
"set palette rgb 21,22,23" , 
"set palette rgb 34,35,36" , 
"set palette rgb 7,5,15" , 
"set palette rgb 3,11,6" , 
"set palette rgb 23,28,3" , 
"set palette rgb 33,13,10" , 
"set palette rgb 30,31,32" } ; 
Color = ( int ) gtk_spin_button_get_value_as_int ( SP_color ) ; 
Rotacion = ( int ) gtk_spin_button_get_value_as_int ( SP_rotacion ) ; 
Smooth \end{lstlisting}
\end{frame}
\begin{frame}[fragile]
\frametitle{C\'odigo Preprocesado (Sin Pretty Print)}
\begin{lstlisting}[style=CStyle]
= gtk_toggle_button_get_active ( CK_smooth ) ; 
Calcular_Tabla ( ) ; 
Archivo_gnuplot = fopen ( "EX4010-p.gp" , "w" ) ; 
fprintf ( Archivo_gnuplot , "set term postscript eps enhanced color \"Times\" 12\n" ) ; 
fprintf ( Archivo_gnuplot , "set encoding iso_8859_1\n" ) ; 
fprintf ( Archivo_gnuplot , "set pm3d\n" ) ; 
fprintf ( Archivo_gnuplot , "set style line 100 lt 5 lw 0.5 lc 2\n" ) ; 
fprintf ( Archivo_gnuplot , "set pm3d hidden3d 100\n" ) ; 
fprintf ( Archivo_gnuplot , "%s\n" , Colores [ Color ] ) ; 
fprintf ( Archivo_gnuplot , "set size 1.55, 1.0\n" ) ; 
if ( Niveles_Discriminacion < 12 ) 
fprintf ( Archivo_gnuplot , "set xtics -1.0, %4.3Lf, 1.0 offset 1\n" , ( long double ) 2.0 / ( Niveles_Discriminacion ) ) ; 
else 
fprintf \end{lstlisting}
\end{frame}
\begin{frame}[fragile]
\frametitle{C\'odigo Preprocesado (Sin Pretty Print)}
\begin{lstlisting}[style=CStyle]
( Archivo_gnuplot , "set xtics -1.0, %4.3Lf, 1.0 offset 1\n" , 2 * ( long double ) 2.0 / ( Niveles_Discriminacion ) ) ; 
fprintf ( Archivo_gnuplot , "set ytics  0.0, %4.3Lf, 1.0 offset 2\n" , ( long double ) 1.0 / ( Niveles_Dificultad ) ) ; 
fprintf ( Archivo_gnuplot , "set grid xtics\n" ) ; 
fprintf ( Archivo_gnuplot , "set grid ytics\n" ) ; 
fprintf ( Archivo_gnuplot , "set contour base\n" ) ; 
fprintf ( Archivo_gnuplot , "set output \"EX4010p.eps\"\n" ) ; 
hilera_GNUPLOT ( "set xlabel \"Discriminación\"\n" , Hilera_Despues ) ; 
fprintf ( Archivo_gnuplot , "%s" , Hilera_Despues ) ; 
fprintf ( Archivo_gnuplot , "set ylabel \"Dificultad\"\n" ) ; 
fprintf ( Archivo_gnuplot , "set xyplane at -0.5\n" ) ; 
fprintf ( Archivo_gnuplot , "set hidden3d\n" ) ; 
if ( Smooth ) fprintf ( Archivo_gnuplot , "set dgrid3 %d,%d,gauss 0.15 0.15\n" , Niveles_Dificultad + 1 , Niveles_Discriminacion + 1 ) ; 
fprintf ( Archivo_gnuplot , "set view 60,%d,1,1\n" , Rotacion ) ; 
fprintf \end{lstlisting}
\end{frame}
\begin{frame}[fragile]
\frametitle{C\'odigo Preprocesado (Sin Pretty Print)}
\begin{lstlisting}[style=CStyle]
( Archivo_gnuplot , "splot \"EX4010p.dat\" title \" \" with lines lt 1 lw 1 lc 2\n" ) ; 
fclose ( Archivo_gnuplot ) ; 
sprintf ( comando , "%s EX4010-p.gp" , parametros . gnuplot ) ; 
system ( comando ) ; 
sprintf ( comando , "mv EX4010-p.gp %s" , parametros . ruta_gnuplot ) ; 
system ( comando ) ; 
sprintf ( comando , "%s EX4010p.eps" , parametros . epstopdf ) ; 
system ( comando ) ; 
system ( "rm EX4010p.eps" ) ; 
Archivo_gnuplot = fopen ( "EX4010c.gp" , "w" ) ; 
fprintf ( Archivo_gnuplot , "set term postscript eps enhanced color \"Times\" 10\n" ) ; 
fprintf ( Archivo_gnuplot , "set encoding iso_8859_1\n" ) ; 
fprintf ( Archivo_gnuplot , "set pm3d\n" ) ; 
fprintf \end{lstlisting}
\end{frame}
\begin{frame}[fragile]
\frametitle{C\'odigo Preprocesado (Sin Pretty Print)}
\begin{lstlisting}[style=CStyle]
( Archivo_gnuplot , "%s\n" , Colores [ Color ] ) ; 
fprintf ( Archivo_gnuplot , "set size 1.55, 1.0\n" ) ; 
fprintf ( Archivo_gnuplot , "set output \"EX4010c.eps\"\n" ) ; 
hilera_GNUPLOT ( "set xlabel \"Discriminación\"\n" , Hilera_Despues ) ; 
fprintf ( Archivo_gnuplot , "%s" , Hilera_Despues ) ; 
fprintf ( Archivo_gnuplot , "set ylabel \"Dificultad\"\n" ) ; 
fprintf ( Archivo_gnuplot , "set view map\n" ) ; 
fprintf ( Archivo_gnuplot , "set contour\n" ) ; 
if ( Niveles_Discriminacion == 18 ) 
fprintf ( Archivo_gnuplot , "set xtics -1.0, %4.3Lf, 1.0\n" , 2 * ( long double ) 2.0 / ( Niveles_Discriminacion ) ) ; 
else 
fprintf ( Archivo_gnuplot , "set xtics -1.0, %4.3Lf, 1.0\n" , ( long double ) 2.0 / ( Niveles_Discriminacion ) ) ; 
fprintf ( Archivo_gnuplot , "set ytics  0.0, %4.3Lf, 1.0\n" , ( long double ) 1.0 / ( Niveles_Dificultad ) ) ; 
fprintf \end{lstlisting}
\end{frame}
\begin{frame}[fragile]
\frametitle{C\'odigo Preprocesado (Sin Pretty Print)}
\begin{lstlisting}[style=CStyle]
( Archivo_gnuplot , "set key at -1.1, -0.07 left box lt 1\n" ) ; 
if ( Smooth ) fprintf ( Archivo_gnuplot , "set dgrid3 %d,%d,gauss 0.15 0.15\n" , Niveles_Dificultad + 1 , Niveles_Discriminacion + 1 ) ; 
fprintf ( Archivo_gnuplot , "splot \"EX4010p.dat\" notitle with lines lt 1 lw 1 lc 2\n" ) ; 
fclose ( Archivo_gnuplot ) ; 
sprintf ( comando , "%s EX4010c.gp" , parametros . gnuplot ) ; 
system ( comando ) ; 
sprintf ( comando , "mv EX4010c.gp %s" , parametros . ruta_gnuplot ) ; 
system ( comando ) ; 
sprintf ( comando , "%s EX4010c.eps" , parametros . epstopdf ) ; 
system ( comando ) ; 
system ( "rm EX4010c.eps" ) ; 
system ( "rm EX4010p.dat" ) ; 
} 
void \end{lstlisting}
\end{frame}
\begin{frame}[fragile]
\frametitle{C\'odigo Preprocesado (Sin Pretty Print)}
\begin{lstlisting}[style=CStyle]
Calcular_Tabla ( ) 
{ 
int i , j , k ; 
long double dificultad , discriminacion ; 
FILE * Archivo_Datos ; 
Niveles_Dificultad = ( int ) gtk_spin_button_get_value_as_int ( SP_resolucion ) ; 
Niveles_Discriminacion = Niveles_Dificultad * 2 ; 
for ( i = 0 ; i <= Niveles_Dificultad ; i ++ ) 
for ( j = 0 ; j <= Niveles_Discriminacion ; j ++ ) 
Frecuencia_total [ i ] [ j ] = 0 ; 
for ( k = 0 ; k < N_preguntas ; k ++ ) 
{ 
dificultad = preguntas [ k ] . porcentaje ; 
discriminacion \end{lstlisting}
\end{frame}
\begin{frame}[fragile]
\frametitle{C\'odigo Preprocesado (Sin Pretty Print)}
\begin{lstlisting}[style=CStyle]
= preguntas [ k ] . Rpb ; 
i = ( int ) round ( dificultad * Niveles_Dificultad ) ; 
j = ( int ) round ( ( discriminacion + 1.0 ) / 2 * Niveles_Discriminacion ) ; 
Frecuencia_total [ i ] [ j ] ++ ; 
} 
Archivo_Datos = fopen ( "EX4010p.dat" , "w" ) ; 
for ( j = 0 ; j <= Niveles_Discriminacion ; j ++ ) 
{ 
for ( i = 0 ; i <= Niveles_Dificultad ; i ++ ) 
{ 
fprintf ( Archivo_Datos , "%Lf %Lf %d\n" , ( long double ) j / Niveles_Discriminacion * 2.0 - 1.0 , ( long double ) i / Niveles_Dificultad , Frecuencia_total [ i ] [ j ] ) ; 
} 
fprintf ( Archivo_Datos , "\n" ) ; 
} \end{lstlisting}
\end{frame}
\begin{frame}[fragile]
\frametitle{C\'odigo Preprocesado (Sin Pretty Print)}
\begin{lstlisting}[style=CStyle]

fclose ( Archivo_Datos ) ; 
} 
void Lista_de_Preguntas_Beamer ( FILE * Archivo_Latex , GtkWidget * PB , long double base , long double limite ) 
{ 
int i , actual ; 
char ejercicio_actual [  7 ] = "00000" ; 
char PG_command [ 3000 ] ; 
PGresult * res , * res_tema ; 
char hilera_antes [ 3000 ] , hilera_despues [ 3000 ] ; 
char tema_actual [ CODIGO_TEMA_SIZE + 1 ] = "          " ; 
gchar * materia ; 
char tema_descripcion [ 201 ] ; 
char \end{lstlisting}
\end{frame}
\begin{frame}[fragile]
\frametitle{C\'odigo Preprocesado (Sin Pretty Print)}
\begin{lstlisting}[style=CStyle]
opciones_frame [ 200 ] ; 
int N_preguntas_ejercicio , N_validas ; 
materia = gtk_editable_get_chars ( GTK_EDITABLE ( EN_materia ) , 0 , - 1 ) ; 
actual = 0 ; 
while ( actual < N_preguntas ) 
{ 
if ( strcmp ( tema_actual , preguntas [ actual ] . tema ) != 0 ) 
{ 
strcpy ( tema_actual , preguntas [ actual ] . tema ) ; 
sprintf ( PG_command , "SELECT descripcion_materia from bd_materias where codigo_materia = '%s' and codigo_tema = '%s' and codigo_subtema = '%s'" , 
materia , tema_actual , CODIGO_VACIO ) ; 
res_tema = PQEXEC ( DATABASE , PG_command ) ; 
strcpy ( tema_descripcion , PQgetvalue ( res_tema , 0 , 0 ) ) ; 
sprintf \end{lstlisting}
\end{frame}
\begin{frame}[fragile]
\frametitle{C\'odigo Preprocesado (Sin Pretty Print)}
\begin{lstlisting}[style=CStyle]
( hilera_antes , "section{%s}" , tema_descripcion ) ; 
hilera_LATEX ( hilera_antes , hilera_despues ) ; 
fprintf ( Archivo_Latex , "%s\n" , hilera_despues ) ; 
PQclear ( res_tema ) ; 
fprintf ( Archivo_Latex , "frame{tableofcontents[currentsection]}\n" ) ; 
} 
strcpy ( ejercicio_actual , preguntas [ actual ] . ejercicio ) ; 
N_preguntas_ejercicio = N_validas = 0 ; 
for ( i = actual ; ( i < N_preguntas ) && ( strcmp ( ejercicio_actual , preguntas [ i ] . ejercicio ) == 0 ) ; i ++ ) 
{ 
if ( ! preguntas [ actual ] . excluir ) N_validas ++ ; 
N_preguntas_ejercicio ++ ; 
} 
sprintf \end{lstlisting}
\end{frame}
\begin{frame}[fragile]
\frametitle{C\'odigo Preprocesado (Sin Pretty Print)}
\begin{lstlisting}[style=CStyle]
( PG_command , "SELECT usa_header, texto_header from bd_texto_ejercicios, bd_ejercicios where codigo_ejercicio = '%s' and texto_ejercicio = consecutivo_texto" , 
ejercicio_actual ) ; 
res = PQEXEC ( DATABASE , PG_command ) ; 
if ( * PQgetvalue ( res , 0 , 0 ) == 't' ) 
{ 
if ( N_validas > 1 ) 
{ 
prepara_opciones ( opciones_frame , actual , - 1 ) ; 
if ( N_validas == 2 ) 
{ 
fprintf ( Archivo_Latex , "begin{frame}%s{Preguntas %d y %d}\n" , opciones_frame , actual + 1 , actual + 2 ) ; 
fprintf ( Archivo_Latex , "textbf{Las preguntas %d y %d requieren la siguiente informaci'{o}n:}\n\n \n\n" , actual + 1 , actual + 2 ) ; 
} 
else \end{lstlisting}
\end{frame}
\begin{frame}[fragile]
\frametitle{C\'odigo Preprocesado (Sin Pretty Print)}
\begin{lstlisting}[style=CStyle]

{ 
fprintf ( Archivo_Latex , "begin{frame}%s{Preguntas %d a %d}\n" , opciones_frame , actual + 1 , actual + N_preguntas_ejercicio ) ; 
fprintf ( Archivo_Latex , "textbf{Las preguntas %d a %d requieren la siguiente informaci'{o}n:}\n\n \n\n" , actual + 1 , actual + N_preguntas_ejercicio ) ; 
} 
fprintf ( Archivo_Latex , "begin{beamercolorbox}[shadow=true, rounded=true]{pregunta}\n" ) ; 
strcpy ( hilera_antes , PQgetvalue ( res , 0 , 1 ) ) ; 
hilera_LATEX ( hilera_antes , hilera_despues ) ; 
fprintf ( Archivo_Latex , "%s\n" , hilera_despues ) ; 
fprintf ( Archivo_Latex , "end{beamercolorbox}\n" ) ; 
fprintf ( Archivo_Latex , "end{frame}\n" ) ; 
} 
else 
{ \end{lstlisting}
\end{frame}
\begin{frame}[fragile]
\frametitle{C\'odigo Preprocesado (Sin Pretty Print)}
\begin{lstlisting}[style=CStyle]

if ( ( N_validas > 0 ) && ! preguntas [ actual ] . excluir ) 
{ 
prepara_opciones ( opciones_frame , actual , - 2 ) ; 
fprintf ( Archivo_Latex , "{\n" ) ; 
Marca_agua_ajuste ( Archivo_Latex , actual ) ; 
sprintf ( hilera_antes , "begin{frame}%s{Pregunta %d hfill {small %s}}" , opciones_frame , actual + 1 , tema_descripcion ) ; 
hilera_LATEX ( hilera_antes , hilera_despues ) ; 
fprintf ( Archivo_Latex , "%s\n" , hilera_despues ) ; 
Imprime_pregunta_Beamer ( actual , Archivo_Latex , PQgetvalue ( res , 0 , 1 ) , tema_descripcion ) ; 
fprintf ( Archivo_Latex , "end{frame}\n" ) ; 
fprintf ( Archivo_Latex , "}\n" ) ; 
} 
N_preguntas_ejercicio \end{lstlisting}
\end{frame}
\begin{frame}[fragile]
\frametitle{C\'odigo Preprocesado (Sin Pretty Print)}
\begin{lstlisting}[style=CStyle]
= 0 ; 
actual ++ ; 
} 
} 
PQclear ( res ) ; 
if ( PB ) Update_PB ( PB , base + ( ( long double ) actual / N_preguntas * limite ) ) ; 
for ( i = 0 ; i < N_preguntas_ejercicio ; i ++ ) 
{ 
if ( ! preguntas [ actual + i ] . excluir ) 
{ 
fprintf ( Archivo_Latex , "{\n" ) ; 
Marca_agua_ajuste ( Archivo_Latex , actual + i ) ; 
if ( N_preguntas_ejercicio > 1 ) 
{ \end{lstlisting}
\end{frame}
\begin{frame}[fragile]
\frametitle{C\'odigo Preprocesado (Sin Pretty Print)}
\begin{lstlisting}[style=CStyle]

prepara_opciones ( opciones_frame , actual + i , - 2 ) ; 
sprintf ( hilera_antes , "begin{frame}%s{Pregunta %d (%d - %d) hfill {small %s}}" , 
opciones_frame , actual + i + 1 , actual + 1 , actual + N_preguntas_ejercicio , tema_descripcion ) ; 
hilera_LATEX ( hilera_antes , hilera_despues ) ; 
fprintf ( Archivo_Latex , "%s\n" , hilera_despues ) ; 
} 
else 
{ 
prepara_opciones ( opciones_frame , actual + i , - 2 ) ; 
sprintf ( hilera_antes , "begin{frame}%s{Pregunta %d hfill {small %s}}" , 
opciones_frame , actual + i + 1 , tema_descripcion ) ; 
hilera_LATEX ( hilera_antes , hilera_despues ) ; 
fprintf \end{lstlisting}
\end{frame}
\begin{frame}[fragile]
\frametitle{C\'odigo Preprocesado (Sin Pretty Print)}
\begin{lstlisting}[style=CStyle]
( Archivo_Latex , "%s\n" , hilera_despues ) ; 
} 
Imprime_pregunta_Beamer ( actual + i , Archivo_Latex , " " , tema_descripcion ) ; 
fprintf ( Archivo_Latex , "end{frame} \%\% CIERRA PREGUNTA\n" ) ; 
fprintf ( Archivo_Latex , "}\n" ) ; 
} 
} 
actual += N_preguntas_ejercicio ; 
if ( PB ) Update_PB ( PB , base + ( ( long double ) actual / N_preguntas * limite ) ) ; 
} 
g_free ( materia ) ; 
} 
void Marca_agua_ajuste ( FILE * Archivo_Latex , int i ) 
{ \end{lstlisting}
\end{frame}
\begin{frame}[fragile]
\frametitle{C\'odigo Preprocesado (Sin Pretty Print)}
\begin{lstlisting}[style=CStyle]

if ( preguntas [ i ] . revision_especial ) 
fprintf ( Archivo_Latex , 
"usebackgroundtemplate{vbox to paperheight{vfilhbox to paperwidth{hfilincludegraphics[width=0.5in]{.imagenes/fix.png}hfil}vfil}}\n" ) ; 
} 
void prepara_opciones ( char * opciones_frame , int i , int k ) 
{ 
opciones_frame [ 0 ] = '\0' ; 
if ( k == - 2 ) 
{ 
if ( preguntas [ i ] . encoger && ! preguntas [ i ] . verbatim ) 
strcpy ( opciones_frame , "[shrink]" ) ; 
else 
if \end{lstlisting}
\end{frame}
\begin{frame}[fragile]
\frametitle{C\'odigo Preprocesado (Sin Pretty Print)}
\begin{lstlisting}[style=CStyle]
( preguntas [ i ] . encoger && preguntas [ i ] . verbatim ) 
strcpy ( opciones_frame , "[shrink, fragile]" ) ; 
else 
if ( ! preguntas [ i ] . encoger && preguntas [ i ] . verbatim ) 
strcpy ( opciones_frame , "[fragile]" ) ; 
} 
else 
if ( k == - 1 ) 
{ 
if ( preguntas [ i ] . header_encoger && ! preguntas [ i ] . header_verbatim ) 
strcpy ( opciones_frame , "[shrink]" ) ; 
else 
if ( preguntas [ i ] . header_encoger && preguntas [ i ] . header_verbatim ) 
strcpy \end{lstlisting}
\end{frame}
\begin{frame}[fragile]
\frametitle{C\'odigo Preprocesado (Sin Pretty Print)}
\begin{lstlisting}[style=CStyle]
( opciones_frame , "[shrink, fragile]" ) ; 
else 
if ( ! preguntas [ i ] . header_encoger && preguntas [ i ] . header_verbatim ) 
strcpy ( opciones_frame , "[fragile]" ) ; 
} 
else 
{ 
if ( preguntas [ i ] . encoger_opcion [ k ] && ! preguntas [ i ] . verbatim_opcion [ k ] ) 
strcpy ( opciones_frame , "[shrink]" ) ; 
else 
if ( preguntas [ i ] . encoger_opcion [ k ] && preguntas [ i ] . verbatim_opcion [ k ] ) 
strcpy ( opciones_frame , "[shrink, fragile]" ) ; 
else 
if \end{lstlisting}
\end{frame}
\begin{frame}[fragile]
\frametitle{C\'odigo Preprocesado (Sin Pretty Print)}
\begin{lstlisting}[style=CStyle]
( ! preguntas [ i ] . encoger_opcion [ k ] && preguntas [ i ] . verbatim_opcion [ k ] ) 
strcpy ( opciones_frame , "[fragile]" ) ; 
} 
} 
void Imprime_pregunta_Beamer ( int i , FILE * Archivo_Latex , char * prefijo , char * tema_descripcion ) 
{ 
long double cota_inferior ; 
char hilera_antes [ 10000 ] , hilera_despues [ 10000 ] ; 
char PG_command [ 2000 ] ; 
PGresult * res , * res_ejercicio ; 
long double Por_A , Por_B , Por_C , Por_D , Por_E , Total ; 
gchar * materia ; 
char opciones_frame [ 200 ] ; 
int \end{lstlisting}
\end{frame}
\begin{frame}[fragile]
\frametitle{C\'odigo Preprocesado (Sin Pretty Print)}
\begin{lstlisting}[style=CStyle]
j , N_flags ; 
materia = gtk_editable_get_chars ( GTK_EDITABLE ( EN_materia ) , 0 , - 1 ) ; 
cota_inferior =  1.15  /  6.65  * 100.0 ; 
sprintf ( PG_command , "SELECT texto_pregunta, texto_opcion_A, texto_opcion_B, texto_opcion_C, texto_opcion_D, texto_opcion_E from bd_texto_preguntas where codigo_unico_pregunta = '%s'" , preguntas [ i ] . pregunta ) ; 
res = PQEXEC ( DATABASE , PG_command ) ; 
fprintf ( Archivo_Latex , "begin{beamercolorbox}[shadow=true, rounded=true]{pregunta}\n" ) ; 
sprintf ( hilera_antes , "%s %s" , prefijo , PQgetvalue ( res , 0 , 0 ) ) ; 
hilera_LATEX ( hilera_antes , hilera_despues ) ; 
fprintf ( Archivo_Latex , "%s\n" , hilera_despues ) ; 
fprintf ( Archivo_Latex , "end{beamercolorbox}\n" ) ; 
Por_A = ( long double ) preguntas [ i ] . acumulado_opciones [ 0 ] / N_estudiantes * 100.0 ; 
Por_B = ( long double ) preguntas [ i ] . acumulado_opciones [ 1 ] / N_estudiantes * 100.0 ; 
Por_C = ( long double ) preguntas [ i ] . acumulado_opciones [ 2 ] / N_estudiantes * 100.0 ; 
Por_D \end{lstlisting}
\end{frame}
\begin{frame}[fragile]
\frametitle{C\'odigo Preprocesado (Sin Pretty Print)}
\begin{lstlisting}[style=CStyle]
= ( long double ) preguntas [ i ] . acumulado_opciones [ 3 ] / N_estudiantes * 100.0 ; 
Por_E = ( long double ) preguntas [ i ] . acumulado_opciones [ 4 ] / N_estudiantes * 100.0 ; 
if ( preguntas [ i ] . slide [ 0 ] ) 
{ 
fprintf ( Archivo_Latex , "end{frame}\n" ) ; 
fprintf ( Archivo_Latex , "}\n" ) ; 
prepara_opciones ( opciones_frame , i , 0 ) ; 
fprintf ( Archivo_Latex , "{\n" ) ; 
Marca_agua_ajuste ( Archivo_Latex , i ) ; 
sprintf ( hilera_antes , "begin{frame}%s{Pregunta %d - cont. hfill {small %s}}" , opciones_frame , i + 1 , tema_descripcion ) ; 
hilera_LATEX ( hilera_antes , hilera_despues ) ; 
fprintf ( Archivo_Latex , "%s\n" , hilera_despues ) ; 
} 
fprintf \end{lstlisting}
\end{frame}
\begin{frame}[fragile]
\frametitle{C\'odigo Preprocesado (Sin Pretty Print)}
\begin{lstlisting}[style=CStyle]
( Archivo_Latex , "begin{itemize}\n" ) ; 
Imprime_Opcion_Beamer ( Archivo_Latex , res , Por_A , i , 0 ) ; 
if ( preguntas [ i ] . slide [ 1 ] ) 
{ 
fprintf ( Archivo_Latex , "end{itemize}\n" ) ; 
fprintf ( Archivo_Latex , "end{frame}\n" ) ; 
fprintf ( Archivo_Latex , "}\n" ) ; 
prepara_opciones ( opciones_frame , i , 1 ) ; 
fprintf ( Archivo_Latex , "{\n" ) ; 
Marca_agua_ajuste ( Archivo_Latex , i ) ; 
sprintf ( hilera_antes , "begin{frame}%s{Pregunta %d - cont. hfill {small %s}}" , opciones_frame , i + 1 , tema_descripcion ) ; 
hilera_LATEX ( hilera_antes , hilera_despues ) ; 
fprintf ( Archivo_Latex , "%s\n" , hilera_despues ) ; 
fprintf \end{lstlisting}
\end{frame}
\begin{frame}[fragile]
\frametitle{C\'odigo Preprocesado (Sin Pretty Print)}
\begin{lstlisting}[style=CStyle]
( Archivo_Latex , "begin{itemize}\n" ) ; 
} 
Imprime_Opcion_Beamer ( Archivo_Latex , res , Por_B , i , 1 ) ; 
if ( preguntas [ i ] . slide [ 2 ] ) 
{ 
fprintf ( Archivo_Latex , "end{itemize}\n" ) ; 
fprintf ( Archivo_Latex , "end{frame}\n" ) ; 
fprintf ( Archivo_Latex , "}\n" ) ; 
prepara_opciones ( opciones_frame , i , 2 ) ; 
fprintf ( Archivo_Latex , "{\n" ) ; 
Marca_agua_ajuste ( Archivo_Latex , i ) ; 
sprintf ( hilera_antes , "begin{frame}%s{Pregunta %d - cont. hfill {small %s}}" , opciones_frame , i + 1 , tema_descripcion ) ; 
hilera_LATEX ( hilera_antes , hilera_despues ) ; 
fprintf \end{lstlisting}
\end{frame}
\begin{frame}[fragile]
\frametitle{C\'odigo Preprocesado (Sin Pretty Print)}
\begin{lstlisting}[style=CStyle]
( Archivo_Latex , "%s\n" , hilera_despues ) ; 
fprintf ( Archivo_Latex , "begin{itemize}\n" ) ; 
} 
Imprime_Opcion_Beamer ( Archivo_Latex , res , Por_C , i , 2 ) ; 
if ( preguntas [ i ] . slide [ 3 ] ) 
{ 
fprintf ( Archivo_Latex , "end{itemize}\n" ) ; 
fprintf ( Archivo_Latex , "end{frame}\n" ) ; 
fprintf ( Archivo_Latex , "}\n" ) ; 
prepara_opciones ( opciones_frame , i , 3 ) ; 
fprintf ( Archivo_Latex , "{\n" ) ; 
Marca_agua_ajuste ( Archivo_Latex , i ) ; 
sprintf ( hilera_antes , "begin{frame}%s{Pregunta %d - cont. hfill {small %s}}" , opciones_frame , i + 1 , tema_descripcion ) ; 
hilera_LATEX \end{lstlisting}
\end{frame}
\begin{frame}[fragile]
\frametitle{C\'odigo Preprocesado (Sin Pretty Print)}
\begin{lstlisting}[style=CStyle]
( hilera_antes , hilera_despues ) ; 
fprintf ( Archivo_Latex , "%s\n" , hilera_despues ) ; 
fprintf ( Archivo_Latex , "begin{itemize}\n" ) ; 
} 
Imprime_Opcion_Beamer ( Archivo_Latex , res , Por_D , i , 3 ) ; 
if ( preguntas [ i ] . slide [ 4 ] ) 
{ 
fprintf ( Archivo_Latex , "end{itemize}\n" ) ; 
fprintf ( Archivo_Latex , "end{frame}\n" ) ; 
fprintf ( Archivo_Latex , "}\n" ) ; 
prepara_opciones ( opciones_frame , i , 4 ) ; 
fprintf ( Archivo_Latex , "{\n" ) ; 
Marca_agua_ajuste ( Archivo_Latex , i ) ; 
sprintf \end{lstlisting}
\end{frame}
\begin{frame}[fragile]
\frametitle{C\'odigo Preprocesado (Sin Pretty Print)}
\begin{lstlisting}[style=CStyle]
( hilera_antes , "begin{frame}%s{Pregunta %d - cont. hfill {small %s}}" , opciones_frame , i + 1 , tema_descripcion ) ; 
hilera_LATEX ( hilera_antes , hilera_despues ) ; 
fprintf ( Archivo_Latex , "%s\n" , hilera_despues ) ; 
fprintf ( Archivo_Latex , "begin{itemize}\n" ) ; 
} 
Imprime_Opcion_Beamer ( Archivo_Latex , res , Por_E , i , 4 ) ; 
fprintf ( Archivo_Latex , "end{itemize}\n" ) ; 
PQclear ( res ) ; 
N_flags = 0 ; 
for ( j = 0 ; j <  16 ; j ++ ) N_flags += preguntas [ i ] . flags [ j ] ; 
if ( N_flags && ! gtk_toggle_button_get_active ( CK_sin_banderas ) ) 
{ 
fprintf ( Archivo_Latex , "end{frame}\n" ) ; 
fprintf \end{lstlisting}
\end{frame}
\begin{frame}[fragile]
\frametitle{C\'odigo Preprocesado (Sin Pretty Print)}
\begin{lstlisting}[style=CStyle]
( Archivo_Latex , "}\n" ) ; 
fprintf ( Archivo_Latex , "{\n" ) ; 
sprintf ( hilera_antes , "begin{frame}{Análisis de Pregunta %d hfill {small %s}}" , i + 1 , tema_descripcion ) ; 
hilera_LATEX ( hilera_antes , hilera_despues ) ; 
fprintf ( Archivo_Latex , "%s\n" , hilera_despues ) ; 
Analiza_Ajuste ( Archivo_Latex , preguntas [ i ] , 1 ) ; 
Analiza_Banderas ( Archivo_Latex , preguntas [ i ] , 1 , N_flags , i + 1 , tema_descripcion ) ; 
} 
g_free ( materia ) ; 
} 
void Imprime_Opcion_Beamer ( FILE * Archivo_Latex , PGresult * res , long double Porcentaje , int pregunta , int opcion ) 
{ 
char hilera_antes [ 3000 ] , hilera_despues [ 3000 ] ; 
if \end{lstlisting}
\end{frame}
\begin{frame}[fragile]
\frametitle{C\'odigo Preprocesado (Sin Pretty Print)}
\begin{lstlisting}[style=CStyle]
( preguntas [ pregunta ] . correcta == ( 'A' + opcion ) ) 
fprintf ( Archivo_Latex , "item [correcta{%c}]" , 'A' + opcion ) ; 
else 
fprintf ( Archivo_Latex , "item [circled{%c}]" , 'A' + opcion ) ; 
strcpy ( hilera_antes , PQgetvalue ( res , 0 , 1 + opcion ) ) ; 
hilera_LATEX ( hilera_antes , hilera_despues ) ; 
fprintf ( Archivo_Latex , "%s" , hilera_despues ) ; 
fprintf ( Archivo_Latex , "\n\n{color{green} rule{%Lfcm}{5pt} {footnotesize textbf{texttt{%6.2Lf %%}}}} (textbf{%d}) hfill " , 
0.05 +  6.65  * Porcentaje / 100.0 , Porcentaje , 
preguntas [ pregunta ] . acumulado_opciones [ opcion ] ) ; 
if ( preguntas [ pregunta ] . correcta == ( 'A' + opcion ) ) 
{ 
if ( preguntas [ pregunta ] . Rpb_opcion [ opcion ] < 0.0 ) 
fprintf \end{lstlisting}
\end{frame}
\begin{frame}[fragile]
\frametitle{C\'odigo Preprocesado (Sin Pretty Print)}
\begin{lstlisting}[style=CStyle]
( Archivo_Latex , "fcolorbox{black}{red}{color{white} $r_{pb}$ = textbf{%7.4Lf}}\n\n" , preguntas [ pregunta ] . Rpb_opcion [ opcion ] ) ; 
else 
if ( preguntas [ pregunta ] . Rpb_opcion [ opcion ] >= 0.3 ) 
fprintf ( Archivo_Latex , "fcolorbox{black}{blue}{color{white} $r_{pb}$ = textbf{%7.4Lf}}\n\n" , preguntas [ pregunta ] . Rpb_opcion [ opcion ] ) ; 
else 
if ( preguntas [ pregunta ] . Rpb_opcion [ opcion ] >= 0.2 ) 
fprintf ( Archivo_Latex , "fcolorbox{black}{cyan}{color{white} $r_{pb}$ = textbf{%7.4Lf}}\n\n" , preguntas [ pregunta ] . Rpb_opcion [ opcion ] ) ; 
else 
if ( preguntas [ pregunta ] . acumulado_opciones [ opcion ] == 0 ) 
fprintf ( Archivo_Latex , "fcolorbox{black}{red}{color{white} $r_{pb}$ = textbf{%7.4Lf}}\n\n" , preguntas [ pregunta ] . Rpb_opcion [ opcion ] ) ; 
else 
if ( preguntas [ pregunta ] . Rpb_opcion [ opcion ] == 0.0 ) 
fprintf ( Archivo_Latex , "fcolorbox{black}{orange}{color{white} $r_{pb}$ = textbf{%7.4Lf}}\n\n" , preguntas [ pregunta ] . Rpb_opcion [ opcion ] ) ; 
else \end{lstlisting}
\end{frame}
\begin{frame}[fragile]
\frametitle{C\'odigo Preprocesado (Sin Pretty Print)}
\begin{lstlisting}[style=CStyle]

fprintf ( Archivo_Latex , "fcolorbox{black}{white}{color{black} $r_{pb}$ = textbf{%7.4Lf}}\n\n" , preguntas [ pregunta ] . Rpb_opcion [ opcion ] ) ; 
fprintf ( Archivo_Latex , "setlength{fboxrule}{2fboxrule}\n" ) ; 
fprintf ( Archivo_Latex , "\n\nfcolorbox{black}{red}{color{white} $starstarstar$ textbf{CORRECTA} $starstarstar$}" ) ; 
fprintf ( Archivo_Latex , "setlength{fboxrule}{0.5fboxrule}\n" ) ; 
} 
else 
{ 
if ( preguntas [ pregunta ] . Rpb_opcion [ opcion ] > 0.3 ) 
fprintf ( Archivo_Latex , "fcolorbox{black}{red}{color{white} $r_{pb}$ = textbf{%7.4Lf}}\n\n" , preguntas [ pregunta ] . Rpb_opcion [ opcion ] ) ; 
else 
if ( preguntas [ pregunta ] . Rpb_opcion [ opcion ] > 0.0 ) 
fprintf ( Archivo_Latex , "fcolorbox{black}{orange}{color{white} $r_{pb}$ = textbf{%7.4Lf}}\n\n" , preguntas [ pregunta ] . Rpb_opcion [ opcion ] ) ; 
else \end{lstlisting}
\end{frame}
\begin{frame}[fragile]
\frametitle{C\'odigo Preprocesado (Sin Pretty Print)}
\begin{lstlisting}[style=CStyle]

if ( preguntas [ pregunta ] . Rpb_opcion [ opcion ] <= - 0.3 ) 
fprintf ( Archivo_Latex , "fcolorbox{black}{blue}{color{white} $r_{pb}$ = textbf{%7.4Lf}}\n\n" , preguntas [ pregunta ] . Rpb_opcion [ opcion ] ) ; 
else 
fprintf ( Archivo_Latex , "fcolorbox{black}{white}{color{black} $r_{pb}$ = textbf{%7.4Lf}}\n\n" , preguntas [ pregunta ] . Rpb_opcion [ opcion ] ) ; 
} 
fprintf ( Archivo_Latex , "\n\n" ) ; 
} 
void Busca_pregunta_mala_Beamer ( int N ) 
{ 
int i ; 
int Exclusiones_previas [ N ] ; 
int low , high , mid ; 
int \end{lstlisting}
\end{frame}
\begin{frame}[fragile]
\frametitle{C\'odigo Preprocesado (Sin Pretty Print)}
\begin{lstlisting}[style=CStyle]
resultado ; 
char mensaje_error [ 2000 ] ; 
gtk_widget_show ( window6 ) ; 
Update_PB ( PB_revisando_beamer , 0.0 ) ; 
for ( i = 0 ; i < N ; i ++ ) Exclusiones_previas [ i ] = preguntas [ i ] . excluir ; 
low = 0 ; 
high = N - 1 ; 
while ( high != low ) 
{ 
mid = ( high + low ) / 2 ; 
for ( i = 0 ; i < N ; i ++ ) preguntas [ i ] . excluir = 1 ; 
for ( i = low ; i <= mid ; i ++ ) preguntas [ i ] . excluir = 0 ; 
for ( i = low ; i <= mid ; i ++ ) preguntas [ i ] . excluir = Exclusiones_previas [ i ] ; 
resultado \end{lstlisting}
\end{frame}
\begin{frame}[fragile]
\frametitle{C\'odigo Preprocesado (Sin Pretty Print)}
\begin{lstlisting}[style=CStyle]
= Genera_Beamer_reducido ( ) ; 
Update_PB ( PB_revisando_beamer , ( long double ) ( N_preguntas - ( high - mid ) ) / N_preguntas ) ; 
if ( resultado ) 
{ 
low = mid + 1 ; 
} 
else 
{ 
high = mid ; 
} 
} 
sprintf ( mensaje_error , "La pregunta número <b>%2d</b> de este examen tiene\nalgún conflicto con <b><i>Beamer</i></b>. Esta corresponde\na la pregunta <b>%s.%d</b> de la Base de Datos.\n\nRevise detalladamente la sintaxis de\nesta pregunta. Considere marcar ya\nsea la casilla \"<b>verbatim</b>\" o la casilla\n\"<b>Excluir</b>\" en la pregunta <b>%d</b>." , 
low + 1 , preguntas [ low ] . ejercicio , preguntas [ low ] . secuencia , low + 1 ) ; 
gtk_label_set_markup \end{lstlisting}
\end{frame}
\begin{frame}[fragile]
\frametitle{C\'odigo Preprocesado (Sin Pretty Print)}
\begin{lstlisting}[style=CStyle]
( LB_error_encontrado_Beamer , mensaje_error ) ; 
for ( i = 0 ; i < N ; i ++ ) preguntas [ i ] . excluir = Exclusiones_previas [ i ] ; 
gtk_widget_hide ( window6 ) ; 
gtk_widget_show ( window7 ) ; 
} 
int Genera_Beamer_reducido ( ) 
{ 
int k ; 
char codigo [ 10 ] ; 
int resultado_OK = 0 ; 
FILE * Archivo_Latex ; 
gchar * size , * font , * color , * estilo , * materia , * fecha ; 
int aspecto ; 
char \end{lstlisting}
\end{frame}
\begin{frame}[fragile]
\frametitle{C\'odigo Preprocesado (Sin Pretty Print)}
\begin{lstlisting}[style=CStyle]
Directorio [ 3000 ] ; 
Banderas [ 0 ] = Banderas [ 1 ] = Banderas [ 2 ] = Banderas [ 3 ] = Banderas [ 4 ] = Banderas [ 5 ] = 0 ; 
k = ( int ) gtk_spin_button_get_value_as_int ( SP_examen ) ; 
sprintf ( codigo , "%05d" , k ) ; 
materia = gtk_editable_get_chars ( GTK_EDITABLE ( EN_materia ) , 0 , - 1 ) ; 
fecha = gtk_editable_get_chars ( GTK_EDITABLE ( EN_fecha ) , 0 , - 1 ) ; 
estilo = gtk_combo_box_get_active_text ( CB_estilo ) ; 
color = gtk_combo_box_get_active_text ( CB_color ) ; 
font = gtk_combo_box_get_active_text ( CB_font ) ; 
size = gtk_combo_box_get_active_text ( CB_size ) ; 
aspecto = gtk_combo_box_get_active ( CB_aspecto ) ; 
Establece_Directorio ( Directorio , materia , fecha + 6 , fecha + 3 , fecha ) ; 
Archivo_Latex = fopen ( "analisis-beamer.tex" , "w" ) ; 
Beamer_Preamble_reducido \end{lstlisting}
\end{frame}
\begin{frame}[fragile]
\frametitle{C\'odigo Preprocesado (Sin Pretty Print)}
\begin{lstlisting}[style=CStyle]
( Archivo_Latex , aspecto , size , estilo , color , font ) ; 
Beamer_Preguntas ( Archivo_Latex , NULL , 0.0 , 0.0 ) ; 
Beamer_Cierre ( Archivo_Latex ) ; 
fclose ( Archivo_Latex ) ; 
resultado_OK = latex_2_pdf ( & parametros , Directorio , parametros . ruta_latex , "analisis-beamer" , 0 , NULL , 0.0 , 0.0 , NULL , NULL ) ; \end{lstlisting}
\end{frame}
\end{document}
